\section{Lecture 14 -- 2nd June 2025}\label{sec: lecture 14}
We make preparations towards the proof of the theorem of formal functions, which allows the computation of stalks of derived pushforwards as limit of cohomology along thickened neighborhoods. 

To wit, the tools of analytic geometry, in particular the idea of functions in a small open neighborhood of a point given by power series, can be captured in the setting of algebraic geometry using complete rings. 
\begin{definition}[Graded Construction]\label{def: graded module}
    Let $A$ be a Noetherian ring and $M$ an $A$-module with decreasing filtration
    \begin{equation}\label{eqn: decreasing filtration}
        \dots\subsetneq M_{n}\subsetneq\dots\subsetneq M_{1}\subsetneq M_{0}=M.
    \end{equation}
    The graded construction $\gr^{\bullet}(M)$ of $M$ is the direct sum $\bigoplus_{0\leq i\leq n-1} M_{i}/M_{i+1}$.
\end{definition}
\begin{remark}\label{rmk: associated graded}
    Evidently we have by definition $\gr^{i}(M)\cong M_{i}/M_{i+1}$. 
\end{remark}
\begin{definition}[Completion Along Filtration]\label{def: completion along filtration}
    Let $A$ be a Noetherian ring and $M$ an $A$-module with decreasing filtration (\ref{eqn: decreasing filtration}). The completion of $M$ along this filtration is the limit $\widehat{M}=\lim_{i}M/M_{i}$. 
\end{definition}
The filtration induces a topology on $M$ with basis given by the cosets $x+M_{i}$ for $x\in M,i\in\NN$ which is Hausdorff if and only if $\bigcap_{i\geq0}M_{i}=0$. 
\begin{definition}[Equivalent Filtrations]\label{def: equivalent filtrations}
    Let $A$ be a Noetherian ring and $M$ an $A$-module. 
    \begin{equation}\label{eqn: filtration 1}
        \dots\subsetneq M_{n}\subsetneq \dots\subsetneq M_{1}\subseteq M_{0}=M
    \end{equation} 
    \begin{equation}\label{eqn: filtration 2}
        \dots\subsetneq M_{n}'\subsetneq \dots\subsetneq M_{1}'\subseteq M_{0}'=M
    \end{equation}
    be two filtrations of $M$. The filtrations (\ref{eqn: filtration 1}) and (\ref{eqn: filtration 2}) are equivalent if the systems are final in each other -- for every $M_{n}$ there exists $M_{m}'$ such that $M_{m}'\subseteq M_{n}$ and for each $M_{n}'$ there exists $M_{m}$ such that $M_{m}\subseteq M_{n}'$. 
\end{definition}
\begin{remark}
    Any two equivalent filtrations in the sense of \Cref{def: equivalent filtrations} induce the same topology on the module $M$. 
\end{remark}
In what follows, we will consider $I\subseteq A$ and $M_{d}=I^{d}M$. 
\begin{example}
    If $M=A$, the induced topology on $A$ (as an $A$-module) is the $I$-adic topology. 
\end{example}
\begin{example}
    Let $M=A=k[x]$. There is a filtration of $A$ by $I=(x)$. Two elements of $A$ being close in the $I$-adic filtration imply that the coefficients of these polynomials agree in small degree, that is, that the polynomials are the same close to $0\in k$. 
\end{example}
Let $A$ be a Noetherian ring, $M$ a finite $A$-module, and $N\subseteq M$ an $A$-module. We seek to understand the filtration $\{I^{d}N\}_{d\geq0}$ in terms of the filtration $\{I^{d}M\cap N\}_{d\geq0}$. The Artin-Rees lemma gives us an equivalence beteween these two filtrations. 
\begin{theorem}[Artin-Rees Lemma]\label{thm: Artin-Rees}
    Let $A$ be a Noetherian ring, $M$ a finitely generated $A$-module, $N\subseteq M$ a submodule, and $I\subseteq A$ an ideal. The filtrations $\{I^{d}M\}_{d\geq0}$ and $\{I^{d}M\cap N\}_{d\geq0}$ are equivalent as filtrations. 
\end{theorem}
\begin{proof}
    For any $d$, we have the containment $I^{d}N\subseteq I^{d}M\cap N$. By \Cref{def: equivalent filtrations}, it suffices to find $m\geq0$ such that $I^{m}M\cap N\subseteq I^{d}N$. For any $c\geq0$ we have that $I^{c}M\cap N\subseteq N$ so it suffices to find $c\geq0$ such that $I^{d+c}M\cap N\subseteq I^{d}(I^{c}M\cap N)$ in which case equality will hold. 
    
    Consider the graded ring $S=\bigoplus_{k\geq0}I^{k}$ and the $S$-module $\widetilde{M}=\bigoplus_{k\geq0}I^{k}M$. Since $I$ is finitely generated, $S$ is a Noetherian ring being the quotient of a finitely-generated $A$-algebra, and $\widetilde{M}$ is finite over $S$ being generated by the generators of $M$ over $A$. The submodule $\widetilde{N}=\bigoplus_{k\geq0}(I^{k}M\cap N)$ of $\widetilde{M}$ is finitely generated over $S$ as well. Let $x_{1},\dots,x_{r}$ be generators of $\widetilde{N}$ with $x_{j}$ in some $I^{k_{j}}M\cap N$. By the finite generation hypothesis, we can take $c$ such that $k_{j}<c$ for all $1\leq j\leq r$. By construction, each $x\in I^{d+c}M\cap N$ can be written as $\sum_{j=1}^{r}\alpha_{j}x_{j}$ with $\alpha_{j}\in I^{d+c-k_{j}}$. This shows $x\in I^{d}(I^{c}M\cap N)$, whence the claim. 
\end{proof}
This ``commutativity'' of intersections with submodules allows us to show that rings map injectively into their completions. 
\begin{theorem}[Krull -- Intersection]\label{thm: Krull intersection}
    Let $A$ be a Noetherian local ring with maximal ideal $\mfrak$. Then $\bigcap_{d\geq0}\mfrak^{d}=0$. 
\end{theorem}
\begin{proof}
    Let $I=\bigcap_{d\geq0}\mfrak^{d}$. We have that $I=\mfrak^{d}\cap I$ for every $d\geq0$. By the Artin-Rees lemma \Cref{thm: Artin-Rees}, there exists $k\geq0$ such that $I=\mfrak^{k}\cap I\subseteq\mfrak I$. Thus $\mfrak I=I$. By Nakayama's lemma, $I=0$. 
\end{proof}
We immediately deduce the following. 
\begin{corollary}\label{corr: injective map into completion}
    Let $A$ be a Noetherian local ring with maximal ideal $\mfrak$. Then $A\to\widehat{A}$ is injective. 
\end{corollary}
\begin{proof}
    The kernel of the map $A\to\widehat{A}$ is given by $\bigcap_{d\geq0}\mfrak^{d}$ -- the intersection of the kernels of the component maps $A\mapsto A/\mfrak^{d}$ in the system 
    $$% https://q.uiver.app/#q=WzAsNSxbMCwwLCJBIl0sWzEsMSwiQS9cXG1mcmFrXnszfSJdLFsyLDEsIkEvXFxtZnJha157Mn0iXSxbMywxLCJBL1xcbWZyYWsiXSxbMCwxLCJcXGRvdHMiXSxbMCwxXSxbMCwyXSxbMSwyXSxbMiwzXSxbNCwxXSxbMCwzXV0=
    \begin{tikzcd}
        A \\
        \dots & {A/\mfrak^{3}} & {A/\mfrak^{2}} & {A/\mfrak}
        \arrow[from=1-1, to=2-2]
        \arrow[from=1-1, to=2-3]
        \arrow[from=1-1, to=2-4]
        \arrow[from=2-1, to=2-2]
        \arrow[from=2-2, to=2-3]
        \arrow[from=2-3, to=2-4]
    \end{tikzcd}$$
    By \Cref{thm: Krull intersection}, the kernel is trivial, showing the map is injective. 
\end{proof}
We collect some important properties of complete rings. 
\begin{proposition}\label{prop: completions of rings}
    Let $A$ be a Noetherian ring, $M$ an $A$-module, and $I\subseteq A$ an ideal. 
    \begin{enumerate}[label=(\roman*)]
        \item The functor $\Mod_{A}\to\Mod_{\widehat{A}}$ by $M\mapsto \widehat{M}$ is exact on finitely generated modules. 
        \item If $M$ is a finitely generated $A$-module, then $M\otimes_{A}\widehat{A}\to\widehat{M}$ is an isomorphism. 
        \item $\widehat{A}$ is flat over $A$. 
        \item $\widehat{A}$ is Noetherian. 
        \item If $A$ is a further a local ring, then $\widehat{A}$ with maximal ideal $\widehat{\mfrak}$ is a Noetherian local ring and $\gr_{\mfrak}^{\bullet}(A)\cong\gr_{\widehat{\mfrak}}^{\bullet}(\widehat{A})$. 
        \item Let $A$ be a local Noetherian $k=A/\mfrak$-algebra of dimension $d$. The following are equivalent: 
        \begin{enumerate}[label=(\alph*)]
            \item $A$ is regular. 
            \item The local ring $\widehat{A}$ with maximal ideal $\widehat{\mfrak}$ of (v) is regular. 
            \item $\gr_{\mfrak}^{\bullet}(A)\cong\gr_{\widehat{\mfrak}}^{\bullet}(\widehat{A})=k[x_{1},\dots,x_{d}]$. 
        \end{enumerate}
        \item Let $A$ be a local Noetherian $k=A/\mfrak$-algebra of dimension $d$. Then $\widehat{A}\cong k[[x_{1},\dots,x_{d}]]$. 
    \end{enumerate}
\end{proposition}
\begin{proof}[Proof of (i)]
    If $M$ is finitely generated, there is a surjection $A^{\oplus r}\to M$ yielding a surjection $\widehat{A}^{\oplus r}\to\widehat{M}$ showing that each $\widehat{M}$ is finitely generated over $\widehat{A}$. For $0\to M_{1}\to M_{2}\to M_{3}\to0$ a short exact sequence of $A$-modules, we get a short exact sequence 
    $$0\to I^{d}M_{2}\cap M_{1}\to I^{d}M_{2}\to I^{d}M_{3}\to0$$
    where applying the Artin-Rees lemma \Cref{thm: Artin-Rees} and on passage to the limit we get the exact sequence $0\to\widehat{M_{1}}\to\widehat{M_{2}}\to\widehat{M_{3}}$ since the limit need not preserve exactness on the right. It thus remains to prove exactness on the right. Let $x=(m_{1},m_{2},\dots)\in\widehat{M_{3}}$ where $m_{d}\in M_{3}/I^{d}M_{3}$. By surjectivity of the map $\varphi:M_{2}\to M_{3}$, we have $m_{d}=\varphi(m_{d}')$ for some $m_{d}'\in M_{2}/I^{d}M_{2}$. Denote $\pi:M_{1}/(I^{2}M_{2}\cap M_{1})\to M_{1}/(IM_{2}\cap M_{1})$ and $\rho_{i}:M_{i}/I^{2}M_{i}\to M_{i}/IM_{i}$ for $i\in\{2,3\}$. This gives the diagram 
    $$% https://q.uiver.app/#q=WzAsMTAsWzEsMCwiTV97MX0vKEleezJ9TV97Mn1cXGNhcCBNX3sxfSkiXSxbMSwxLCJNX3sxfS8oSU1fezJ9XFxjYXAgTV97MX0pIl0sWzMsMCwiTV97Mn0vSV57Mn1NX3syfSJdLFszLDEsIk1fezJ9L0lNX3syfSJdLFs1LDAsIk1fezN9L0leezJ9TV97M30iXSxbNSwxLCJNX3szfS9JTV97M30iXSxbMCwwLCIwIl0sWzAsMSwiMCJdLFs2LDAsIjAiXSxbNiwxLCIwIl0sWzAsMSwiXFxwaSIsMl0sWzYsMF0sWzcsMV0sWzEsM10sWzMsNV0sWzUsOV0sWzQsOF0sWzIsNF0sWzAsMl0sWzQsNSwiXFxyaG9fezN9IiwyXSxbMiwzLCJcXHJob197Mn0iLDJdXQ==
    \begin{tikzcd}
        0 & {M_{1}/(I^{2}M_{2}\cap M_{1})} && {M_{2}/I^{2}M_{2}} && {M_{3}/I^{2}M_{3}} & 0 \\
        0 & {M_{1}/(IM_{2}\cap M_{1})} && {M_{2}/IM_{2}} && {M_{3}/IM_{3}} & 0
        \arrow[from=1-1, to=1-2]
        \arrow[from=1-2, to=1-4]
        \arrow["\pi"', from=1-2, to=2-2]
        \arrow[from=1-4, to=1-6]
        \arrow["{\rho_{2}}"', from=1-4, to=2-4]
        \arrow[from=1-6, to=1-7]
        \arrow["{\rho_{3}}"', from=1-6, to=2-6]
        \arrow[from=2-1, to=2-2]
        \arrow[from=2-2, to=2-4]
        \arrow[from=2-4, to=2-6]
        \arrow[from=2-6, to=2-7]
    \end{tikzcd}$$
    where $\pi,\rho_{2},\rho_{3}$ are surjective. Then $m_{1}=\rho_{3}(m_{2})$ by definition so $\rho_{2}(m_{2}'),m_{1}'\in M_{2}/IM_{2}$ have the same image in $M_{3}/IM_{3}$. That is, 
    $$% https://q.uiver.app/#q=WzAsNCxbMCwwLCJtX3syfSciXSxbMCwxLCJtX3sxfSc9XFxyaG9fezJ9KG1fezJ9JykiXSxbMiwwLCJtX3syfSJdLFsyLDEsIm1fezF9PVxccmhvX3szfShtX3syfSkiXSxbMCwxLCIiLDIseyJzdHlsZSI6eyJ0YWlsIjp7Im5hbWUiOiJtYXBzIHRvIn19fV0sWzEsMywiIiwyLHsic3R5bGUiOnsidGFpbCI6eyJuYW1lIjoibWFwcyB0byJ9fX1dLFsyLDMsIiIsMCx7InN0eWxlIjp7InRhaWwiOnsibmFtZSI6Im1hcHMgdG8ifX19XSxbMCwyLCIiLDAseyJzdHlsZSI6eyJ0YWlsIjp7Im5hbWUiOiJtYXBzIHRvIn19fV1d
    \begin{tikzcd}
        {m_{2}'} && {m_{2}} \\
        {m_{1}'=\rho_{2}(m_{2}')} && {m_{1}=\rho_{3}(m_{2})}
        \arrow[maps to, from=1-1, to=1-3]
        \arrow[maps to, from=1-1, to=2-1]
        \arrow[maps to, from=1-3, to=2-3]
        \arrow[maps to, from=2-1, to=2-3]
    \end{tikzcd}$$
    for the rightmost square. Thus $\rho_{2}(m_{2}')-m_{1}'\in M_{1}/IM_{1}$. Since $\pi$ is surjective, we can choose $m_{2}''\in M_{1}/(I^{2}M_{2}\cap M_{1})$ such that $\pi(m_{2}'')=\rho_{2}(m_{2}')-m_{1}'$. That is, 
    $$% https://q.uiver.app/#q=WzAsNCxbMCwwLCJtX3syfScnIl0sWzIsMSwiXFxyaG9fezJ9KG1fezJ9JyktbV97MX0nPTAiXSxbMiwwLCJtX3syfScnIl0sWzAsMSwiXFxyaG9fezJ9KG1fezJ9JyktbV97MX0nIl0sWzAsM10sWzMsMV0sWzIsMV0sWzAsMl1d
    \begin{tikzcd}
        {m_{2}''} && {m_{2}''} \\
        {\rho_{2}(m_{2}')-m_{1}'} && {\rho_{2}(m_{2}')-m_{1}'=0}
        \arrow[from=1-1, to=1-3]
        \arrow[from=1-1, to=2-1]
        \arrow[from=1-3, to=2-3]
        \arrow[from=2-1, to=2-3]
    \end{tikzcd}$$
    for the leftmost square. Substituting $m_{2}'$ by $m_{2}'-m_{2}''$ we get surjectivity, and by induction we can lift each element of the sequence, giving surjectivity $\widehat{M_{2}}\to\widehat{M_{3}}$. 
\end{proof}
\begin{proof}[Proof of (ii)]
    By the finite generation hypothesis, we have an exact sequence $A^{\oplus m}\to A^{\oplus n}\to M\to0$. Tensoring with $A$ yields a diagram 
    $$% https://q.uiver.app/#q=WzAsOCxbMCwwLCJcXHdpZGVoYXR7QX1ee1xcb3BsdXMgbX0iXSxbMCwxLCJcXHdpZGVoYXR7QX1ee1xcb3BsdXMgbX0iXSxbMiwwLCJcXHdpZGVoYXR7QX1ee1xcb3BsdXMgbn0iXSxbMiwxLCJcXHdpZGVoYXR7QX1ee1xcb3BsdXMgbn0iXSxbNCwwLCJNXFxvdGltZXNfe0F9XFx3aWRlaGF0e0F9Il0sWzQsMSwiXFx3aWRlaGF0e019Il0sWzUsMCwiMCJdLFs1LDEsIjAiXSxbMCwyXSxbMiw0XSxbNCw2XSxbNSw3XSxbMyw1XSxbMSwzXSxbMCwxLCJcXHdyIiwyXSxbMiwzLCJcXHdyIiwyXSxbNCw1XSxbNiw3LCJcXHdyIiwyXV0=
    \begin{tikzcd}
        {\widehat{A}^{\oplus m}} && {\widehat{A}^{\oplus n}} && {M\otimes_{A}\widehat{A}} & 0 \\
        {\widehat{A}^{\oplus m}} && {\widehat{A}^{\oplus n}} && {\widehat{M}} & 0
        \arrow[from=1-1, to=1-3]
        \arrow["\wr"', from=1-1, to=2-1]
        \arrow[from=1-3, to=1-5]
        \arrow["\wr"', from=1-3, to=2-3]
        \arrow[from=1-5, to=1-6]
        \arrow[from=1-5, to=2-5]
        \arrow["\wr"', from=1-6, to=2-6]
        \arrow[from=2-1, to=2-3]
        \arrow[from=2-3, to=2-5]
        \arrow[from=2-5, to=2-6]
    \end{tikzcd}$$
    with exact rows, giving the isomorphism $M\otimes_{A}\widehat{A}\to\widehat{M}$ by the four-lemma. 
\end{proof}
\begin{proof}[Proof of (iii)]
    This is immediate from (i) and (ii) which show that completion given by $-\otimes_{A}\widehat{A}$ is exact so $\widehat{A}$ is flat as an $A$-module.  
\end{proof}
\begin{proof}[Proof of (iv)]
    Let $a_{1},\dots,a_{r}$ be generators of $I$. Consider the map $A[[x_{1},\dots,x_{r}]]\to\widehat{A}$ by $x_{i}\mapsto a_{i}$ induced by the map $A[x_{1},\dots,x_{r}]\to A$ by $x_{i}\mapsto a_{i}$ -- a surjective map from a Noetherian ring. This shows that $A[[x_{1},\dots,x_{r}]]\to\widehat{A}$ is a surjection by (i). And since $A[[x_{1},\dots,x_{d}]]$ is Noetherian, we get the claim. 
\end{proof}
\begin{proof}[Proof of (v)]
    Let $\widehat{\mfrak}=\lim_{d\in\NN}\mfrak/\mfrak^{d}$. By (i) we have $\widehat{A}/\widehat{\mfrak}\cong\widehat{A/\mfrak}\cong A/\mfrak$ since $\mfrak$ is trivial in $A/\mfrak$. Hence $\widehat{\mfrak}\subseteq\widehat{A}$ is maximal. To see that $\widehat{A}$ is local, it suffices to show that every element of $\widehat{A}\setminus\widehat{m}$ is a unit. Let $x=(a_{1},a_{2},\dots)$ be such an element. Necessarily $a_{1}\in (A/\mfrak)^{\times}$ is nonzero, and $a_{1}$ is the projection in $A/\mfrak$ of each $a_{i}$ so each $a_{i}$ is a unit, and $x^{-1}(a_{1}^{-1},a_{2}^{-1},\dots)$, as desired. 

    For the claim on the gradeds, it suffices to show that $A/\mfrak^{d}\cong\widehat{A}\cong\widehat{\mfrak}^{d}$ in which case the graded rings will agree in each degree. Note that the proof of the first part of the statement already gives $\gr_{\mfrak}^{0}(A)\cong\gr_{\widehat{\mfrak}}^{0}(\widehat{A})$. By (i) we have that $\widehat{A}/\widehat{\mfrak}^{d}\cong\widehat{A/\mfrak^{d}}$. Using that $\mfrak^{n}/\mfrak^{d}=0$ for all $n\geq d$, we use the diagram 
    $$% https://q.uiver.app/#q=WzAsMTAsWzEsMCwiXFxtZnJha157ZH0vXFxtZnJha157ZCsxfSJdLFszLDAsIkEvXFxtZnJha157ZCsxfSJdLFs1LDAsIkEvXFxtZnJha157ZH0iXSxbMCwwLCIwIl0sWzAsMSwiMCJdLFsxLDEsIlxcd2lkZWhhdHtcXG1mcmFrfV57ZH0vXFx3aWRlaGF0e1xcbWZyYWt9XntkKzF9Il0sWzMsMSwiXFx3aWRlaGF0e0F9L1xcd2lkZWhhdHtcXG1mcmFrfV57ZCsxfSJdLFs1LDEsIlxcd2lkZWhhdHtBfS9cXHdpZGVoYXR7XFxtZnJha31ee2R9Il0sWzYsMCwiMCJdLFs2LDEsIjAiXSxbMywwXSxbMCwxXSxbMSwyXSxbMiw4XSxbNyw5XSxbNiw3XSxbNCw1XSxbNSw2XSxbMCw1XSxbMSw2LCJcXHdyIl0sWzIsNywiXFx3ciJdXQ==
    \begin{tikzcd}
        0 & {\mfrak^{d}/\mfrak^{d+1}} && {A/\mfrak^{d+1}} && {A/\mfrak^{d}} & 0 \\
        0 & {\widehat{\mfrak}^{d}/\widehat{\mfrak}^{d+1}} && {\widehat{A}/\widehat{\mfrak}^{d+1}} && {\widehat{A}/\widehat{\mfrak}^{d}} & 0
        \arrow[from=1-1, to=1-2]
        \arrow[from=1-2, to=1-4]
        \arrow[from=1-2, to=2-2]
        \arrow[from=1-4, to=1-6]
        \arrow["\wr", from=1-4, to=2-4]
        \arrow[from=1-6, to=1-7]
        \arrow["\wr", from=1-6, to=2-6]
        \arrow[from=2-1, to=2-2]
        \arrow[from=2-2, to=2-4]
        \arrow[from=2-4, to=2-6]
        \arrow[from=2-6, to=2-7]
    \end{tikzcd}$$
    with exact rows to deduce that $\mfrak^{d}/\mfrak^{d+1}\cong\widehat{\mfrak}^{d}/\widehat{\mfrak}^{d+1}$ so each $\gr_{\mfrak}^{i}(A)\cong\gr_{\widehat{\mfrak}}^{i}(\widehat{A})$ yielding the claim. 
\end{proof}
\begin{proof}[Proof of (vi)]
    We show (a)$\Leftrightarrow$(c)$\Leftrightarrow$(b). 

    (a)$\Rightarrow$(c) Recall from \Cref{def: regular local ring} that for $A$ regular we have $\mfrak=(a_{1},\dots,a_{d})$. Denoting $k=A/\mfrak$ the residue field of $A$, we have a map $k[x_{1},\dots,x_{d}]\to\gr^{\bullet}_{\mfrak}(A)$ by $x_{i}\mapsto a_{i}$. This is an isomorphism in degree 0 and 1, and therefore an isomorphism since $\gr^{i}_{\mfrak}(A)\cong\Sym^{i}(\mfrak/\mfrak^{2})\cong\mfrak^{i}/\mfrak^{i+1}$ by regularity. 

    (c)$\Rightarrow$(a) If $\gr_{\mfrak}^{\bullet}(A)\cong k[x_{1},\dots,x_{d}]$ then $\mfrak/\mfrak^{2}$ is generated by $d$ elements, and $A$ is regular. 

    (b)$\Rightarrow$(c) This is the argument of (a)$\Rightarrow$(c) verbatim. By regularity, we take $\widehat{\mfrak}=(\widehat{a}_{1},\dots,\widehat{a}_{d})$ and we have $k=A/\mfrak\cong \widehat{A}/\widehat{\mfrak}$ by (i). The map $k[x_{1},\dots,x_{d}]\to\gr^{\bullet}_{\widehat{\mfrak}}(\widehat{A})$ is an isomorphism in degrees 0 and 1, and an isomorphism globally by $\gr^{i}_{\widehat{A}}(\widehat{A})\cong\Sym^{i}(\widehat{\mfrak}/\widehat{\mfrak}^{2})\cong\widehat{\mfrak}^{i}/\widehat{\mfrak}^{i+1}\cong\mfrak^{i}/\mfrak^{i+1}$ due to regularity. 

    (c)$\Rightarrow$(a) This is the argument of (c)$\Rightarrow$(a) verbatim. $\gr^{\bullet}_{\widehat{\mfrak}}(\widehat{A})\cong k[x_{1},\dots,x_{d}]$ is regular, so $\widehat{\mfrak}$ is generated by $d$ elements, hence regular. 
\end{proof}
\begin{proof}[Proof of (vii)]
    If $\mfrak=(a_{1},\dots,a_{d})$, we have the map $k[x_{1},\dots,x_{d}]\to\widehat{A}$ by $x_{i}\mapsto a_{i}$. We have an isomorphism $A/\mfrak\cong k$. So by (vi), the isomorphism of graded constructions $\gr_{\mfrak}^{\bullet}(A)\cong\gr_{\widehat{\mfrak}}^{\bullet}(\widehat{A})=k[x_{1},\dots,x_{d}]$ induce an isomorphism between the completions and $k[[x_{1},\dots,x_{d}]]$. 
\end{proof}
We conclude our discussion of completion by stating Cohen's structure theorem. 
\begin{theorem}[Cohen -- Structure]\label{thm: Cohen structure}
    Let $A$ be a complete Noetherian local ring with maximal ideal $\mfrak$ containing some field. Then there exixts a field $k$ in $A$ such that $k=A/\mfrak$ is the residue field of $A$, and $A\cong k[[x_{1},\dots,x_{d}]]/I$ fo some $d\geq0$ and ideal $I$. Furthermore, if $A$ is regular, then $A\cong k[[x_{1},\dots,x_{d}]]$. 
\end{theorem}
\begin{proof}
    See \cite[\href{https://stacks.math.columbia.edu/tag/032A}{Tag 032A}]{stacks-project}.
\end{proof}
We consider some examples. 
\begin{example}
    Let $C=V(y^{2}-x^{3}-x^{2})\subseteq\A^{2}_{k},C'=V(y^{2}-x^{2})\subseteq\A^{2}_{k}$. Both of these curves are nodal at the origin, but their local rings $(k[x,y]/(y^{2}-x^{3}-x^{2}))_{(x,y)},(k[x,y]/(y^{2}-x^{2}))_{(x,y)}$ are not isomorphic. In a certain sense, they are not sufficiently ``local'' to capture the geometric behavior of these two being nodes. However, their completions at $(x,y)$ are isomorphic. 
\end{example}