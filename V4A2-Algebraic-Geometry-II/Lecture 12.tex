\section{Lecture 12 -- 22nd May 2025}\label{sec: lecture 12}
We begin a discussion of blowups. Recall that Cartier divisors are sections of $\Gamma(X,\Kcal_{X}^{\times}/\Ocal_{X}^{\times})$. Roughly speaking, this provides the data of collections of rational functions on an affine open cover with regular quotients. In this way, Cartier divisors play a fundamental role in the study of schemes. However, divisors may fail to be Cartier for two reasons: singularties and being of the wrong codimension. 

Blowups give a ``universal'' construction to modify a scheme with its closed subscheme to an effective Cartier divisor. 
\begin{definition}[Blowup]\label{def: blowup}
    Let $X$ be locally Noetherian and $Z\subseteq X$ a closed subscheme. The blowup of $X$ along $Z$ is a Cartesian diagram 
    $$% https://q.uiver.app/#q=WzAsNCxbMCwwLCJFX3tafVgiXSxbMiwwLCJcXEJsX3tafVgiXSxbMiwxLCJYIl0sWzAsMSwiWiJdLFszLDJdLFsxLDJdLFswLDNdLFswLDFdXQ==
    \begin{tikzcd}
        {E_{Z}X} && {\Bl_{Z}X} \\
        Z && X
        \arrow[from=1-1, to=1-3]
        \arrow[from=1-1, to=2-1]
        \arrow[from=1-3, to=2-3]
        \arrow[from=2-1, to=2-3]
    \end{tikzcd}$$
    where the exceptional divisor $E_{Z}X$ is an efective Cartier divisor in $\Bl_{Z}X$ and final amongst cartier divisor-scheme pairs $(D,W)$ fitting into Cartesian diagrams 
    $$% https://q.uiver.app/#q=WzAsNCxbMCwwLCJEIl0sWzIsMCwiVyJdLFsyLDEsIlguIl0sWzAsMSwiWiJdLFszLDJdLFsxLDJdLFswLDNdLFswLDFdXQ==
    \begin{tikzcd}
        D && W \\
        Z && {X.}
        \arrow[from=1-1, to=1-3]
        \arrow[from=1-1, to=2-1]
        \arrow[from=1-3, to=2-3]
        \arrow[from=2-1, to=2-3]
    \end{tikzcd}$$
\end{definition}
\begin{remark}
    That is, for any Cartesian square 
    $$% https://q.uiver.app/#q=WzAsNCxbMCwwLCJEIl0sWzIsMCwiVyJdLFsyLDEsIlguIl0sWzAsMSwiWiJdLFszLDJdLFsxLDJdLFswLDNdLFswLDFdXQ==
    \begin{tikzcd}
        D && W \\
        Z && {X}
        \arrow[from=1-1, to=1-3]
        \arrow[from=1-1, to=2-1]
        \arrow[from=1-3, to=2-3]
        \arrow[from=2-1, to=2-3]
    \end{tikzcd}$$
    where $D$ is an effective Cartier divisor in $W$, there exists a factorization of this diagram 
    $$% https://q.uiver.app/#q=WzAsNixbMCwxLCJFX3tafVgiXSxbMiwxLCJcXEJsX3tafVgiXSxbMiwyLCJYIl0sWzAsMiwiWiJdLFswLDAsIkQiXSxbMiwwLCJXIl0sWzMsMl0sWzEsMl0sWzAsM10sWzAsMV0sWzQsNV0sWzUsMV0sWzQsMF1d
    \begin{tikzcd}
        D && W \\
        {E_{Z}X} && {\Bl_{Z}X} \\
        Z && X
        \arrow[from=1-1, to=1-3]
        \arrow[from=1-1, to=2-1]
        \arrow[from=1-3, to=2-3]
        \arrow[from=2-1, to=2-3]
        \arrow[from=2-1, to=3-1]
        \arrow[from=2-3, to=3-3]
        \arrow[from=3-1, to=3-3]
    \end{tikzcd}$$
    where both squares are Cartesian. 
\end{remark}
\begin{remark}
    Having defined blowups in \Cref{def: blowup} by its universal property, it is unique up to unique isomorphism if it exists. 
\end{remark}
\begin{remark}
    If $Z\subseteq X$ is Cartier, then the blowup is just $X$, since the pair $Z\to X$ trivially satisfies the desired universal property. 
\end{remark}
We can show that these exist, first affine-locally, then globally by gluing. 
\begin{lemma}\label{lem: blowup exists affine locally}
    Let $A$ be a Noetherian ring and $I\subseteq Z$ an ideal. The blowup of $\spec(A)$ along $V(I)$ exists. 
\end{lemma}
\begin{proof}
    Since $A$ is Noetherian, $I$ is finitely generated, say by $a_{0},\dots,a_{n}$. Consider $\proj(\bigoplus_{d\geq0}I^{d})$. We show this satisfies the universal property. 

    Note $\beta^{-1}(I)=I\cdot\bigoplus_{d\geq0}I^{d}=\Ocal_{\proj(\bigoplus_{d\geq0}I^{d})}(1)$ which is a Cartier divisor. We can define a map of graded rings $\varphi:A[x_{0},\dots,x_{n}]\to\bigoplus_{d\geq0}I^{d}$ by $x_{i}\mapsto a_{i}$ which is by inspection a surjective morphism of graded rings. This induces contravariantly on $\proj$ 
    $$% https://q.uiver.app/#q=WzAsMyxbMCwwLCJcXHByb2ooXFxiaWdvcGx1c197ZFxcZ2VxMH1JXntkfSkiXSxbMSwxLCJcXHNwZWMoQSkiXSxbMiwwLCJcXFBQXntufV97QX0iXSxbMCwyXSxbMiwxXSxbMCwxXV0=
    \begin{tikzcd}
        {\proj(\bigoplus_{d\geq0}I^{d})} && {\PP^{n}_{A}} \\
        & {\spec(A)}
        \arrow[from=1-1, to=1-3]
        \arrow[from=1-1, to=2-2]
        \arrow[from=1-3, to=2-2]
    \end{tikzcd}$$
    where $\PP^{n}_{A}\to\spec(A)$ and $\proj(\bigoplus_{d\geq0}I^{d})\to\PP^{n}_{A}$ are both closed, so $\proj(\bigoplus_{d\geq0}I^{d})\to\spec(A)$ is closed and the kernel of $\varphi$ is the ideal generated by homogeneous polynomials in $n+1$ variables which vanish at $(a_{0},\dots,a_{n})$. 
    
    Let 
    $$% https://q.uiver.app/#q=WzAsNCxbMCwwLCJEIl0sWzIsMCwiVyJdLFswLDEsIlYoSSkiXSxbMiwxLCJcXHNwZWMoQSkiXSxbMiwzXSxbMSwzLCJmIl0sWzAsMV0sWzAsMl1d
    \begin{tikzcd}
        D && W \\
        {V(I)} && {\spec(A)}
        \arrow[from=1-1, to=1-3]
        \arrow[from=1-1, to=2-1]
        \arrow["f", from=1-3, to=2-3]
        \arrow[from=2-1, to=2-3]
    \end{tikzcd}$$
    be a Cartesian square where $D$ is an effective Cartier divisor in $W$ -- that is, where $f^{-1}I\cdot\Ocal_{W}=\Ical_{D}$. Since $I$ is finitely generated by the $a_{i}$'s, their images $s_{0},\dots,s_{n}$ in $\Ocal_{W}$ generate $\Ical_{D}$. This induces a unique morphism $g:W\to\PP^{n}_{A}$ over $\spec(A)$ such that $g^{*}\Ocal_{\PP^{n}_{A}}(1)\cong\Ical_{D}$ with $s_{i}=g^{-1}(x_{i})$. Moreover, this morphism factors over the closed subscheme $\proj(\bigoplus_{d\geq0}I^{d})\subseteq\PP^{n}_{A}$ -- any element of the kernel $\ker(\varphi)$ of the morphism of graded rings is a homogeneous polynomial of degree $m$ that vanishes on $(a_{0},\dots,a_{n})$ and hence on $(s_{0},\dots,s_{n})$ in $\Gamma(W,\Ical_{D}^{m})$. This shows that $\proj(\bigoplus_{d\geq0}I^{d})$ satisfies the desired universal property. 
\end{proof}
\begin{remark}
    Let $f:X\to Y$ be any morphism and $Z\subseteq Y$ closed with sheaf of ideals $\Ical_{Z}$. $f^{-1}\Ical_{Z}$ does not necessarily agree with $f^{*}\Ical_{Z}=\Ocal_{X}\otimes_{f^{-1}\Ocal_{Y}}f^{-1}\Ical_{Z}$, and $f^{*}\Ical_{Z}$ need not even be a subsheaf of $\Ocal_{X}$. There exists a morphism $f^{*}\Ical_{Z}\to\Ocal_{X}$ whose image is $f^{-1}\Ical_{Z}$ the ideal sheaf which need not be an isomorphism. 
\end{remark}
We now treat the general case by gluing.  
\begin{theorem}\label{thm: existence of blowups}
    Let $X$ be locally Noetherian and $Z\subseteq X$ a closed subscheme. The blowup of $X$ along $Z$ exists. 
\end{theorem}
\begin{proof}
    Note that for $U\subseteq X$ open, $\Bl_{(Z\cap U)}U\cong \beta^{-1}(U)$ uniquely by the universal property. Covering $X$ with affine opens, and the intersections of any two such affine opens with distinguished opens, existence and uniqueness of the blowup affine-locally \Cref{lem: blowup exists affine locally} shows that the construction glues to the blowup of $X$. 
\end{proof}
\begin{remark}\label{rmk: blowup of open subschemes}
    If $U\subseteq X$ is open, then $\Bl_{U\cap Z}U=\beta^{-1}(U)$, and if $U=X\setminus Z$ then $\beta^{-1}(U)\to U$ is an isomorphism. 
\end{remark}
We want to consider how subschemes in $X$ behave in the blowup. 
\begin{definition}[Strict Transform]\label{def: strict transform}
    Let $X$ be locally Noetherian, $Z\subseteq X$ a closed subscheme, and $\beta:\Bl_{Z}X\to X$ the blowup of $X$ along $Z$. If $Y\subseteq X$ is a closed subscheme of $X$ not contained in $Z$ the total transform of $Y$ is the scheme-theoretic preimage $Y\times_{X}\Bl_{Z}X$ of $Y$ in $\Bl_{Z}X$. 
\end{definition}
\begin{definition}[Total Transform]\label{def: total transform}
    Let $X$ be locally Noetherian, $Z\subseteq X$ a closed subscheme, and $\beta:\Bl_{Z}X\to X$ the blowup of $X$ along $Z$. If $Y\subseteq X$ is a closed subscheme of $X$ not contained in $Z$ the total transform of $Y$ is $\widetilde{Y}=\overline{\beta^{-1}(Y\setminus(Y\cap Z))}\subseteq\Bl_{Z}X$. 
\end{definition}
Let us make some computations of the line bundles associated to the exceptional divisor of blowups. 
\begin{proposition}\label{prop: exceptional divisor is O minus one}
    Let $X$ be locally Noetherian, $Z\subseteq X$ a closed subscheme, and $\beta:\Bl_{Z}X\to X$ the blowup of $X$ along $Z$. Denote the structure sheaf of $\Bl_{Z}X$ by $\Ocal_{\beta}$. Then $\Ocal_{\beta}(E_{Z}X)=\Ocal_{\beta}(-1)$. 
\end{proposition}
\begin{proof}
    It suffices to observe that the ideal sheaf of $E_{Z}X$ is $\Ocal_{\beta}(1)$, so twisting by this divisor gives the dual of the ideal sheaf $\Ocal_{\beta}(E_{Z}X)\cong\Ocal_{\beta}(-1)$
\end{proof}
\begin{remark}
    The use of $\Ocal_{\beta}$ for $\Ocal_{\Bl_{Z}X}$ is justified as $\Bl_{Z}X$ is a projective bundle as the $\underline{\proj}$ of the sheaf of graded algebras locally given by the Rees algebra as shown in the construction of the blowup \Cref{lem: blowup exists affine locally}. 
\end{remark}
Moreover, smoothess is preserved under blowups. 
\begin{proposition}
    Let $X$ be locally Noetherian, $Z\subseteq X$ a closed subscheme, and $\beta:\Bl_{Z}X\to X$ the blowup of $X$ along $Z$. If $X$ and $Z$ are smooth, then $\Bl_{Z}X$ is smooth, and $\Ncal_{E_{Z}X/\Bl_{Z}X}=\Ocal_{E_{Z}X}(-1)$.\todo{Prove this.}
\end{proposition}
\begin{example}\label{ex: blowup of A2 at origin}
    Consider $\Bl_{\{(0,0)\}}\A^{2}_{k}$. Note that $\A^{2}_{k}=\spec(k[x,y])$ and the defining ideal of $\{(0,0)\}$ is $I=(x,y)$. We know that $\Bl_{\{(0,0)\}}\A^{2}_{k}\subseteq\PP^{1}_{\A^{2}_{k}}=\PP^{1}_{A}$ where $A=k[x,y]$. We can define a ring map $\varphi:A[u,v]\mapsto\bigoplus_{d\geq0}I^{d}$ by $u\mapsto x,v\mapsto y$. The kernel is generated by $uy-vx$ which defines the blowup as a closed subscheme of $\PP^{1}_{A}$. We can easily see that the fiber over any point away from the origin is a single point, and the fiber over the origin is an entire $\PP^{1}_{k}$. Indeed for any $L$ a line through the origin defined by $\{y=tx\}$ the strict transform is the union of the line itself with the $\PP^{1}_{k}$ over the origin, and the line intersects the $\PP^{1}_{k}$ over the origin at its slope $[1:t]$. 
\end{example}
We prove the property of \Cref{rmk: blowup of open subschemes} for closed subschemes. 
\begin{lemma}\label{lem: blowup of closed subschemes}
    Let $X$ be locally Noetherian, $Z\subseteq X$ a closed subscheme, and $\beta:\Bl_{Z}X\to X$ the blowup of $X$ along $Z$. Let $Y\subseteq X$ be a closed subscheme not contained in $Z$. Then $\Bl_{(Z\cap Y)}Y\cong\widetilde{Y}$ the strict transform of $Y$ in the blowup of $X$. 
\end{lemma}
\begin{proof}
    We show this satisfies the universal property. Consider the following diagram. 
    $$% https://q.uiver.app/#q=WzAsOCxbMCwxLCJEIl0sWzEsMCwiVyJdLFsyLDEsIllcXGNhcCBaIl0sWzMsMCwiWSJdLFs0LDEsIloiXSxbNSwwLCJYIl0sWzMsMiwiXFxCbF97Wn1YIl0sWzIsMywiRV97Wn1YIl0sWzAsMl0sWzIsNF0sWzMsNV0sWzQsNV0sWzIsM10sWzAsMV0sWzEsM10sWzAsNywiIiwxLHsiY3VydmUiOjJ9XSxbMSw2LCJnIiwxLHsibGFiZWxfcG9zaXRpb24iOjcwLCJjdXJ2ZSI6M31dLFs2LDUsIiIsMSx7ImN1cnZlIjoyfV0sWzcsNCwiIiwxLHsiY3VydmUiOjN9XSxbNyw2XV0=
    \begin{tikzcd}
        & W && Y && X \\
        D && {Y\cap Z} && Z \\
        &&& {\Bl_{Z}X} \\
        && {E_{Z}X}
        \arrow[from=1-2, to=1-4]
        \arrow["g"{description, pos=0.7}, curve={height=18pt}, from=1-2, to=3-4]
        \arrow[from=1-4, to=1-6]
        \arrow[from=2-1, to=1-2]
        \arrow[from=2-1, to=2-3]
        \arrow[curve={height=12pt}, from=2-1, to=4-3]
        \arrow[from=2-3, to=1-4]
        \arrow[from=2-3, to=2-5]
        \arrow[from=2-5, to=1-6]
        \arrow[curve={height=12pt}, from=3-4, to=1-6]
        \arrow[curve={height=18pt}, from=4-3, to=2-5]
        \arrow[from=4-3, to=3-4]
    \end{tikzcd}$$
    We seek to show that the image of $g$ is contained in $\widetilde{Y}$. But this is clear from the construction of $\widetilde{Y}=\overline{\beta^{-1}(Y\setminus(Y\cap Z))}$ and uniqueness from the universal property of the blowup $\Bl_{Z}X$. 
\end{proof}
We show some properties of the blowup morphism $\beta:\Bl_{Z}X\to X$. 
\begin{proposition}\label{prop: properties of blowup map}
    Let $X$ be locally Noetherian, $Z\subseteq X$ a closed subscheme, and $\beta:\Bl_{Z}X\to X$ the blowup of $X$ along $Z$. Then $\beta$ is proper, birational, and surjective. Moreover, if $X$ is projective, so too is $\Bl_{Z}X$. 
\end{proposition}
\begin{proof}
    Properness is local on target and $\beta$ is locally given by the composition of a closed immersion and projection $\PP^{n}_{A}\to\spec(A)$, hence proper. Birationality is clear as $\Bl_{Z}X\setminus E_{Z}X\to X\setminus Z$ is an isomorphism between dense open sets of the sorce and target. If $X$ is projective, then the structrure map of $\Bl_{Z}X$ is the composition of two projective morphisms. hence projective. 
\end{proof}
\begin{remark}
    The blowup is rarely flat or unramified, hence almost never \'{e}tale. 
\end{remark}
We consider some examples of blowups of curves, illustrating that they can be used to resolve singularties. 
\begin{example}\label{ex: blowup of node}
    Consider the nodal cubic $V(y^{2}-x^{3}-x^{2})\subseteq\A^{2}_{k}$. Since this is singular at the origin, we can build on our computation of the blowup of $\A^{2}_{k}$ at the origin \Cref{ex: blowup of A2 at origin} to see that $\Bl_{\{(0,0)\}}C$ is given by $\{y^{2}-x^{3}-x^{2},uy-vx\}$. On the $u\neq0$ chart, the equations $y=xv,x^{2}v^{2}=x^{3}+x^{2}$ shows that $\Bl_{\{(0,0)\}}C$ meets the exceptional divisor at two points, since the latter equation is quadratic in the $\PP^{1}_{k}$-variable $v$. 
\end{example}
\begin{example}
    The cuspidal curve $V(y^{2}-x^{3})\subseteq\A^{2}_{k}$ is similarly singular at the origin. Repeating the computation of \Cref{ex: blowup of node}, we get that $\Bl_{\{(0,0)\}}C$ is given by the equations $\{y^{2}-x^{3},uy-vx\}$ which on the $u\neq0$ chart gives $\{y=xv,x^{2}v^{2}=x^{3}\}$ showing that the blowup meets the exceptional divisor at one point with multiplicity 2. 
\end{example}