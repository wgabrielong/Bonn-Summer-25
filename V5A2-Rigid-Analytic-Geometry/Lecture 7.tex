\section{Lecture 7 -- 5th June 2025}\label{sec: lecture 7}
We continue with the proofs of results relating to vertical generifications in adic spaces. 
\begin{lemma}\label{lem: maximal convex subgroup}
    Let $A$ be a Tate ring with topologically nilpotent unit $s$ and $(\nu/\sim)\in\Spa(A,A^{\circ})$. 
    \begin{enumerate}[label=(\roman*)]
        \item $\Lambda_{\max}=\{\gamma\in\Gamma:\nu(s)\leq\gamma^{n}\leq\nu(s)^{-1},\forall n\in\NN\}$ is the largest proper convex subgroup of $\Gamma$. 
        \item The set of convex subgroups of $\Gamma$ is linearly ordered.
        \item The most vertical generification of $(\nu/\sim)$ is $((\nu/\Lambda_{\max})/\sim)$. 
        \item All generifications of $\nu$ are vertical. 
    \end{enumerate}
\end{lemma}
\begin{proof}[Proof of (i)]
    By continuity $\Gamma=\bigcup_{m\in\NN}[\nu(s)^{m},\nu(s)^{-m}]_{\Gamma}$ so $\Gamma=\bigcup_{m\in\NN}[\gamma^{m},\gamma^{-m}]_{\Gamma}$ for any $\gamma\in\Gamma\setminus\Lambda_{\max}$ with
$|\gamma|<1$, hence $\Lambda_{\max}$ is indeed maximal.
\end{proof}
\begin{proof}[Proof of (ii)]
Let $\widetilde{\nu}$ be a generification of $\nu$. By \Cref{lem: disjoint neighborhoods on Spa} their supports coincide $\supp(\nu)=\supp(\tilde{\nu})=\pfrak$, and $\tilde{\nu}=\nu/\Lambda$ where $\Lambda=\bigcup_{a,b\in A\setminus\pfrak, \tilde{\nu}\in R_{\Spa(A,A^{\circ})}(a|b, s^m)}\left[\frac{\nu(a)}{\nu(b)},\frac{\nu(b)}{\nu(a)}\right]_{\Gamma}$ so $\widetilde{\nu}\in R_{\Spa(A,A^{\circ})}(a|b,s^{m})$. 
\end{proof}
\begin{proof}[Proof of (iii)]
    This is immediate from maximality of $\Lambda_{\max}$. 
\end{proof}
\begin{proof}[Proof of (iv)]
    Let $\mu/\sim$ be a generification of $\nu/\sim$. By \Cref{lem: disjoint neighborhoods on Spa}, the two valuations have identical supports. We claim that if $\nu(a)\leq\nu(b)$ then $\mu(a)\leq\mu(b)$. If $\nu(a)=0$ then the statement is trivial as it follows from equality of supports. Otherwise if $\nu(a)\neq 0$, then also $\nu(b)\neq0$ and by continuity there is some natural number $n$ such that $\nu(s^{n})\leq\nu(b)$. Then $(\nu/\sim)\in R_{\Spa(A,A^{\circ})}(b|a,s^{n})$. Since $\mu/\sim$ is a generification, it lies in $R_{\Spa(A,A^{\circ})}(b|a,s^{n})$ as well. So $\mu(a)\leq\mu(b)$ as desired. It then follows that $\mu\simeq\nu/\Lambda$ where $\Lambda\subseteq\Gamma$ is the convex subgroup generated by $\{\nu(a):a\in A, \nu(a)=1\}$. 
\end{proof}
\begin{remark}
    Note that \Cref{lem: disjoint neighborhoods on Spa} (which implies that generifications have identical supports) only holds in the Tate case. 
\end{remark}
\begin{proposition}
    Let $(A,A^{+})$ be a Huber pair with $A$ Tate and $x,y\in\Spa(A,A^{+})$. Then $x,y$ have disjoint open neighborhoods if and only if their most generic vertical generifications differ.
\end{proposition}
\begin{proof}
    Suppose that $x,y$ have disjoint open neighborhoods. Then they obviously cannot have a common vertical generification. 

    Conversely suppose $\xi,\nu$ are the most generic vertical generifications of $x,y$, repsectively. By \Cref{lem: cts map to spec A} their supports agree which we denote $\pfrak\subseteq A$. Without loss of generality, there are $a,b\in A\setminus\pfrak$ such that $|a|_{\xi}\leq|b|_{\xi}$ but $|b|_{\nu}\leq |a|_{\nu}$. Then $\frac{|a|_{\nu}}{|b|_{\nu}}$ is not an element of the largest proper convex subgroup of $\Gamma_{y}$. Hence by \Cref{lem: maximal convex subgroup} there is $m\in\NN$ such that $|b|_{y}^{m}\leq |s|_{y}\cdot|a|_{y}^{m}$. Since $|a|_{\xi}\geq|b|_{\xi}$ but this cannot happen for $\xi$: since $|a|_{\xi}\geq |b|_{\xi}$ we have $|b|_{x}^{2m}\geq|s|_{x}|a|^{2m}_{x}$. Since $a,b\notin\pfrak$ there is $\ell\in\NN$ such that $|c|_{z}\geq|s^{\ell}|_{z}$ when $z\in\{x,y\}$ and $c\in \{a,b\}$. Then $x\in R_{\Spa(A,A^{\circ})}(b^{2m}|sa^{2m},s^{2\ell m})$ and $y\in R_{\Spa(A,A^{\circ})}(sa^{m}|b^{m},s^{\ell m+1})$ and these open subsets are disjoint as $|b^{2m}|_{z}\geq|sa^{2m}|_{z}$ and $|sa^{m}|_{z}\geq|b^{m}|_{z}$ imply $|s^{2}a^{2m}|_{z}\geq|b^{2m}|_{z}$ hence $|a|=|b|=0$ and $0<s<1$.
\end{proof}
We now recollect some additional results from the theory of affinoid algebras. 
\begin{proposition}
    If $A\to B$ is a morphism of affinoid $K$-algebras such that $B$ is finite over $A$ then there is $D_{B/A}\in\NN$ such that for all $b\in B$ there is a polynomial $P(T)=T^{d}+\sum_{i=0}^{d-1}p_{i}T^{i}$ with $p_{i}\in A$ such that $P(b)=0$ and $\Vert p_{i}|A\Vert_{\max}\leq\Vert b|B\Vert^{d-i}_{\max}$ and $d<D_{B/A}$. 
\end{proposition}
\begin{corollary}
    Let $A$ be an affinoid $K$-algebra. Then 
    \begin{align*}
        A^{\circ}=\{a\in A:\Vert a|A\Vert_{\max}\leq 1\} \\
        A^{\circ}=\{a\in A:\Vert a|A\Vert_{\max}\leq 1\}.
    \end{align*}
\end{corollary}
In what follows, we denote $\Kfrak=K^{\circ}/K^{\circ\circ}$ and for any affinoid algebra $A$, $\widetilde{A}=A^{\circ}/A^{\circ\circ}$ considered as an algebra over $\Kfrak$. 
\begin{proposition}
    Let $A\to B$ be a morphism of affinoid $K$-algebras such that $B$ is fintiely generated as an $A$-module. Then $B^{\circ}$ is integral over $A^{\circ}$. 
\end{proposition}
\begin{proposition}\label{prop: integral over for fg modules}
    Let $A\to B$ be a morphism of affinoid $K$-algebras such that $B$ is finitely generated as an $A$-module. Then $\widetilde{B}$ is integral over $\widetilde{A}$. 
\end{proposition}
\begin{proof}
    This is merely the observation that the property of being integral over persists under quotients. 
\end{proof}
\begin{remark}
    $B^{\circ}$ may fail to be finite over $A^{\circ}$, even when $A=K$ and $B$ is a finite field extension of $A$. 
\end{remark}
\begin{corollary}
    Let $A$ be an affinoid $K$-algebra of dimension $d$. There is a homomorphism $\Kfrak[X_{1},\dots,X_{d}]\to \widetilde{A}$ of $\Kfrak$-algebras such that $\widetilde{A}$ is integral over $\Kfrak[X_{1},\dots,X_{d}]$. In particular, if $\pfrak$ is a prime ideal of $\widetilde{A}$ then the transcendence degree of $\kappa(\pfrak)/\Kfrak$ is $d$. 
\end{corollary}
\begin{proof}
    Apply \Cref{prop: integral over for fg modules} in the case of the map $\TT_{d}\to A$ which exists by Noether normalization. 
\end{proof}
For a valuation $\nu$, let $\supp(\nu)=\pfrak\subseteq A$ be its support. Denote $\pfrak_{0}=\{a\in A^{\circ}:\exists\varepsilon\in K^{\circ\circ}\text{ s.t. }\nu(a)<\nu(\varepsilon)\}$. The quotient $\widetilde{\pfrak}=\pfrak_{0}/A^{\circ\circ}$ is prime in $\widetilde{A}$. 
\begin{proposition}\label{prop: Krull dimension statement}
    Let $\nu$ be a valuation and $0=\Gamma_{0}\subsetneq\Gamma_{1}\subsetneq\Gamma_{2}\subsetneq\dots\subsetneq\Gamma_{d}\subsetneq\Gamma$ a tower of convex subgroups. Then $d\leq \dim(A)$.  
\end{proposition}
\begin{proof}
    Without loss of generality, we can assume there is $b_{i}\in A^{\times}$ and $\alpha_{i}=\frac{a_{i}}{b_{i}}\in A^{\circ}$. Using that $\nu(\alpha_{i})\in\Gamma_{i}\setminus\Gamma_{i-1}$, it follows that for $P\in K^{\circ}[X_{1},\dots,X_{d}]/K^{\circ\circ}[X_{1},\dots,X_{d}]$ we have $\nu(P(\alpha_{1},\dots,\alpha_{d}))\in\Gamma_{d}$ a proper convex subgroup. It follows that $P(\alpha_{1},\dots,\alpha_{d})\notin\pfrak_{0}\subseteq A^{\circ}$. Thus the images in $\widetilde{A}$ are algebraically independent over $\Kfrak$. When $d>\dim(A)$ this contradicts a corollary from the Winter semester course. 
\end{proof}
In sum, we have shown the following. 
\begin{theorem}[Huber-Schneider-van der Put]\label{thm: Huber-Schneider-vdP}
    Let $A$ be an affinoid $K$-algebra. There is a bijection between the van der Put points of $\Sp(A)$ and $\Spa(A,A^{\circ})$ via (\ref{eqn: vdP points map}):
    $$(\nu/\sim)\mapsto\xi_{\nu}=\left\{R_{\Sp(A)}(f_{0}|f_{1},\dots,f_{n}):\nu(f_{i})\leq\nu(f_{0}),\forall 1\leq i\leq n\right\}.$$
    In particular, $\Spa(A,A^{\circ})$ is a spectral space with Krull dimension at most that of $A$.
\end{theorem}
\begin{proof}
    The map was shown to be well-defined in \Cref{prop: map from Spa exists} and was shown to be a bijection in \Cref{prop: mutually inverse}. Spectrality of $\Spa(A,A^{\circ})$ is immediate from \Cref{corr: adic spectrum is spectral}. Finally the statement of the Krull dimension is \Cref{prop: Krull dimension statement}. 
\end{proof}
