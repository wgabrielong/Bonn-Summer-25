\section{Lecture 3 -- 8th May 2025}\label{sec: lecture 3}
We prove the locality statement earlier alluded to. As before
$X=\Sp(A)$ for an affinoid $K$-algebra $A$.
\begin{proposition}\label{prop: coherent module is local}\marginpar{Proposition 2.4}
    Let $\Mcal$ be a sheaf of $\Ocal_X$-modules and $\Scal$ 
the sieve on $X$ generated by these $\Omega\in \mathsf{Rat}_X$ 
for which $\Mcal|_{\Omega}\cong \widetilde{M_{\Omega}}$ where $M_{\Omega}$ is a finitely generated $\Ocal_{X}(\Omega)$-module. If $\Scal$ is
a covering sieve, then $\Mcal$ is coherent. 
\end{proposition}
\begin{proof}
    By induction on Laurent order, it suffices to show that for $g\in A$ and $\Omega_{1}=R_{A}(1|g),\Omega_{2}=R_{A}(g|1)$ rational opens of $X$ on which $\Mcal|_{\Omega_{1}}=\widetilde{M_{1}},\Mcal|_{\Omega_{2}}=\widetilde{M_{2}}$
where $M_1, M_2$ are finitely generated $\Ocal_{X}(\Omega_{1}),\Ocal_{X}(\Omega_{2})$-modules respectively, that $\Mcal$ is finitely generated too. 

    By \Cref{corr: generating tuple,corr: generate restriction to Omega2} there are sections $(m_{i})_{i=1}^{n}$ generating $\Mcal$ as an $\Ocal_{X}$-module such that their restrictions to $\Omega_{1},\Omega_{2}$ generate $\Mcal|_{\Omega_{1}},\Mcal|_{\Omega_{2}}$. Consider 
    $$\Kcal=\ker\left(\Ocal_{X}^{n}\xrightarrow{(m_{i})_{i=1}^{N}}\Mcal\right).$$
    We have $\Kcal|_{\Omega_{j}}= \widetilde{K}_{j}$ where 
    $$K_{j}=\ker\left(\Ocal_{X}(\Omega_{j})^{n}\xrightarrow{(m_{i}|_{\Omega_{j}})_{i=1}^{n}} M_{j}\right).$$
    Applying the same reasoning, we have that there are $(k_{i})_{i=1}^{m}$ that generate $\Kcal$ as an $\Ocal_{X}$-module. It follows that $\Mcal$ is the cokernel of 
    $$\Ocal_{X}^{m}\xrightarrow{(k_{i})_{i=1}^{m}}\Ocal_{X}^{n}.$$
    The universal property shows that $\Mcal$ is isomorphic to the cokernel as it is an isomorphism of each $\Omega_{j}$. Since $\widetilde{(-)}$ is exact, we obtain $\Mcal=\widetilde{M}$ where $M$ is the cokernel of $A^{\oplus m}\to A^{\oplus n}$ by the $k_{i}$'s. 
\end{proof}
\begin{remark}\label{rmk: finitely generated modules over sites}
    In general if $\Rcal$ is a sheaf of rings on a site, we say an $\Rcal$-module is finitely generated if there are fintely many global sections such that $\Rcal^{n}\to\Mcal$ is an epimorphism of sheaves. We say $\Mcal$ is locally finitely generated if the objects on which $\Mcal$ is finitely generated form a covering sieve, and $\Mcal$ is coherent if it is locally finitely generated and the kernels of the local maps $\Rcal|_{X}^{n}\to\Mcal|_{X}$ are finitely generated. 
\end{remark}
\begin{remark}
    On a one point space, this is the condition of the kernel being finitely generated. That is, that the ring is a coherent ring. 
\end{remark}
\begin{example}
    Let us consider \Cref{rmk: finitely generated modules over sites} in the setting of $X=\Sp(A),\Rcal=\Ocal_{X},\Mcal=\widetilde{M}$ for $M$ a finitely generated $A$-module. In this case, the kernel of the map $A^{\oplus n}\to M$ generate the kernel sheaf $\Ocal_{\Sp(A)}^{n}\to\Mcal$ so sheaves coherent in the sense of \Cref{def: coherent sheaves} are coherent in the sense of \Cref{rmk: finitely generated modules over sites}. 

    Dually, if $\Mcal$ is coherent in the sense of \Cref{rmk: finitely generated modules over sites}, we can use locality of coherence \Cref{prop: coherent module is local} to observe that the global sections generating $\Mcal$ and the kernel sheaf $\Kcal$ give rise to $A$-modules $M,K$ such that $M$ is the cokernel of $K\to A^{n}$. 
\end{example}
This concludes our discussion of coherent sheaves.

Recall that $\Sp(\TT_{1})$ for $K$ algebraically closed has van der Put points $\xi$ given by the balls of radius $\leq R$ for $R\in|K|\subseteq\RR_{\geq0}$. Denote $\Kfrak_{\leq R}$ of all balls $K_{\leq R}(X)$ for $x\in\Sp(A)$. For a van der Put point $\xi$ of $\Sp(A)$, we can define $M_{\xi}$ to be the set of all $r\in [0,1)\cap |K^{\times}|$ for which there exists an $x\in\Kfrak_{\leq r}\cap\xi$ -- that is, $\xi$ contains a ball of radius $r$. Denote $K_{\leq R}(\xi)$ be set of rational open sets of radius at most $R$ in the van der Put point $\xi$. 
\begin{example}
    If $R$ is arbitrarily small, then $K_{\leq R}(\xi)=\{x\}$. In this case, $x\in\Omega$ if and only if $R(f_{0}|f_{1},\dots,f_{n})=\Omega\in\xi$ if and only if $\nu(f_{0})\geq\nu(f_{i})$ where $\nu(f)=|f(x)|$ for all $1\leq i\leq n$. 
\end{example}
