\section{Lecture 8 -- 11th July 2025}\label{sec: lecture 8}
We will construct the Habiro ring stack. Recall that we have defined the analytic version of the Habiro ring \Cref{def: analytic Habiro ring} as the free $\ZZ((u))_{\square}$-algebra in one variable $q$ with appropriate growth conditions at roots of unity. We will consider a variant of this, where the condition that for all $\varepsilon>0$ the sequence $\left(\frac{u^{\varepsilon n}}{1-q^{n}}\right)_{n\geq0}$ is a nullsequence is removed. This will make the construction easier, but ring stack that we produce will be ill behaved on the locus where the abovementioned condition does not hold. 

Consider the analogous situation in de Rham cohomology. In characteristic zero, the de Rham stack $X^{\mathsf{dR}}$ is defined to be the quotient of $X$ along the formal completion of the diagonal. But in characteristic $p>0$, one either needs to take the quotient by divided powers along the diagona. If not, the ring stack will compute infinitesmal cohomology, and does not recover de Rham cohomology. 

Recalling the discussion from the end of \Cref{sec: lecture 6}, the crux is defining a map $x\mapsto 1-^{\Hab}x$ which will allow us to define the gluing of $(\GG_{m})^{\Hab}$ with itself to $(\A^{1}_{\ZZ})^{\Hab}$. 
\begin{lemma}
    Let $G$ be the subgroup of $\GG_{m,\Hcal^{\an}}^{\an}$ given by 
    $$\{x\in\GG_{m,\Hcal^{\an}}^{\an}:\left((x;q)_{n}\right)_{n\geq0}\text{ has rapid decay}\}\subseteq\GG_{m,\Hcal^{\an}}^{\an}$$
    at $q=\zeta_{m}$ is the overconvergent neighborhood of the subgroup of $m$th roots of unity $\bigcap_{\mu_{m}\subseteq U}U$ where $\mu_{m}=\{x\in\GG_{m,\Hcal^{\an}}^{\an}:x^{m}=1\}$. 
\end{lemma}
\begin{proof}
    Obserbe the symbol $(x;\zeta_{m})_{n}$ is periodic. So it suffices to show that $(1-x^{m})^{n}$ has rapid decay. Equivalently, $1-x^{m}$ lies in an overconvergent neighborhood of zero, and $x$ is in an overconvergent neighborhood of $\mu_{m}$. 
\end{proof}
\begin{example}\label{ex: analytic dR stack}
    Let $q=1$, then $G=(\{1\}\subseteq\GG_{m,\Hcal^{\an}}^{\an})^{\dagger}$ and the quotient $\GG_{m,\Hcal^{\an}}^{\an}/G$ is the analytic de Rham stack of Rodr\'{i}guez Camargo.  
\end{example}
This leads to the expectation that there is an isomorphism $(\A^{1,\an}_{\Hcal^{\an}})^{\Hab}|_{q=1}\cong (\A^{1,\an}_{\Hcal^{\an}})^{\mathsf{dR}}$ between the $q=1$ specialization of the Habiro ring stack $(\A^{1,\an}_{\Hcal^{\an}})^{\Hab}$ and the analytic de Rham stack over the Habiro ring. In particular, at $q=1$, $(1-^{\Hab}x)|_{q=1}=1-x$. More generally, for $m$ arbitrary, we are considering the quotient $\GG_{m,\Hcal^{\an}}^{\an}/(\mu_{m}\subseteq\GG_{m,\Hcal^{\an}}^{\an})^{\dagger}$. Note that by passing to $m$th powers, there should be an isomorphism 
$$% https://q.uiver.app/#q=WzAsNCxbMCwwLCJcXEdHX3ttLFxcSGNhbF57XFxhbn19XntcXGFufS8oXFxtdV97bX1cXHN1YnNldGVxXFxHR197bSxcXEhjYWxee1xcYW59fV57XFxhbn0pXntcXGRhZ2dlcn0iXSxbMiwwLCJcXGxlZnQoXFxHR197bSxcXEhjYWxee1xcYW59fV57XFxhbn1cXHJpZ2h0KV57XFxtYXRoc2Z7ZFJ9fSJdLFswLDEsIngiXSxbMiwxLCJ4XnttfSJdLFswLDEsIlxcc2ltIl0sWzIsMywiIiwwLHsic3R5bGUiOnsidGFpbCI6eyJuYW1lIjoibWFwcyB0byJ9fX1dXQ==
\begin{tikzcd}
	{\GG_{m,\Hcal^{\an}}^{\an}/(\mu_{m}\subseteq\GG_{m,\Hcal^{\an}}^{\an})^{\dagger}} && {\left(\GG_{m,\Hcal^{\an}}^{\an}\right)^{\mathsf{dR}}} \\
	x && {x^{m}}
	\arrow["\sim", from=1-1, to=1-3]
	\arrow[maps to, from=2-1, to=2-3]
\end{tikzcd}$$
so taking $(\A^{1,\an}_{\Hcal^{\an}})^{\Hab}|_{q=\zeta_{m}}$ should recover the analytic de Rham stack $(\A^{1,\an}_{\Hcal^{\an}})^{\mathsf{dR}}$ but the specialization of the operation of $(1-^{\Hab}x)|_{q=\zeta_{m}}$ should be sent to $(1-x^{m})^{1/m}$ on the analytic de Rham stack. 

By virtue of $(\A^{1,\an}_{\Hcal^{\an}})^{\Hab}$ being a ring object, there should be a canonical map $\ZZ\to (\A^{1,\an}_{\Hcal^{\an}})^{\Hab}$ taking an integer $n$ to its Habiroization $n^{\Hab}$ whose specialization at $q=\zeta_{m}$ is $n^{1/m}$ when $n^{\Hab}$ is invertible. 
\begin{example}
    Consider the subring of $\Hcal^{\an}[q^{1/2}]$ consisting of those series such that $(-q^{1/2};q)_{n}$ has rapid decay. Then $(-1)^{\Hab}=q^{1/2}$ as its square is sent to the coset $G$ under the quotient of \Cref{ex: analytic dR stack}. In particular, at $q=\zeta_{m}$, $(-1)^{\Hab}|_{q=\zeta_{m}}=\zeta_{2m}=q^{1/2}$. 
\end{example}
To construct $x\mapsto 1-^{\Hab}x$, we pick $x\in\GG_{m,\Hcal^{\an}}^{\an}\setminus G$ and seek to construct a $G$-torsor together with an injective map to $\GG_{m,\Hcal^{\an}}^{\an}$. Equivalently, this will be a subspace of $\GG_{m}^{\an}$ that is a $G$-torsor. 
\begin{definition}[$1-^{\Hab}x$]\label{def: 1-Habx}
    $1-^{\Hab}x$ is the set of all $y\in\GG_{m,\Hcal^{\an}}^{\an}$ such that the sequence 
    $$\sum_{k,\ell\geq0, k+\ell=n}(-1)^{n}q^{\frac{k(k-1)}{2}+\frac{\ell(\ell-1)}{2}}\begin{bmatrix}
    & n & \\
    n & \ell & n-k-\ell
     \end{bmatrix}_{q}x^{k}y^{\ell}$$
    has rapid decay. 
\end{definition}
The origin of this theory arises from some Cartier duality computations for algebraic groups. 