\section{Lecture 8 -- 11th July 2025}\label{sec: lecture 8}
We will construct the Habiro ring stack. Recall that we have defined the analytic version of the Habiro ring \Cref{def: analytic Habiro ring} as the free $\ZZ((u))_{\square}$-algebra in one variable $q$ with appropriate growth conditions at roots of unity. We will consider a variant of this, where the condition that for all $\varepsilon>0$ the sequence $\left(\frac{u^{\varepsilon n}}{1-q^{n}}\right)_{n\geq0}$ is a nullsequence is removed. This will make the construction easier, but ring stack that we produce will be ill behaved on the locus where the abovementioned condition does not hold. 

Consider the analogous situation in de Rham cohomology. In characteristic zero, the de Rham stack $X^{\mathsf{dR}}$ is defined to be the quotient of $X$ along the formal completion of the diagonal. But in characteristic $p>0$, one either needs to take the quotient by divided powers along the diagona. If not, the ring stack will compute infinitesmal cohomology, and does not recover de Rham cohomology. 

Recalling the discussion from the end of \Cref{sec: lecture 6}, the crux is defining a map $x\mapsto 1-^{\Hab}x$ which will allow us to define the gluing of $(\GG_{m})^{\Hab}$ with itself to $(\A^{1}_{\ZZ})^{\Hab}$. 
\begin{lemma}
    Let $G$ be the subgroup of $\GG_{m,\Hcal^{\an}}^{\an}$ given by 
    $$\{x\in\GG_{m,\Hcal^{\an}}^{\an}:\left((x;q)_{n}\right)_{n\geq0}\text{ has rapid decay}\}\subseteq\GG_{m,\Hcal^{\an}}^{\an}$$
    at $q=\zeta_{m}$ is the overconvergent neighborhood of the subgroup of $m$th roots of unity $\bigcap_{\mu_{m}\subseteq U}U$ where $\mu_{m}=\{x\in\GG_{m,\Hcal^{\an}}^{\an}:x^{m}=1\}$. 
\end{lemma}
\begin{proof}
    Obserbe the symbol $(x;\zeta_{m})_{n}$ is periodic. So it suffices to show that $(1-x^{m})^{n}$ has rapid decay. Equivalently, $1-x^{m}$ lies in an overconvergent neighborhood of zero, and $x$ is in an overconvergent neighborhood of $\mu_{m}$. 
\end{proof}
\begin{example}\label{ex: analytic dR stack}
    Let $q=1$, then $G=(\{1\}\subseteq\GG_{m,\Hcal^{\an}}^{\an})^{\dagger}$ and the quotient $\GG_{m,\Hcal^{\an}}^{\an}/G$ is the analytic de Rham stack of Rodr\'{i}guez Camargo.  
\end{example}
This leads to the expectation that there is an isomorphism $(\A^{1,\an}_{\Hcal^{\an}})^{\Hab}|_{q=1}\cong (\A^{1,\an}_{\Hcal^{\an}})^{\mathsf{dR}}$ between the $q=1$ specialization of the Habiro ring stack $(\A^{1,\an}_{\Hcal^{\an}})^{\Hab}$ and the analytic de Rham stack over the Habiro ring. In particular, at $q=1$, $(1-^{\Hab}x)|_{q=1}=1-x$. More generally, for $m$ arbitrary, we are considering the quotient $\GG_{m,\Hcal^{\an}}^{\an}/(\mu_{m}\subseteq\GG_{m,\Hcal^{\an}}^{\an})^{\dagger}$. Note that by passing to $m$th powers, there should be an isomorphism 
$$% https://q.uiver.app/#q=WzAsNCxbMCwwLCJcXEdHX3ttLFxcSGNhbF57XFxhbn19XntcXGFufS8oXFxtdV97bX1cXHN1YnNldGVxXFxHR197bSxcXEhjYWxee1xcYW59fV57XFxhbn0pXntcXGRhZ2dlcn0iXSxbMiwwLCJcXGxlZnQoXFxHR197bSxcXEhjYWxee1xcYW59fV57XFxhbn1cXHJpZ2h0KV57XFxtYXRoc2Z7ZFJ9fSJdLFswLDEsIngiXSxbMiwxLCJ4XnttfSJdLFswLDEsIlxcc2ltIl0sWzIsMywiIiwwLHsic3R5bGUiOnsidGFpbCI6eyJuYW1lIjoibWFwcyB0byJ9fX1dXQ==
\begin{tikzcd}
	{\GG_{m,\Hcal^{\an}}^{\an}/(\mu_{m}\subseteq\GG_{m,\Hcal^{\an}}^{\an})^{\dagger}} && {\left(\GG_{m,\Hcal^{\an}}^{\an}\right)^{\mathsf{dR}}} \\
	x && {x^{m}}
	\arrow["\sim", from=1-1, to=1-3]
	\arrow[maps to, from=2-1, to=2-3]
\end{tikzcd}$$
so taking $(\A^{1,\an}_{\Hcal^{\an}})^{\Hab}|_{q=\zeta_{m}}$ should recover the analytic de Rham stack $(\A^{1,\an}_{\Hcal^{\an}})^{\mathsf{dR}}$ but the specialization of the operation of $(1-^{\Hab}x)|_{q=\zeta_{m}}$ should be sent to $(1-x^{m})^{1/m}$ on the analytic de Rham stack. 

By virtue of $(\A^{1,\an}_{\Hcal^{\an}})^{\Hab}$ being a ring object, there should be a canonical map $\ZZ\to (\A^{1,\an}_{\Hcal^{\an}})^{\Hab}$ taking an integer $n$ to its Habiroization $n^{\Hab}$ whose specialization at $q=\zeta_{m}$ is $n^{1/m}$ when $n^{\Hab}$ is invertible. 
\begin{example}
    Consider the subring of $\Hcal^{\an}[q^{1/2}]$ consisting of those series such that $(-q^{1/2};q)_{n}$ has rapid decay. Then $(-1)^{\Hab}=q^{1/2}$ as its square is sent to the coset $G$ under the quotient of \Cref{ex: analytic dR stack}. In particular, at $q=\zeta_{m}$, $(-1)^{\Hab}|_{q=\zeta_{m}}=\zeta_{2m}=q^{1/2}$. 
\end{example}
To construct $x\mapsto 1-^{\Hab}x$, we pick $x\in\GG_{m,\Hcal^{\an}}^{\an}\setminus G$ and seek to construct a $G$-torsor together with an injective map to $\GG_{m,\Hcal^{\an}}^{\an}$. Equivalently, this will be a subspace of $\GG_{m}^{\an}$ that is a $G$-torsor. 
\begin{definition}[$1-^{\Hab}x$]\label{def: 1-Habx}
    $1-^{\Hab}x$ is the set of all $y\in\GG_{m,\Hcal^{\an}}^{\an}$ such that the sequence 
    $$\sum_{k,\ell\geq0, k+\ell=n}(-1)^{n}q^{\frac{k(k-1)}{2}+\frac{\ell(\ell-1)}{2}}\left[\substack{n \\ n \hspace{0.3cm}\ell\hspace{0.3cm}n-k-\ell}\right]_{q}x^{k}y^{\ell}$$
    has rapid decay. 
\end{definition}
The origin of this theory arises from some Cartier duality computations for algebraic groups: we seek to produce a map $\GG_{m,\Hcal}^{\an}\setminus G\to \GG_{m}/G$, but we can identify the latter with $\Hom_{\mathsf{add}}(G^{\vee}/q^{\ZZ},B\GG_{m})$ of additive homomorphisms where $G^{\vee}$ is the Cartier dual of $G$. We can write 
$$G^{\vee}=\bigcup_{k>0}\AnSpec\left(\Hcal^{\an}\left\langle u^{kn}\frac{(T,q)_{n}}{(q;q)_{n}}\right\rangle^{\dagger}\right).$$
We then want to specify a map $\GG_{m}^{\an}\setminus G\to\Hom_{\mathsf{add}}(G^{\vee}/q^{\ZZ},B\GG_{m})$ instead. Using that $\Hom_{\mathsf{add}}(G^{\vee}/q^{\ZZ},B\GG_{m})\subseteq\Mor_{*}(G^{\vee}/q^{\ZZ},B\GG_{m})$ of pointed maps of stacks on which a certain line bundle is trivial. So for $x\in \GG_{m}^{\an}\setminus G$, we can construct a line bundle on $G^{\vee}/q^{\ZZ}$ by taking the trivial line bundle on $G^{\vee}$ and prescribe it descent data to the quotient where $q$ acts by multiplication by $1-x$ as the composite 
$$% https://q.uiver.app/#q=WzAsNixbMCwwLCJHXntcXHZlZX0iXSxbMiwwLCJcXEdHX3ttLFxcSGNhbF57XFxhbn19XntcXG1hdGhzZnthbn19XFxzZXRtaW51cyBHIl0sWzQsMCwiXFxHR197bSxcXEhjYWxee1xcYW59fV57XFxhbn0iXSxbMCwxLCJnIl0sWzIsMSwiZ3giXSxbNCwxLCIxLV57XFxIYWJ9Z3guIl0sWzAsMV0sWzEsMl0sWzMsNCwiIiwwLHsic3R5bGUiOnsidGFpbCI6eyJuYW1lIjoibWFwcyB0byJ9fX1dLFs0LDUsIiIsMCx7InN0eWxlIjp7InRhaWwiOnsibmFtZSI6Im1hcHMgdG8ifX19XV0=
\begin{tikzcd}
	{G^{\vee}} && {\GG_{m,\Hcal^{\an}}^{\mathsf{an}}\setminus G} && {\GG_{m,\Hcal^{\an}}^{\an}} \\
	g && gx && {1-^{\Hab}gx.}
	\arrow[from=1-1, to=1-3]
	\arrow[from=1-3, to=1-5]
	\arrow[maps to, from=2-1, to=2-3]
	\arrow[maps to, from=2-3, to=2-5]
\end{tikzcd}$$
The action of $q^{n}$ in general given by multiplication by $(1-gx)(1-qgx)\dots(1-q^{n-1}gx)=(gx;q)_{n}$. Recall that $(t;q)_{\infty}$ was a section of a line bundle on $\Hcal_{\ZZ[t^{\pm},\frac{1}{1-t}]\ZZ[t]}$ and $(1-t;q)_{\infty}$ its inverse defined on $\Hcal_{\ZZ[t^{\pm},\frac{1}{1-t}]/\ZZ[1-t]}$. So the product $(t;q)_{\infty}(t-1;q)_{\infty}$ should actually be a function on the $G$ torsor $$(\GG_{m,\Hcal^{\an}}^{\an}\setminus G)\times_{(\GG_{m}\setminus\{1\})^{\Hab}} (\GG_{m,\Hcal^{\an}}^{\an}\setminus G)$$ which solves two $q$-difference equations, viewing both the Habiro rings as $G$-torsors $\GG_{m,\Hcal^{\an}}^{\an}\setminus G\to (\GG_{m}\setminus\{1\})^{\Hab}$. Conversely, solutions to those two $q$-difference equations will give rise to sections of line bundles on the product $G^{\vee}/q^{\ZZ}\times G^{\vee}/q^{\ZZ}$. 

This construction gives rise to a ring stack defining $x-y=x(1-\frac{y}{x})$ and we can verify the ring axioms. The only nontrivial verification here is for the associativity, which is given by the five-term relation for the dilogarithm. 
\begin{remark}
    Alternatively, $(\A^{1,\an}_{\Hcal^{\an}})\cong\A^{1,\an}_{\Hcal^{\an}}/\sim$ where $x\sim y$ if $(x;y)_{n}$ has rapid decay for all $n$. 
\end{remark}
\begin{example}\marginpar{This completes the computation alluded to at the end of \Cref{sec: lecture 1}.}
    $1+^{\Hab}1=2^{\Hab}$ is the set of all $y\in\Hcal^{\an}$ such that
    $$\sum_{m=0}^{n}(-1)^{m}\begin{bmatrix}
        n \\ m
    \end{bmatrix}_{q}q^{\frac{m(m-1)}{2}}x^{m}(y;q)_{n-m}$$
    has rapid decay.
\end{example}