\documentclass{amsart}
\usepackage[margin=1.5in]{geometry} 
\usepackage{amsmath}
\usepackage{tcolorbox}
\usepackage{amssymb}
\usepackage{amsthm}
\usepackage{lastpage}
\usepackage{fancyhdr}
\usepackage{accents}
\usepackage{hyperref}
\usepackage{xcolor}
\usepackage{color}
\usepackage[bbgreekl]{mathbbol}
\DeclareSymbolFontAlphabet{\mathbb}{AMSb}
\DeclareSymbolFontAlphabet{\mathbbl}{bbold}
% Fields
\newcommand{\A}{\mathbb{A}}
\newcommand{\CC}{\mathbb{C}}
\newcommand{\EE}{\mathbb{E}}
\newcommand{\RR}{\mathbb{R}}
\newcommand{\QQ}{\mathbb{Q}}
\newcommand{\ZZ}{\mathbb{Z}}
\newcommand{\HH}{\mathbb{H}}
\newcommand{\KK}{\mathbb{K}}
\newcommand{\NN}{\mathbb{N}}
\newcommand{\FF}{\mathbb{F}}
\newcommand{\PP}{\mathbb{P}}
\newcommand{\GG}{\mathbb{G}}
\newcommand{\LL}{\mathbb{L}}
\newcommand{\WW}{\mathbb{W}}

% mathcal letters
\newcommand{\Acal}{\mathcal{A}}
\newcommand{\Bcal}{\mathcal{B}}
\newcommand{\Ccal}{\mathcal{C}}
\newcommand{\Dcal}{\mathcal{D}}
\newcommand{\Ecal}{\mathcal{E}}
\newcommand{\Fcal}{\mathcal{F}}
\newcommand{\Gcal}{\mathcal{G}}
\newcommand{\Hcal}{\mathcal{H}}
\newcommand{\Ical}{\mathcal{I}}
\newcommand{\Jcal}{\mathcal{J}}
\newcommand{\Kcal}{\mathcal{K}}
\newcommand{\Lcal}{\mathcal{L}}
\newcommand{\Mcal}{\mathcal{M}}
\newcommand{\Ncal}{\mathcal{N}}
\newcommand{\Ocal}{\mathcal{O}}
\newcommand{\Pcal}{\mathcal{P}}
\newcommand{\Qcal}{\mathcal{Q}}
\newcommand{\Rcal}{\mathcal{R}}
\newcommand{\Scal}{\mathcal{S}}
\newcommand{\Tcal}{\mathcal{T}}
\newcommand{\Ucal}{\mathcal{U}}
\newcommand{\Vcal}{\mathcal{V}}
\newcommand{\Wcal}{\mathcal{W}}
\newcommand{\Xcal}{\mathcal{X}}
\newcommand{\Ycal}{\mathcal{Y}}
\newcommand{\Zcal}{\mathcal{Z}}

% abstract categories
\newcommand{\Asf}{\mathsf{A}}
\newcommand{\Bsf}{\mathsf{B}}
\newcommand{\Csf}{\mathsf{C}}
\newcommand{\Dsf}{\mathsf{D}}
\newcommand{\Esf}{\mathsf{E}}
\newcommand{\Ssf}{\mathsf{S}}

% algebraic geometry
\newcommand{\spec}{\operatorname{Spec}}
\newcommand{\proj}{\operatorname{Proj}}

% categories 
\newcommand{\id}{\mathrm{id}}
\newcommand{\Obj}{\mathrm{Obj}}
\newcommand{\Mor}{\mathrm{Mor}}
\newcommand{\Hom}{\mathrm{Hom}}
\newcommand{\Ext}{\mathrm{Ext}}
\newcommand{\Aut}{\mathrm{Aut}}
\newcommand{\Sets}{\mathsf{Sets}}
\newcommand{\SSets}{\mathsf{SSets}}
\newcommand{\kVect}{\mathsf{Vect}_{k}}
\newcommand{\Vect}{\mathsf{Vect}}
\newcommand{\Alg}{\mathsf{Alg}}
\newcommand{\Ring}{\mathsf{Ring}}
\newcommand{\Mod}{\mathsf{Mod}}
\newcommand{\Grp}{\mathsf{Grp}}
\newcommand{\AbGrp}{\mathsf{AbGrp}}
\newcommand{\PSh}{\mathsf{PSh}}
\newcommand{\Sh}{\mathsf{Sh}}
\newcommand{\PSch}{\mathsf{PSch}}
\newcommand{\Sch}{\mathsf{Sch}}
\newcommand{\Top}{\mathsf{Top}}
\newcommand{\Com}{\mathsf{Com}}
\newcommand{\Coh}{\mathsf{Coh}}
\newcommand{\QCoh}{\mathsf{QCoh}}
\newcommand{\Opens}{\mathsf{Opens}}
\newcommand{\Opp}{\mathsf{Opp}}
\newcommand{\Cat}{\mathsf{Cat}}
\newcommand{\NatTrans}{\mathrm{NatTrans}}
\newcommand{\pr}{\mathrm{pr}}
\newcommand{\Fun}{\mathrm{Fun}}
\newcommand{\colim}{\mathrm{colim}}
\newcommand{\lifts}{\boxslash}
\DeclareMathOperator\squarediv{\lifts}
\newcommand{\Kan}{\mathsf{Kan}}
\newcommand{\Path}{\mathsf{Path}}
\newcommand{\SPSh}{\mathsf{SPSh}}
\newcommand{\SSh}{\mathsf{SSh}}
\newcommand{\Bord}{\mathsf{Bord}}

% simplicial sets
\newcommand{\DDelta}{\Updelta}
\newcommand{\Sing}{\operatorname{Sing}}

% ideal theory
\newcommand{\mfrak}{\mathfrak{m}}
\newcommand{\afrak}{\mathfrak{a}}
\newcommand{\bfrak}{\mathfrak{b}}
\newcommand{\pfrak}{\mathfrak{p}}
\newcommand{\qfrak}{\mathfrak{q}}

% number theory
\newcommand{\Tr}{\mathrm{Tr}}
\newcommand{\Nm}{\mathrm{Nm}}
\newcommand{\Gal}{\mathrm{Gal}}
\newcommand{\Frob}{\mathrm{Frob}}

\newcommand{\SL}{\mathrm{SL}}
\newcommand{\GL}{\mathrm{GL}}
\newcommand{\Li}{\mathrm{Li}}
\newcommand{\sfPic}{\mathsf{Pic}}
\newcommand{\img}{\mathrm{Im}}
\newcommand{\Reg}{\mathrm{Reg}}
\newcommand{\Cscr}{\EuScript{C}}
\newcommand{\Dscr}{\EuScript{D}}
\newcommand{\Ani}{\mathsf{Ani}}
\newcommand{\Proj}{\mathsf{Proj}}
\newcommand{\Free}{\mathsf{Free}}
\newcommand{\CMon}{\mathsf{CMon}}
\newcommand{\cond}{\mathsf{cond}}
\newcommand{\cont}{\mathsf{cont}}
\newcommand{\Liq}{\mathsf{Liq}}
\newcommand{\Gas}{\mathsf{Gas}}
\newcommand{\ku}{\mathsf{ku}}
\newcommand{\KU}{\mathsf{KU}}
\newcommand{\SSS}{\mathbb{S}}
\newcommand{\univ}{\mathrm{univ}}
\newcommand{\DMod}{\mathsf{DMod}}
\newcommand{\FGauge}{\mathsf{FGauge}}
\newcommand{\prism}{\mathbbl{\Delta}}
\newcommand{\Hab}{\mathsf{Hab}}
\newcommand{\Hdg}{\mathsf{Hdg}}
\newcommand{\HHdg}{\mathcal{H}\mathsf{dg}}
\newcommand{\dash}{\text{-}}
\newcommand{\gh}{\mathrm{gh}}
\newcommand{\CAlg}{\mathsf{CAlg}}
\newcommand{\Fil}{\mathrm{Fil}}
\newcommand{\THH}{\mathrm{THH}}
\newcommand{\an}{\mathsf{an}}
\newcommand{\Berk}{\mathsf{Berk}}
\newcommand{\Betti}{\mathsf{Betti}}
\newcommand{\AnSpec}{\mathrm{AnSpec}}
\setlength{\headheight}{40pt}


\newenvironment{solution}
  {\renewcommand\qedsymbol{$\blacksquare$}
  \begin{proof}[Solution]}
  {\end{proof}}
\renewcommand\qedsymbol{$\blacksquare$}

\usepackage{amsmath, amssymb, tikz, amsthm, csquotes, multicol, footnote, tablefootnote, biblatex, wrapfig, float, quiver, mathrsfs, cleveref, enumitem, stmaryrd, marginnote, todonotes, euscript, bbm}
\addbibresource{refs.bib}
\theoremstyle{definition}
\newtheorem{theorem}{Theorem}[section]
\newtheorem{lemma}[theorem]{Lemma}
\newtheorem{corollary}[theorem]{Corollary}
\newtheorem{exercise}[theorem]{Exercise}
\newtheorem{question}[theorem]{Question}
\newtheorem{example}[theorem]{Example}
\newtheorem{proposition}[theorem]{Proposition}
\newtheorem{conjecture}[theorem]{Conjecture}
\newtheorem{remark}[theorem]{Remark}
\newtheorem{definition}[theorem]{Definition}
\numberwithin{equation}{section}
\setuptodonotes{color=blue!20, size=tiny}
\begin{document}
\large
\title[Habiro Cohomology -- Bonn, Summer 2025]{V5A4 -- Habiro Cohomology \\ Summer Semester 2025}
\author{Wern Juin Gabriel Ong}
\address{Universit\"{a}t Bonn, Bonn, D-53113}
\email{wgabrielong@uni-bonn.de}
\urladdr{https://wgabrielong.github.io/}
\maketitle
\section*{Preliminaries}
These notes roughly correspond to the course \textbf{V5A4 -- Habiro Cohomology} taught by Prof. Peter Scholze at the Universit\"{a}t Bonn in the Summer 2025 semester. These notes are \LaTeX-ed after the fact with significant alteration and are subject to misinterpretation and mistranscription. Use with caution. Any errors are undoubtedly my own and any virtues that could be ascribed to these notes ought be attributed to the instructor and not the typist. Recordings of the lecture are availible at the following link:
\begin{center}
  \href{https://archive.mpim-bonn.mpg.de/id/eprint/5155/}{\texttt{archive.mpim-bonn.mpg.de/id/eprint/5155/}}
\end{center}

\tableofcontents
\section{Lecture 1 -- 17th April 2025}\label{sec: lecture 1}
We fix the following notation. 
\begin{notation}
    \begin{enumerate}[label=(\roman*)]
        \item $K$ is a field complete with respect to a non-Archimedean norm. 
        \item We denote the Tate algebra 
        $$\TT_{n}=\left\{f\in\sum_{\alpha\in\NN^{n}}f_{\alpha}X^{\alpha}:\forall\varepsilon>0, |\{\alpha:|f_{\alpha}|_{K}\geq\varepsilon\}|<\infty\right\}\subseteq K[[X_{1},\dots,X_{n}]]$$
        the subring of convergent power series, with norm $\Vert f|\TT_{n}\Vert=\max_{\alpha\in\NN^{n}}|f_{\alpha}|$. 
    \end{enumerate}
\end{notation}
\begin{remark}
    \begin{itemize}
        \item [(i)] The norm $|\cdot|_{K}$ extends uniquely to any algebraic extension of $K$. 
        \item [(ii)] The Tate algebra $\TT_{n}$ is Noetherian, hence all ideals are closed. 
    \end{itemize}
\end{remark}
Affinoid algebras are quotients of Tate algebras. 
\begin{definition}[Affiniod Algebra]\label{def: affinoid algebra}
    A $K$-algebra $A$ is an affinoid $K$-algebra if it is of the form $\TT_{n}/I$. 
\end{definition}
There is an induced norm on the Tate algebra known as the residual norm. 
\begin{definition}[Residual Norm]\label{def: residual norm}
    Let $A$ be an affinoid $K$-algebra. The residue norm of $a\in A$ is 
    $$\Vert a\Vert=\inf\{\Vert f|\TT_{n}\Vert:\overline{f}=a\}.$$
\end{definition}
\begin{remark}
    \Cref{def: residual norm} is independent of the choice of representative. 
\end{remark}
As in algebraic geometry, affinoid algebras give rise to ringed spaces via the Tate spectrum. We discuss the construction by first defining the space, and the sheaf of rings on it. 
\begin{definition}[Tate Spectrum -- Set]\label{def: space of Tate spectrum}
    Let $A$ be an affinoid $K$-algebra. The set underlying the Tate spectrum $\Sp(A)$ is $\mathrm{mSpec}(A)$. 
\end{definition}
\begin{remark}
    The Tate spectrum is endowed with the property that $[\kappa(x):K]<\infty$, where $\kappa(x)=A/\mfrak_{x}$ is a field as the ideal $\mfrak_{x}$ corresponding to $x$ is maximal. 
\end{remark}
The topology on the set is defined by rational sieves. 
\begin{definition}[Rational Open Set]\label{def: rational set}
    Let $\langle f_{0},\dots,f_{n}\rangle_{A}=A$. The rational open associated to the generators $R_{A}(f_{0}|f_{1},\dots,f_{n})$ is given by 
    $$R_{A}(f_{0}|f_{1},\dots,f_{n})=\{x\in\Sp(A):|f_{i}(x)|<|f_{0}(x)|,1\leq i\leq n\}.$$
\end{definition}
\begin{remark}
    Rational open subsets are preserved under finite intersection. For $\langle f_{0},\dots,f_{n}\rangle_{A},\langle g_{0},\dots,g_{m}\rangle_{A}$ generators of $A$, the intersection 
    $$R_{A}(f_{0}|f_{1},\dots,f_{n})\cap R_{A}(g_{0}|g_{1},\dots,g_{m})=R_{A}(f_{0}g_{0}|f_{i}g_{j}, 1\leq i\leq n, 1\leq j\leq m).$$
\end{remark}
These rational open sets form the basis for the topology on the Tate spectrum $\Sp(A)$. 
\begin{definition}[Tate Spectrum -- Topology]\label{def: topology of Tate spectrum}
    Let $A$ be an affinoid $K$-algebra. The set underlying the Tate spectrum $\Sp(A)$ has a topology with basis consisting of the rational open sets $R_{A}(f_{0}|f_{1},\dots,f_{n})$ and with Grothendieck topology obtained by enforcing quasicompactness of the rational open sets. 
\end{definition}
In some simple cases, the underlying space of the Tate spectrum admits a description. 
\begin{example}
    Let $K=\overline K$. $\Sp(\TT_{n})= (K^{\circ})^{n}$, where $K^{\circ}$ is the subring of powerbounded elements of $K$. Each point $x\in\Sp(A)$ is taken to $(\xi_{i})_{i=1}^{n}$ where $\xi_{i}$ is the image of $X_{i}$ in $K\cong\kappa(x)$ and an $n$-tuple of powerbounded elements of $K$ is taken to the ideal of $\TT_{n}$ consisting of functions vanishing at that tuple. In this case, the basis for the ordinary topology on the Tate spectrum is identified with non-Archimedean balls $d(\xi,\nu)=\max_{1\leq i\leq n}|\xi_{i}-\nu_{i}|$. 
\end{example}
We now want to define the structure sheaf on $\Sp(A)$ which will be valued in the category affinoid $K$-algebras $\Aff_{K}$. This is a full subcategory of the category of $K$-algebras as all maps between affinoid $K$-algebras are automatically continuous. 

The structure sheaf is defined as follows. 
\begin{definition}[Tate Spectrum -- Structure Sheaf]\label{def: structure sheaf of the Tate spectrum}
    Let $A$ be an affinoid $K$-algebra. The functor $$R_{A}(f_{0}|f_{1},\dots,f_{n})\mapsto A\left\langle\frac{\varepsilon}{f_{0}}\right\rangle\left\langle\frac{f_{1}}{f_{0}},\dots,\frac{f_{n}}{f_{0}}\right\rangle$$ where $\varepsilon\in K^{\times}$ such that $\max_{0\leq i\leq n}|f_{i}(x)|\geq|\varepsilon|$ for all $x\in\Sp(A)$ represents the functor $\mathsf{Rat}_{A}^{\Opp}\to\Aff_{K}$
    $$F_{\Omega}(B)=\left\{\varphi\in\Hom_{\Aff_{K}}(A,B):\Sp(\varphi)(\Sp(B))\subseteq\Omega\right\},$$
where $\Omega=\Ocal_{\Sp(A)}(A\left\langle\frac{\varepsilon}{f_{0}}\right\rangle\left\langle\frac{f_{1}}{f_{0}},\dots,\frac{f_{n}}{f_{0}}\right\rangle$.
\end{definition}
Summing up the preceding constructions, we have:
\begin{definition}[Tate Spectrum -- Ringed Space]\label{def: Tate spectrum as a ringed space}
    Let $A$ be an affinoid $K$-algebra. The Tate spectrum is given by:
    \begin{itemize}
        \item Topological space $\mathrm{mSpec}(A)$ with basis for the topology given by rational open subsets $R_{A}(f_{0}|f_{1},\dots,f_{n})$ with $\langle f_{0},\dots,f_{n}\rangle_{A}=A$. 
        \item Sheaf of rings given by $R_{A}(f_{0}|f_{1},\dots,f_{n})\mapsto A\left\langle\frac{\varepsilon}{f_{0}}\right\rangle\left\langle\frac{f_{1}}{f_{0}},\dots,\frac{f_{n}}{f_{0}}\right\rangle$. 
    \end{itemize}
\end{definition}
Here we used the fact that any sheaf on the base extends to a sheaf on the space. 
\begin{remark}
    \begin{enumerate}[label=(\roman*)]
        \item There are identifications $\Ocal_{\Sp(\Ocal_{\Sp(A)}(\Omega))}\cong\Ocal_{\Sp(A)}|_{\Omega}$. 
        \item By Tate acyclicity, the higher cohomology of $\Ocal_{\Sp(A)}$ vanishes.
    \end{enumerate} 
\end{remark}
We state some additional results surrounding Tate acyclicity. 
\begin{definition}[Laurent Order]\label{def: Laurent order}
    Let $\Scal$ be a sieve on $\Sp(A)$. We define the Laurent order $\ofrak_{L}(\Scal)$ inductively as follows:
    \begin{itemize}
        \item $\ofrak_{L}(\Scal)=0$ if and only if it is the all sieve. 
        \item $\ofrak_{L}(\Scal)\leq k$ if there is $g\in\Ocal_{X}(\Omega)$ such that the restriction sieves $\Scal|_{R_{\Omega}(g|1)}$ and $\Scal|_{R_{\Omega}(1|g)}$ have Laurent order at most $k$. 
        \item $\Scal$ is of Laurent order $k$ if $k$ is the smallest number such that $\Scal$ is of Laurent order at most $k$
    \end{itemize}
\end{definition}
Finiteness of the Laurent order characterizes covering sieves. 
\begin{proposition}\label{prop: finite Laurent order iff covering}
    Let $\Scal$ be a sieve on $\Sp(A)$ for $A$ an affinoid $K$-algebra. $\Scal$ is a covering sieve if and only if $\ofrak_{L}(\Scal)<\infty$. 
\end{proposition}
This immediately gives a simple sufficient condition for Tate acyclicity. 
\begin{corollary}\label{def: two set condition for acyclicity}
    Let $\Fcal$ be a sheaf of Abelian groups on $\Sp(A)$. If 
    $$0\to\Fcal(\Omega)\to\Fcal(R_{\Omega}(g|1))\oplus\Fcal(R_{\Omega}(1|g))\to\Fcal(R_{\Omega}(g|1)\cap R_{\Omega}(1|g))\to0$$
    is exact for all $\Omega\subseteq\Sp(A)$ rational and $g\in\Ocal_{\Sp(A)}(\Omega)$ then $\Fcal$ is acyclic. 
\end{corollary}
We state two additional results concerning the unviersality of certain affinoid $K$-algebras. We first recall the following definitions. 
\begin{definition}[nat Ring]\label{def: nat ring}
    Let $A$ be a topological ring. $A$ is a nat ring if it has a basis of neighborhoods of zero consisting of open subgroups. 
\end{definition}
\begin{definition}[Tate Ring]\label{def: Tate ring}
    A nat ring $A$ is Tate if if it has a powerbounded neighborhood of zero and has a topologically nilpotent unit known as a quasi-uniformizer. 
\end{definition}
In turn:
\begin{proposition}\label{prop: universality of quotients}
    Let $A$ be a Tate ring and 
    $$A\langle f_{1},\dots,f_{n}\rangle=A\langle X_{1},\dots,X_{n}\rangle/\langle X_{1}-f_{1},\dots,X_{n}-f_{n}\rangle.$$
    $A\langle f_{1},\dots,f_{n}\rangle$ is initial among nat $A$-algebras $B$ where $f_{1},\dots,f_{n}$ are powerbounded. Furthermore, $A\langle f_{1},\dots,f_{n}\rangle$ contains $A$ as a dense subring. 
\end{proposition}
\begin{proposition}\label{prop: universality of localizations}
    Let $A$ be a Tate ring and 
    $$A\left\langle\frac{1}{f_{1}},\dots,\frac{1}{f_{n}}\right\rangle=A\langle X_{1},\dots,X_{n}\rangle/\left\langle X_{1}-\frac{1}{f_{1}},\dots,X_{n}-\frac{1}{f_{n}}\right\rangle.$$
    $A\langle\frac{1}{f_{1}},\dots,\frac{1}{f_{n}}\rangle$ is initial among nat $A$-algebras $B$ where $f_{1},\dots,f_{n}$ are units with $\frac{1}{f_{1}},\dots,\frac{1}{f_{n}}$ powerbounded. Furthermore, $A\langle\frac{1}{f_{1}},\dots,\frac{1}{f_{n}}\rangle$ contains $A[\frac{1}{f_{1}},\dots,\frac{1}{f_{n}}]$ as a dense subring. 
\end{proposition}
We are now ready to define coherent sheaves. 

We begin with the following preparatory result. \marginpar{We now begin marginal labeling, which follows the lecture. \\\\ Proposition 2.1}
\begin{proposition}\label{prop: flatness of section algebras}
    Let $A$ be an affinoid algebra and $\Omega\subseteq\Sp(A)$ a rational subset. Then:
    \begin{enumerate}[label=(\roman*)]
        \item For $B=\Ocal_{\Sp(A)}(\Omega)$ and $\mfrak\in\Sp(B)$, there is an isomorphism of $K$-algebras $B_{\mfrak}^{\wedge}\cong A_{\widetilde{\mfrak}}^{\wedge}$ where $\widetilde{\mfrak}$ is the preimage of $\mfrak$ under the map $A\to B$ and $(-)^{\wedge}_{I}$ is the completion of a ring with respect to the ideal $I$. 
        \item $B$ is flat as an $A$-algebra. 
    \end{enumerate}
\end{proposition}
\begin{proof}[Proof of (i)]
    We first show a claim:
    \begin{itemize}
        \item [($\dagger$)] For all $n\in\NN$, $A/\widetilde{\mfrak}^{n}\to B/\mfrak^{n}$ is an isomorphism. 
    \end{itemize}
    Note that $B/\mfrak^{n}$ is initial amongst $B$-algebras $C$ such that $\mfrak^{n}C=0$, while $A/\mfrak^{n}$ is initial amongst $A$-algebras $\tilde{C}$ such that $\mfrak^{n}\tilde{C}=0$. For $C$ as above, the image of $\Sp(C)$ in $\Sp(B)$ is $\mfrak$, while the image of $\Sp(\tilde{C})$ in $\Sp(A)$ is $\widetilde{\mfrak}\in\Omega$. Applying the universal property twice, 
$\tilde{C}$ can be endowed uniquely with the structure of a $B$-algebra, and by $\kappa(\mfrak)\cong\kappa(\widetilde{\mfrak})$ it follows that $\tilde{C}$ is annihilated by $\widetilde{\mfrak}$. Thus both $A/\widetilde{\mfrak}^{n}, B/\mfrak^{n}$ satisfy the same universal property, hence are isomorphic. 

    The desired claim follows from ($\dagger$) by passage to the limit. 
\end{proof}
\begin{proof}[Proof of (ii)]
    By a standard result in commutative algebra, it suffices to show $B_{\mfrak}$ is $A$-flat for all $\mfrak\in\mathrm{mSpec}(B)$. $B$ being Noetherian, $B^{\wedge}_{\mfrak}$ is a faithfully flat $B_{\mfrak}$-algebra, whereby it is sufficient to show that $B^{\wedge}_{\mfrak}$ is flat over $A$. But $B^{\wedge}_{\mfrak}\cong A^{\wedge}_{\widetilde{\mfrak}}$ by (i), which is a flat $A$-module as $A$ is Noetherian, giving the claim. 
\end{proof}
As in the case of algebraic geometry, coherent sheaves are defined as $\widetilde{(-)}$-ifications of finitely generated modules. 
\begin{definition}[$\widetilde{(-)}$]\label{def: tildeification}\marginpar{Definition 2.1}
    Let $A$ be an affinoid $K$-algebra and $M$ a finitely generated $A$-module. The sheaf $\widetilde{M}$ is the sheafification of the presheaf $\Omega\mapsto M\otimes_{A}\Ocal_{\Sp(A)}(\Omega)$ on rational open subsets. 
\end{definition}
Since each module in the exact sequence 
$$0\to\Ocal_{\Sp(A)}(\Omega)\to\Ocal_{\Sp(A)}(R_{\Omega}(1|g))\oplus\Ocal_{\Sp(A)}(R_{\Omega}(g|1))\to\Ocal_{\Sp(A)}(R_{\Omega}(g|1,g^{2}))\to0$$
is flat, exactness is preserved under $-\otimes_{A}M$.
In particular, by \Cref{def: two set condition for acyclicity} we have:
\begin{proposition}\label{prop: tildeification is tensor}\marginpar{Proposition 2.2}
    Let $A$ be an affinoid $K$-algebra and $M$ a finitely generated $A$-module with associated sheaf $\widetilde{M}$. Then $H^{p}(\Omega,\widetilde{M})=0$ for all $p>0$ and for all rational $\Omega$. 
\end{proposition}
\begin{proof}
    By flatness of $\Ocal_{\Sp(A)}(\Omega)$ as an $A$-algebra, exactness of the sequence for $\Omega$ rational, we have a short exact sequence 
    $$0\to\Ocal_{\Sp(A)}(\Omega)\to\Ocal_{\Sp(A)}(\Omega_{1})\oplus\Ocal_{\Sp(A)}(\Omega_{2})\to\Ocal_{\Sp(A)}(\Omega_{1}\cap\Omega_{2})\to0$$
    and by flatness of $\Ocal_{\Sp(A)}(\Omega)$ (vis. \cite[\href{https://stacks.math.columbia.edu/tag/00M5}{Tag 00M5}]{stacks-project}), we get that 
    $$0\to\Ocal_{\Sp(A)}(\Omega)\otimes_{A}M\to(\Ocal_{\Sp(A)}(\Omega_{1})\oplus\Ocal_{\Sp(A)}(\Omega_{2}))\otimes_{A}M\to\Ocal_{\Sp(A)}(\Omega_{1}\cap\Omega_{2})\otimes_{A}M\to0$$
    is exact, so by noting that $M\otimes_{A}\Ocal_{\Sp(A)}(\Omega)\cong\widetilde{M}(\Omega)$, we have that $\widetilde{M}$ is acyclic. 
\end{proof}

\section{Lecture 2 -- 2nd May 2025}\label{sec: lecture 2}
The goal of this course is to develop a theory of Habiro cohomology, a functor that associates to a smooth $\ZZ$-scheme $X$ its Habiro cohomology -- a graded module over the Habiro ring, or more generally its ``category of constructible sheaves'' which in this case we tentatively denote $\Dscr_{\Hab}(X)$ of ``variations of Habiro structure.''\footnote{The terminology here borrowed from ``variations of Hodge structure'' as studied in classical algebraic geometry.}

We begin with an exploration of what these structures are in terms of coordinates, and we will later show that the constructions we discuss are in fact independent of these coordinates. Let us make the notion of coordinates precise. 
\begin{definition}[Framed Algebra]\label{def: framed algebra}
    A framed algebra is a pair $(R,\square)$ where $R$ is a smooth $\ZZ$-algebra and an \'{e}tale map $\square:\spec(R)\to\A^{d}_{\ZZ}$ or $\square:\spec(R)\to\GG_{m}^{d}$. 
\end{definition}
\begin{remark}
    It is often simpler to consider the case where the coordinates are invertible, that is, the case of $\GG_{m}^{d}$. 
\end{remark}
As a first pass, let us contemplate these constructions in the case where $X$ is affine and equal to either $\A^{d}_{\ZZ}$ or $\GG_{m}^{d}$ and only later consider the generalization to the case where $X$ is \'{e}tale over one of these spaces. Moreover, under these assumptions, we need not make any completions and one can work directly over $\ZZ[q^{\pm}]$.

Recall Habiro cohomology subsumes de Rham cohomology in an appropriate sense, and takes the $q$-derivative -- the Gaussian $q$-analogue of the derivative -- as an input. These $q$-derivatives were first investigated by Jackson \cite{Jackson}.
\begin{definition}[$q$-Derivative]\label{def: q-derivative}
    Let $R$ be $\ZZ[q^{\pm}][T_{1},\dots,T_{d}]$ or $\ZZ[q^{\pm}][T_{1}^{\pm},\dots,T_{d}^{\pm}]$. The $q$-derivative $\nabla_{i}^{q}:R\to R$ for $1\leq i\leq d$ is defined by 
    $$\nabla_{i}^{q}(f(T_{1},\dots,T_{d}))=\frac{f(T_{1},\dots,qT_{i},\dots,T_{d})-f(T_{1},\dots,T_{i},\dots,T_{d})}{qT_{i}-T_{i}}.$$
\end{definition}
\begin{remark}
    More explicitly, this operation is given on monomials by 
    $$\nabla_{i}^{q}(T_{1}^{n_{1}}\dots T_{d}^{n_{d}})=[n_{i}]_{q}\cdot T_{1}^{n_{1}}\dots T_{i}^{n_{i}-1}\dots T_{d}^{n_{d}}$$
    where $[n]_{q}=\frac{1-q^{n}}{1-q}$ is the Gaussian $q$-analogue of $n$. 
\end{remark}
\begin{remark}\label{rmk: gamma i maps}
    $\nabla_{i}^{q}$ is closely related $\gamma_{i}:R\to R$ the automorphism by 
    $$T_{j}\mapsto\begin{cases}
        T_{j} & j\neq i \\
        qT_{i} & j=i
    \end{cases}$$
    allowing us to write $\nabla_{i}^{q}(f)=\frac{\gamma_{i}(f)-f}{(q-1)T_{i}}$. 
\end{remark}
The $q$-derivative does not satisfy the Leibniz rule on the nose, but does so up to a twist by the automorphism $\gamma_{i}$ of \Cref{rmk: gamma i maps}. 
\begin{lemma}\label{lem: twisted q-leibniz}
    Let $R$ be $\ZZ[q^{\pm}][T_{1},\dots,T_{d}]$ or $\ZZ[q^{\pm}][T_{1}^{\pm},\dots,T_{d}^{\pm}]$. Then for $f,g\in R$ we have equalities 
    $$\nabla_{i}^{q}(fg)=\gamma_{i}(f)\cdot\nabla_{i}^{q}(g)+g\cdot\nabla_{i}^{q}(f)=f\cdot\nabla_{i}^{q}(g)+\gamma_{i}(g)\cdot\nabla^{q}_{i}(f).$$
\end{lemma}
\begin{proof}
    We first show the second equality. We use \Cref{rmk: gamma i maps} to observe that the latter two terms are given by 
    $$\gamma_{i}(f)\cdot\frac{\gamma_{i}(g)-g}{(q-1)T_{i}}+g\cdot\frac{\gamma_{i}(f)-f}{(q-1)T_{i}}=\frac{\gamma_{i}(f)\gamma_{i}(g)-\gamma_{i}(f)g+\gamma_{i}(f)g-fg}{(q-1)T_{i}}=\frac{\gamma_{i}(f)\gamma_{i}(g)-fg}{(q-1)T_{i}}$$
    and 
    $$f\cdot\frac{\gamma_{i}(g)-g}{(q-1)T_{i}}+\gamma_{i}(g)\frac{\gamma_{i}(f)-f}{(q-1)T_{i}}=\frac{\gamma_{i}(g)f-fg+\gamma_{i}(f)\gamma_{i}(g)-\gamma_{i}(g)f}{(q-1)T_{i}}=\frac{\gamma_{i}(f)\gamma_{i}(g)-fg}{(q-1)T_{i}}$$
    respectively, which are evidently equal. 

    We now show the first equality. Note that $\gamma_{i}$ is an automorphism $R\to R$, and in particular a homomorphism so $\gamma_{i}(fg)=\gamma_{i}(f)\gamma_{i}(g)$ in which case we have 
    $$\frac{\gamma_{i}(fg)-fg}{(q-1)T_{i}}=\frac{\gamma_{i}(f)\gamma_{i}(g)-fg}{(q-1)T_{i}}$$
    whence the claim. 
\end{proof}
We can now define the $q$-de Rham complex following Aomoto \cite{Aomoto}. 
\begin{definition}[$q$-de Rham Complex of $\A^{d}_{\ZZ}$ and $\GG_{m}^{d}$]\label{def: q-dR complex}
    Let $R=\ZZ[q^{\pm}][\underline{T}]$ be $\ZZ[q^{\pm}][T_{1},\dots,T_{d}]$ or $\ZZ[q^{\pm}][T_{1}^{\pm},\dots,T_{d}^{\pm}]$. The $q$-de Rham complex $q\dash\Omega^{\bullet}_{R/\ZZ}$ of $\spec(R)$ is the complex
    \begin{equation}\label{eqn: q-dR complex of R}
    \footnotesize
    % https://q.uiver.app/#q=WzAsNyxbMCwwLCIwIl0sWzEsMCwiXFxaWltxXntcXHBtfV1bXFx1bmRlcmxpbmV7VH1dIl0sWzIsMCwiXFxaWltxXntcXHBtfV1bXFx1bmRlcmxpbmV7VH1dXntcXG9wbHVzIGR9Il0sWzMsMCwiXFxaWltxXntcXHBtfV1bXFx1bmRlcmxpbmV7VH1dXntcXG9wbHVzXFxiaW5vbXtkfXsyfX0iXSxbNCwwLCJcXGRvdHMiXSxbNSwwLCJcXFpaW3Fee1xccG19XVtcXHVuZGVybGluZXtUfV0iXSxbNiwwLCIwIl0sWzAsMV0sWzEsMl0sWzIsM10sWzMsNF0sWzQsNV0sWzUsNl1d
    \begin{tikzcd}
        0 & {\ZZ[q^{\pm}][\underline{T}]} & {\ZZ[q^{\pm}][\underline{T}]^{\oplus d}} & {\ZZ[q^{\pm}][\underline{T}]^{\oplus\binom{d}{2}}} & \dots & {\ZZ[q^{\pm}][\underline{T}]} & 0
        \arrow[from=1-1, to=1-2]
        \arrow[from=1-2, to=1-3]
        \arrow[from=1-3, to=1-4]
        \arrow[from=1-4, to=1-5]
        \arrow[from=1-5, to=1-6]
        \arrow[from=1-6, to=1-7]
    \end{tikzcd}
    \normalsize
    \end{equation}
    with differentials given by the differentials for the Koszul complex of commuting operators $\nabla_{1}^{q},\dots,\nabla_{n}^{q}$.
\end{definition}
\begin{remark}
    Recall that these are precisely the differentials for the classical de Rham complex. See \cite[\href{https://stacks.math.columbia.edu/tag/0FKF}{Tag 0FKF}]{stacks-project} for an explicit description via equations. 
\end{remark}
\begin{remark}
    Since the first differential $\ZZ[q^{\pm}][\underline{T}]\to\ZZ[q^{\pm}][\underline{T}]^{\oplus d}$ by $(\nabla_{1}^{q},\dots,\nabla_{d}^{q})$ does not satisfy the ordinary Leibniz rule, the complex (\ref{eqn: q-dR complex of R}) is not a differential graded algebra. Later, we will see that working in the derived ($\infty$-)category, one can endow this with the structure of a commutative ring.  
\end{remark}
The complex (\ref{eqn: q-dR complex of R}) computes $q$-de Rham cohomology, or Aomoto-Jackson cohomology of $\spec(R)$. But to compute Habiro cohomology, we use a closely related variant based on a modified $q$-derivative. 
\begin{definition}[Modified $q$-Derivative]\label{def: modified q-derivative}
    Let $R$ be $\ZZ[q^{\pm}][T_{1},\dots,T_{d}]$ or $\ZZ[q^{\pm}][T_{1}^{\pm},\dots,T_{d}^{\pm}]$. The modified $q$-derivative is given by 
    $$\widetilde{\nabla}_{i}^{q}(f(T_{1},\dots,T_{d}))=\frac{f(T_{1},\dots,qT_{i},\dots,T_{d})-f(T_{1},\dots,T_{i},\dots,T_{d})}{T_{i}}.$$
\end{definition}
\begin{remark}
    In other words, $\widetilde{\nabla}_{i}^{q}(f)=(q-1)\nabla_{i}^{q}(f)=\frac{\gamma_{i}(f)-f}{T_{i}}$.  
\end{remark}
Recomputing everything using this modified derivative gives the $q$-Hodge complex. 
\begin{definition}[$q$-Hodge Complex of $\A^{d}_{\ZZ}$ and $\GG_{m}^{d}$]\label{def: q-Hodge complex}
    Let $R=\ZZ[q^{\pm}][\underline{T}]$ be $\ZZ[q^{\pm}][T_{1},\dots,T_{d}]$ or $\ZZ[q^{\pm}][T_{1}^{\pm},\dots,T_{d}^{\pm}]$. The $q$-Hodge complex $q\dash\Hdg_{R}$ of $\spec(R)$ is the complex
    \begin{equation}\label{eqn: q-Hodge complex of R}
    \footnotesize
    % https://q.uiver.app/#q=WzAsNyxbMCwwLCIwIl0sWzEsMCwiXFxaWltxXntcXHBtfV1bXFx1bmRlcmxpbmV7VH1dIl0sWzIsMCwiXFxaWltxXntcXHBtfV1bXFx1bmRlcmxpbmV7VH1dXntcXG9wbHVzIGR9Il0sWzMsMCwiXFxaWltxXntcXHBtfV1bXFx1bmRlcmxpbmV7VH1dXntcXG9wbHVzXFxiaW5vbXtkfXsyfX0iXSxbNCwwLCJcXGRvdHMiXSxbNSwwLCJcXFpaW3Fee1xccG19XVtcXHVuZGVybGluZXtUfV0iXSxbNiwwLCIwIl0sWzAsMV0sWzEsMl0sWzIsM10sWzMsNF0sWzQsNV0sWzUsNl1d
    \begin{tikzcd}
        0 & {\ZZ[q^{\pm}][\underline{T}]} & {\ZZ[q^{\pm}][\underline{T}]^{\oplus d}} & {\ZZ[q^{\pm}][\underline{T}]^{\oplus\binom{d}{2}}} & \dots & {\ZZ[q^{\pm}][\underline{T}]} & 0
        \arrow[from=1-1, to=1-2]
        \arrow[from=1-2, to=1-3]
        \arrow[from=1-3, to=1-4]
        \arrow[from=1-4, to=1-5]
        \arrow[from=1-5, to=1-6]
        \arrow[from=1-6, to=1-7]
    \end{tikzcd}
    \normalsize
    \end{equation}
    with differentials given by the differentials for the Koszul complex of commuting operators $\widetilde{\nabla}_{1}^{1},\dots,\widetilde{\nabla}_{d}^{q}$. 
\end{definition}
\begin{remark}
    The nomenclature of \Cref{def: q-dR complex,def: q-Hodge complex} are justified by the fact that they recover the ordinary de Rham and Hodge complexes, respectively, at $q=1$.
\end{remark} 
\begin{remark}
    An automorphism of $\A^{d}_{\ZZ}$ or $\GG_{m}^{d}$ would give rise to an automorphism of the complexes (\ref{eqn: q-dR complex of R}) and (\ref{eqn: q-Hodge complex of R}), at least as an object in the derived category, but it is extremely difficult to understand these automorphisms from this explicit perspective. 
\end{remark}\marginpar{The instructor remarks that he does not believe in non-flat connections. We will henceforth omit the adjective ``flat.'' \\\\ Note that a $q$-connection is additional data on a module.}
In parallel to the correspondence between algebraic $D$-modules and modules with flat connection, one would expect the existence of a category of modules with an approrpiate connection to play the role of $\Dscr_{\Hab}(X)$ alluded to earlier. To make this precise, we consider modules with $q$-connection. To simplify matters, we make these considerations on the Abelian and not $\infty$-categorical level. 
\begin{definition}[$q$-Connections on Modules]\label{def: q-connections on modules}
    Let $R=\ZZ[q^{\pm}][\underline{T}]$ be $\ZZ[q^{\pm}][T_{1},\dots,T_{d}]$ or $\ZZ[q^{\pm}][T_{1}^{\pm},\dots,T_{d}^{\pm}]$. A module with (flat) $q$-connection is a $\ZZ[q^{\pm}][\underline{T}]$-module with commuting $\ZZ[q^{\pm}]$-linear operations $\nabla_{i,M}^{q}:M\to M$ which satisfy the $q$-Leibniz rule 
    $$\nabla_{i,M}^{q}(fm)=\gamma_{i}(f)\cdot\nabla_{i,M}^{q}(m)+\nabla_{i}^{q}(f)\cdot m$$
    for $f\in \ZZ[q^{\pm}][\underline{T}]$ and $m\in M$.
\end{definition}
\begin{remark}
    To unwind any possible confusion between the similar-looking $\nabla_{i}^{q}:\ZZ[q^{\pm}][\underline{T}]\to\ZZ[q^{\pm}][\underline{T}],\nabla_{i,M}^{q}:M\to M$, we have 
    $$\underbrace{\underbrace{\gamma_{i}(f)}_{\in \ZZ[q^{\pm}][\underline{T}]}\cdot\underbrace{\nabla_{i,M}^{q}(m)}_{\in M}}_{\in M}+\underbrace{\underbrace{\nabla_{i}^{q}(f)}_{\in \ZZ[q^{\pm}][\underline{T}]}\cdot\underbrace{m}_{\in M}}_{\in M}$$
    so everything type-checks. 
\end{remark}
\begin{example}\label{ex: A1 with Weyl algebra}
    If $X=\A^{1}_{\ZZ}$ then recall that modules with connection are equivalent to modules over the Weyl algebra $\ZZ[q^{\pm}]\{T,\partial_{q}\}/(qT\partial_{q}-\partial_{q}T+1)$ since we have the operators $T\partial_{q},\partial_{q}T$ take $T^{n}$ to $q[n]_{q}T^{n},[n+1]_{q}T^{n}$, respectively, but $q[n]_{q}-[n+1]_{q}=q\cdot\frac{1-q^{n}}{1-q}-\frac{1-q^{n+1}}{1-q}=-1$. Passing to the associated-graded of the degree filtration, one gets commuting variables with the correct $q$-twists. 
\end{example}
Similarly, we can construct modules with a modified $q$-connection. 
\begin{definition}[Modified $q$-Connections on Modules]\label{def: modified q-connections on modules}
    Let $R=\ZZ[q^{\pm}][\underline{T}]$ be $\ZZ[q^{\pm}][T_{1},\dots,T_{d}]$ or $\ZZ[q^{\pm}][T_{1}^{\pm},\dots,T_{d}^{\pm}]$. A module with modified $q$-connection is a $\ZZ[q^{\pm}][\underline{T}]$-module with commuting $\ZZ[q^{\pm}]$-linear operations $\widetilde{\nabla}_{i,M}^{q}:M\to M$ which satisfy the $q$-Leibniz rule 
    $$\widetilde{\nabla}_{i,M}^{q}(fm)=\gamma_{i}(f)\cdot\widetilde{\nabla}_{i,M}^{q}(m)+\widetilde{\nabla}_{i}^{q}(f)\cdot m$$
    for $f\in \ZZ[q^{\pm}][\underline{T}]$ and $m\in M$.
\end{definition}
\begin{remark}
    For a more in-depth discussion of modules with $q$-connection, see Morrow-Tsuji \cite{MorrowTsuji} and Andr\'{e} \cite{Andre}.
\end{remark}
\begin{remark}\label{rmk: invertible case}
    Let $T_{i}$ be invertible. Unwinding the definition of the modified $q$-derivative, we have
    $$\widetilde{\nabla}_{i,M}^{q}(fm)=\gamma_{i}(f)\cdot\widetilde{\nabla}_{i,M}^{q}(m)+(q-1)\nabla_{i}^{q}(f)\cdot m$$
    where in particular we observe that the second summand has denominator $T_{i}$. Define a new operator 
    $$\widetilde{\widetilde{\nabla}}_{i,M}^{q}=T_{i}\cdot\widetilde{\nabla}_{i,M}^{q}$$
    which satisfies 
    \begin{align*}
        \widetilde{\widetilde{\nabla}}_{i,M}^{q}(fm) &= \gamma_{i}(f)\cdot\widetilde{\widetilde{\nabla}}_{i,M}^{q}(m)+(\gamma_{i}(f)-f)m \\
        &= \gamma_{i}(f)\left(\widetilde{\widetilde{\nabla}}_{i,M}^{q}(m)+m\right)-fm.
    \end{align*}
    In particular, 
    $$\left(\widetilde{\widetilde{\nabla}}_{i,M}^{q}+\id_{M}\right)(fm) = \gamma_{i}(f)\left(\widetilde{\widetilde{\nabla}}_{i,M}^{q}+\id_{M}\right)(m)$$
    so denoting $\gamma_{i,M}=\left(\widetilde{\widetilde{\nabla}}_{i,M}^{q}+\id_{M}\right)$, we have $\gamma_{i,M}(fm)=\gamma_{i}(f)\gamma_{i,M}(m)$ simplyfing the relation. 
\end{remark}
The preceding discussion of \Cref{rmk: invertible case} implies the following. 
\begin{corollary}\label{corr: equivalence of q-connection modules and semilinear}
    Let $R=\ZZ[q^{\pm}][T_{1}^{\pm},\dots,T_{d}^{\pm}]$. There is an equivalence of categories between $R$-modules with modified $q$-connection and $R$-modules with commuting $\gamma_{i}:R\to R$-semilinear endomorphisms $\gamma_{i,M}:M\to M$. 
\end{corollary}
Note that for $R=\ZZ[q^{\pm}][\underline{T}]$, $(-)\otimes_{R}(-)$ does not define a symmetric monoidal structure on the category of modules with $q$-connection: for $(M,\nabla_{i,M}^{q}),(N,\nabla_{i,N}^{q})$ two modules with $q$-connecition, $$(M\otimes_{R} N,\nabla_{i,M}^{q}\otimes_{R}\id_{N}+\id_{M}\otimes_{R}\nabla_{i,N}^{q})$$ is not a module with $q$-connection. One needs instead to take the twist $$(M\otimes_{R} N,\nabla_{i,M}^{q}\otimes_{R}\id_{N}+\gamma_{i,M}\otimes_{R}\nabla_{i,N}^{q}),$$
defining $\gamma_{i,M}:M\to M$ in an analogous way to \Cref{rmk: invertible case}. While \emph{a priori} appearing assymetric in $M,N$, there is in fact a canonical isomorphism between them. 
\begin{proposition}
    Let $R=\ZZ[q^{\pm}][T_{1}^{\pm},\dots,T_{d}^{\pm}]$. The category of $R$-modules with modified $q$-connection is symmetric monoidal. 
\end{proposition}
\begin{proof}[Proof Outline]
    Using the equivalence of \Cref{corr: equivalence of q-connection modules and semilinear}, the latter category of $R$-modules with $\gamma_{i}$-semilinear endomrophisms $\gamma_{i,M}$ is symmetric monoidal, hence the former can be promoted to a symmetric monoidal category. 
\end{proof}
\begin{proposition}
    Let $R=\ZZ[q^{\pm}][T_{1}^{\pm},\dots,T_{d}^{\pm}]$. There is a fully faithful embedding from $(q-1)$-torsion free $R$-modules with $q$-connection and $R$-modules with modified $q$-connection by $(M,\nabla_{i,M}^{q})\mapsto(M,\widetilde{\nabla}_{i,M}^{q})$ with essential image those that are $(q-1)$-torsion free and such that $\widetilde{\nabla}_{i,M}^{q}\equiv0\pmod{(q-1)}$. 
\end{proposition}
\begin{proof}
    This is the identity on modules and sends the $q$-connection to its $(q-1)$-multiple. $(q-1)$-torsion-freeness implies that the map is fully faithful. The essential image of the functor is necessarily $(q-1)$-torsion-free and such modules have trivial connection modulo $(q-1)$. 
\end{proof}
The discussion thus far has been done entirely in terms of coordinates. This prompts:
\begin{question}
    To what extent are the cohomologies and categories discussed thus far independent of coordinates? 
\end{question}
Let us consider the following example. 
\begin{example}\label{ex: Habiro stack on Gm}\marginpar{The instructor remarks that in the theory of analytic geometry the quotient would be the Tate elliptic curve for $d=1$. See \cite{AnalyticStacks}.}
    Let $X=\GG_{m}^{d}=\spec\left(\ZZ[q^{\pm}][T_{1}^{\pm},\dots,T_{d}^{\pm}]\right)$. The modules with modified $q$-connection are quasicoherent sheaves on $(\GG_{m}/q^{\ZZ})^{d}$ -- the $\gamma_{i}$'s act by multiplication by $q$ on the coordinates so the data of the endomorphisms $\gamma_{i,M}$ on the modules prescribe descent data to the quotient stack (ie. as an fpqc quotient).
\end{example}
Let us relate the discussion of complexes \Cref{def: q-dR complex,def: q-Hodge complex}, their cohomologies, and these categories of modules with (modified) $q$-connections. 
\begin{proposition}\label{prop: algebra structures on RHom unit Ad and Gm}
    Let $R=\ZZ[q^{\pm}][T_{1}^{\pm},\dots,T_{d}^{\pm}]$.
    \begin{enumerate}[label=(\roman*)]
        \item The $q$ de Rham complex $q\dash\Omega^{\bullet}_{R/\ZZ}$ computes $R\Hom_{q\dash\Mod_{R}}(\mathbbm{1},\mathbbm{1})$ in the derived category of modules with $q$-connection $q\dash\Mod_{R}$ on $\spec(R)$. 
        \item The $q$-Hodge complex $q\dash\Hdg_{R}$ computes $R\Hom_{q\dash\widetilde{\Mod}_{R}}(\mathbbm{1},\mathbbm{1})$ in the derived category of modules with modified $q$-connection $q\dash\widetilde{\Mod}_{R}$ on $\spec(R)$. 
    \end{enumerate}
\end{proposition}
\begin{proof}[Proof Outline of (i)]
    Using the equvialence between modules with $q$-connection and modules over the Weyl algebra, we compute a resolution of the symmetric monoidal unit $\ZZ[q^{\pm}][\underline{T}]$ in the category of modules over the Weyl algebra -- which precisely recovers the de Rham complex, whence the claim. 
\end{proof}
\begin{example}
    Consider the case of $\A^{1}_{\ZZ}$ taking $R=\ZZ[q^{\pm}][T]$. We compute $R\Hom(\ZZ[q^{\pm}][T],\ZZ[q^{\pm}][T])$ as $R\Hom(-,\ZZ[q^{\pm}][T])$ of a free resolution of $\ZZ[q^{\pm}][T]$ in the category of modules over the Weyl algebra $\ZZ[q^{\pm}]\{T,\partial_{q}\}/(qT\partial_{q}-\partial_{q}T+1)$ (vis. \Cref{ex: A1 with Weyl algebra}). This produces 
    $$0\to\ZZ[q^{\pm}][T]\xrightarrow{\nabla_{1}^{q}}\ZZ[q^{\pm}][T]\to0$$
    which is the $q$-de Rham complex (after passing back to modules with $q$-connection along the equivalence). 
\end{example}
Moreover, in the setting of higher algebra, these promote canonically to commutative algebra objects. 
\begin{corollary}
    The $q$-de Rham complex and $q$-Hodge complex have canonical structures as $\EE_{\infty}$-rings. 
\end{corollary}
\section{Lecture 3 -- 8th May 2025}\label{sec: lecture 3}
We prove the locality statement earlier alluded to. As before
$X=\Sp(A)$ for an affinoid $K$-algebra $A$.
\begin{proposition}\label{prop: coherent module is local}\marginpar{Proposition 2.4}
    Let $\Mcal$ be a sheaf of $\Ocal_X$-modules and $\Scal$ 
the sieve on $X$ generated by these $\Omega\in \mathsf{Rat}_X$ 
for which $\Mcal|_{\Omega}\cong \widetilde{M_{\Omega}}$ where $M_{\Omega}$ is a finitely generated $\Ocal_{X}(\Omega)$-module. If $\Scal$ is
a covering sieve, then $\Mcal$ is coherent. 
\end{proposition}
\begin{proof}
    By induction on Laurent order, it suffices to show that for $g\in A$ and $\Omega_{1}=R_{A}(1|g),\Omega_{2}=R_{A}(g|1)$ rational opens of $X$ on which $\Mcal|_{\Omega_{1}}=\widetilde{M_{1}},\Mcal|_{\Omega_{2}}=\widetilde{M_{2}}$
where $M_1, M_2$ are finitely generated $\Ocal_{X}(\Omega_{1}),\Ocal_{X}(\Omega_{2})$-modules respectively, that $\Mcal$ is finitely generated too. 

    By \Cref{corr: generating tuple,corr: generate restriction to Omega2} there are sections $(m_{i})_{i=1}^{n}$ generating $\Mcal$ as an $\Ocal_{X}$-module such that their restrictions to $\Omega_{1},\Omega_{2}$ generate $\Mcal|_{\Omega_{1}},\Mcal|_{\Omega_{2}}$. Consider 
    $$\Kcal=\ker\left(\Ocal_{X}^{n}\xrightarrow{(m_{i})_{i=1}^{N}}\Mcal\right).$$
    We have $\Kcal|_{\Omega_{j}}= \widetilde{K}_{j}$ where 
    $$K_{j}=\ker\left(\Ocal_{X}(\Omega_{j})^{n}\xrightarrow{(m_{i}|_{\Omega_{j}})_{i=1}^{n}} M_{j}\right).$$
    Applying the same reasoning, we have that there are $(k_{i})_{i=1}^{m}$ that generate $\Kcal$ as an $\Ocal_{X}$-module. It follows that $\Mcal$ is the cokernel of 
    $$\Ocal_{X}^{m}\xrightarrow{(k_{i})_{i=1}^{m}}\Ocal_{X}^{n}.$$
    The universal property shows that $\Mcal$ is isomorphic to the cokernel as it is an isomorphism of each $\Omega_{j}$. Since $\widetilde{(-)}$ is exact, we obtain $\Mcal=\widetilde{M}$ where $M$ is the cokernel of $A^{\oplus m}\to A^{\oplus n}$ by the $k_{i}$'s. 
\end{proof}
\begin{remark}\label{rmk: finitely generated modules over sites}
    In general if $\Rcal$ is a sheaf of rings on a site, we say an $\Rcal$-module is finitely generated if there are fintely many global sections such that $\Rcal^{n}\to\Mcal$ is an epimorphism of sheaves. We say $\Mcal$ is locally finitely generated if the objects on which $\Mcal$ is finitely generated form a covering sieve, and $\Mcal$ is coherent if it is locally finitely generated and the kernels of the local maps $\Rcal|_{X}^{n}\to\Mcal|_{X}$ are finitely generated. 
\end{remark}
\begin{remark}
    On a one point space, this is the condition of the kernel being finitely generated. That is, that the ring is a coherent ring. 
\end{remark}
\begin{example}
    Let us consider \Cref{rmk: finitely generated modules over sites} in the setting of $X=\Sp(A),\Rcal=\Ocal_{X},\Mcal=\widetilde{M}$ for $M$ a finitely generated $A$-module. In this case, the kernel of the map $A^{\oplus n}\to M$ generate the kernel sheaf $\Ocal_{\Sp(A)}^{n}\to\Mcal$ so sheaves coherent in the sense of \Cref{def: coherent sheaves} are coherent in the sense of \Cref{rmk: finitely generated modules over sites}. 

    Dually, if $\Mcal$ is coherent in the sense of \Cref{rmk: finitely generated modules over sites}, we can use locality of coherence \Cref{prop: coherent module is local} to observe that the global sections generating $\Mcal$ and the kernel sheaf $\Kcal$ give rise to $A$-modules $M,K$ such that $M$ is the cokernel of $K\to A^{n}$. 
\end{example}
This concludes our discussion of coherent sheaves.

Recall that $\Sp(\TT_{1})$ for $K$ algebraically closed has van der Put points $\xi$ given by the balls of radius $\leq R$ for $R\in|K|\subseteq\RR_{\geq0}$. Denote $\Kfrak_{\leq R}$ of all balls $K_{\leq R}(X)$ for $x\in\Sp(A)$. For a van der Put point $\xi$ of $\Sp(A)$, we can define $M_{\xi}$ to be the set of all $r\in [0,1)\cap |K^{\times}|$ for which there exists an $x\in\Kfrak_{\leq r}\cap\xi$ -- that is, $\xi$ contains a ball of radius $r$. Denote $K_{\leq R}(\xi)$ be set of rational open sets of radius at most $R$ in the van der Put point $\xi$. 
\begin{example}
    If $R$ is arbitrarily small, then $K_{\leq R}(\xi)=\{x\}$. In this case, $x\in\Omega$ if and only if $R(f_{0}|f_{1},\dots,f_{n})=\Omega\in\xi$ if and only if $\nu(f_{0})\geq\nu(f_{i})$ where $\nu(f)=|f(x)|$ for all $1\leq i\leq n$. 
\end{example}

\section{Lecture 4 -- 23rd May 2025}\label{sec: lecture 4}
Using the gluing procedure of (\ref{eqn: gluing map Frobenius}) allows us to correct for the overspecification of prescribing a local algebra $R^{(m)}$ for each positive integer $m$ in characteristic $p$ -- that is, gluing $R^{(m)},R^{(m')}$ where $m_{0}$ is coprime to $p$ and $m=m_{0}p^{a},m'=m_{0}p^{b}$ using the Frobenius. 

More generally, we can define the Habiro ring of a smooth framed $\ZZ$-algebra $(R,\square)$ by passing to the limit of the rings $R_{n}$ where there are surjective transition maps $R_{pm}\to R_{m}$ given by the (necessarily unique) lift of the isomorphism (\ref{eqn: gluing map Frobenius}) along the (necessarily unique) deformation of \'{e}tale algebras $R^{(m)}$ to $R_{m}$.  
\begin{definition}[Habiro Ring of Framed Algebra]\label{def: Habiro ring of framed algebra}
    Let $(R,\square)$ be a smooth framed $\ZZ$-algebra. The Habiro ring $\Hcal_{(R,\square)}$ is given by the limit 
    $$\Hcal_{(R,\square)}=\lim_{n\in\NN}R_{n}$$
    where $R_{n}$ is the completed root of unity algebra of \Cref{def: completed root of unity algebra}. 
\end{definition}
Analogously, we can define $q$-connections.
\begin{definition}[$q$-Connections over Habiro Ring]\label{def: q-connection over Habiro ring}
    Let $(R,\square)$ be a smooth framed $\ZZ$-algebra and $\Hcal_{(R,\square)}$ its Habiro ring. A module with modified $q$-connection over $\Hcal_{(R,\square)}$ is a $\Hcal_{(R,\square)}$-module $M$ with commuting $\gamma_{i}$-semilinear maps $\gamma_{i,M}:M\to M$. 
\end{definition}
\begin{proposition}\label{prop: explicit elements of HR}
    Let $(R,\square)$ be a smooth framed $\ZZ$-algebra. The Habiro ring $\Hcal_{(R,\square)}$ of $(R,\square)$ is given by 
    {\footnotesize
    \begin{equation}\label{eqn: Habiro ring of framed algebra}
        \Hcal_{(R,\square)}=\left\{(f_{m})_{m\geq 1}\in\prod_{m\geq 1}R^{(m)}[[q-\zeta_{m}]]:\substack{\forall m\in\NN,\text{ }\forall p\text{ prime} \\\varphi_{p}(f_{pm})=f_{m}\in (R^{(m)})_{p}^{\wedge}[[q-\zeta_{m}]]\cong (R^{(pm)})_{p}^{\wedge}[[q-\zeta_{pm}]]}\right\}
    \end{equation}
    \normalsize}where $\varphi_{p}$ lifts the Frobenius on $R^{(m)}/(p)$ by raising each variable to the $p$-th power and fixes $q$ and $\zeta_{m}$. 
\end{proposition}
\begin{remark}
    There is an obvious map from the Habiro ring of the torus \Cref{def: Habiro ring of base} $\Hcal_{\ZZ[\underline{T}^{\pm}]}\to\Hcal_{(R,\square)}$ endowing the Habiro ring of $(R,\square)$ with the structure of a(n \'{e}tale) $\Hcal_{\ZZ[\underline{T}^{\pm}]}$-algebra.  
\end{remark}
Let us consider some explicit elements of the Habiro ring. 
\begin{example}\label{ex: element of Habiro ring}\marginpar{The lecture contained a fairly substantive sketch of the proof \Cref{ex: element of Habiro ring}, which the author has defered to \Cref{appdx: explicit elements} for continuity of exposition.}
    Let $R=\ZZ[T_{1},\dots,T_{d},\frac{1}{1-T_{1}-\dots-T_{d}}]$ with framing $\square:\ZZ[T_{1},\dots,T_{d}]\to R$. The element 
    $$\sum_{k_{1},\dots,k_{d}\geq0}\left[\substack{k_{1}+\dots+k_{d} \\ k_{1}\text{ }\dots\text{ }k_{d}}\right]_{q}T_{1}^{k_{1}}\dots T_{d}^{k_{d}}\in\ZZ[q][[\underline{T}]]$$
    is an element of the Habiro ring $\Hcal_{(R,\square)}$ where 
    $$\left[\substack{k_{1}+\dots+k_{d} \\ k_{1}\text{ }\dots\text{ }k_{d}}\right]_{q}=\frac{(q;q)_{k_{1}+\dots+k_{d}}}{(q;q)_{k_{1}}\dots(q;q)_{k_{d}}}$$
    is the $q$-deformation of the multinomial $\binom{k_{1}+\dots+k_{d}}{k_{1}\dots k_{d}}$. More generally, explicit elements of the Habiro ring can be constructed by considering $q$-deformations of rational functions (vis. \Cref{ex: legendre family} and surrounding discussion). 
\end{example}
Returning to a discussion of Habiro cohomology of a smooth $\ZZ$-algebra with framing $\square:\spec(R)\to(\GG_{m})^{d}$, we recall that there are lifts of the automorphism $\gamma_{i}$ to $\Hcal_{(R,\square)}$: more explicitly, for a section $(f_{m})_{m\geq0}$, the action $\gamma_{i}$ acts by $(f_{m})_{m\geq1}\mapsto (\gamma_{i}^{(m)}(f_{m}))_{m\geq1}$ where $\gamma_{i}^{(m)}$ is the automorphism given in \Cref{def: root of unity algebra}. This produces a $\ZZ^{d}$-action on $\Hcal_{(R,\square)}$, and we can define Habiro-Hodge cohomology to be the group cohomology of the action of $\ZZ^{d}$ on $\Hcal_{(R,\square)}$. 
\begin{definition}[$q$-Habiro-Hodge Cohomology]\label{def: q-Habiro-Hodge cohomology}
    Let $(R,\square)$ be a smooth framed $\ZZ$-algebra. The $q$-Habiro-Hodge cohomology is the cohomology of the $q$-Habiro-Hodge complex $q\dash\HHdg_{(R,\square)}$ given by 
    \begin{equation}\label{eqn: q-Habiro-Hodge complex}
        \Hcal_{(R,\square)}\xrightarrow{(\gamma_{i}-1)_{i=1}^{d}}\bigoplus_{i=1}^{d}\Hcal_{(R,\square)}\xrightarrow{(\gamma_{i}-1)_{i=1}^{d}}\bigoplus_{i<j}\Hcal_{(R,\square)}\longrightarrow\dots.
    \end{equation}
\end{definition}
For this to be functorial, we would expect this to be coordinate independent, at least at the level of derived categories. As a first step, we study the cohomology of the complex modulo $(1-q^{m})$ -- that is, at specalizations to roots of unity. 

If $m=1$, then $\Hcal_{(R,\square)}/(1-q)\cong R$ and all differentials are zero, so 
\begin{equation}\label{eqn: de Rham complex at trivial root of unity}
    H^{i}\left(q\dash\HHdg_{(R,\square)}/(1-q)\right)\cong R^{\oplus\binom{d}{i}}\cong \Omega^{i}_{R/\ZZ}
\end{equation}
and is therefore independent of coordinates since the middle term is so. 
\begin{remark}\label{rmk: Bockstein operator}
    While \emph{a priori} we only have a isomorphism to a free module of a certain rank, there is additonal structure that allows us to identify this with the module of K\"{a}hler differentials: the Bockstein map associated to the triangle 
    {\footnotesize 
    $$q\dash\HHdg_{(R,\square)}/(1-q)\xrightarrow{\times(1-q)}q\dash\HHdg_{(R,\square)}/(1-q)^{2}\longrightarrow q\dash\HHdg_{(R,\square)}/(1-q)\longrightarrow \left(q\dash\HHdg_{(R,\square)}/(1-q)\right)[1]$$
    \normalsize}inducing 
    $$H^{i}\left(q\dash\HHdg_{(R,\square)}/(1-q)\right)\longrightarrow H^{i+1}\left(q\dash\HHdg_{(R,\square)}/(1-q)\right)$$
    which gives a derivation 
    $$H^{0}\left(q\dash\HHdg_{(R,\square)}/(1-q)\right)\longrightarrow H^{1}\left(q\dash\HHdg_{(R,\square)}/(1-q)\right)$$
    and hence an isomorphism $H^{1}\left(q\dash\HHdg_{(R,\square)}/(1-q)\right)\to\Omega^{1}_{R/\ZZ}$. In addition, the ring structure on cohomology induces the structure of a commutative differential graded algebra on $H^{\bullet}\left(q\dash\HHdg_{(R,\square)}/(1-q)\right)$ and this structure is in fact independent of coordinates on the nose and not just up to quasi-isomorphism.
\end{remark}

For general $m$, $H^{\bullet}\left(q\dash\HHdg_{(R,\square)}/(1-q^{m})\right)$ has the strucuture of a commutative differential graded algebra that is coordinate independent. 
\begin{theorem}[Wagner; {\cite[Prop. 5.7]{WagnerMSThesis}}]\label{thm: surjection from q witt vectors}
    Let $R$ be a smooth framed $\ZZ$-algebra. There is a canonical surjection 
    $$W_{m}(R)[q]/(1-q^{m})^{\bullet}\longrightarrow H^{0}\left(q\dash\HHdg_{(R,\square)}/(1-q^{m})\right)$$
    inducing 
    $$\Omega_{W_{m}(R)[q]/(1-q^{m})}\longrightarrow H^{\bullet}\left(q\dash\HHdg_{(R,\square)}/(1-q)\right)$$ 
    which is coordinate independent, degreewise surjective, and with kernel independent of coordinates. 
\end{theorem}
\begin{proof}[Proof Outline]
    For every commutative differential graded algebra $B$ receiving a map from a commutative ring $A$ in 0th cohomology, there is an induced map from the initial commutative differential graded algebra generated by $A$ to $B$ -- the latter being the de Rham complex. 
\end{proof}
This produces a description of $H^{i}\left(q\dash\HHdg_{(R,\square)}/(1-q)\right)$ that is visibly independent of coordinates, being the quotient of coordinate-independent objects. 

In fact we can do better. For any $R$, there is a notion of $q$-Witt vectors $q\dash W_{m}(R)$ and $q$-de Rham-Witt complexes $q\dash W_{m}\Omega_{R}$ which is a commutative differential graded algebra with first term given by the $q$-Witt vectors $q\dash W_{m}(R)$ isomorphic to the complex $H^{\bullet}\left(q\dash\HHdg_{(R,\square)}/(1-q^{m})\right)$ as a commutative differential graded algebra. 
\begin{theorem}[Wagner; {\cite[Thm. 5.7]{WagnerMSThesis}}]\label{thm: }
    Let $R$ be a smooth framed $\ZZ$-algebra. There is an isomorphism 
    $$q\dash W_{m}\Omega_{R}^{\bullet}\longrightarrow H^{\bullet}\left(q\dash\HHdg_{(R,\square)}/(1-q^{m})\right)$$
    where $q\dash W_{m}\Omega^{\bullet}_{R}$ is the $q$-de Rham-Witt complex.
\end{theorem}
\begin{remark}
    This is related to the classical construction of the de Rham-Witt complex, though the sense in which the preceding constructions are $q$-deformations are quite subtle. 
\end{remark}
\begin{remark}
    One can often reduce to the case of computing on the torus, since many of the constructions ``commute with \'{e}tale maps'' in the sense that they are preserved under \'{e}tale base change. 
\end{remark}
Based on this, one might hope that these complexes are independent of coordinates. 
\begin{example}\label{ex: q-Habiro-Hodge cohomology of torus}
    Let $R=\ZZ[T^{\pm}]$. The $q$-Habiro-Hodge complex is given by 
    $$\ZZ[q][T^{\pm}]/(1-q^{m})\xrightarrow{\gamma-1}\ZZ[q][T^{\pm}]/(1-q^{m})$$
    by $T^{k}\mapsto(q^{k}-1)T^{k}$. We can compute the kernel of this map -- the 0th cohomology -- by noting that the map preserves the degree of $T$, we can compute the kernel in each degree to see that it is given by 
    $$\bigoplus_{k\in\ZZ}\left(\frac{\frac{q^{m}-1}{q^{\gcd(k,m)}-1}\ZZ[q]}{(q^{m}-1)\ZZ[q]}\right)T^{k}\cong\bigoplus_{k\in\ZZ}\left(\ZZ[q]/(1-q^{\gcd(k,m)})\ZZ[q]\right)T^{k}.$$
    We similarly compute first cohomology to see it is also given by 
    $$\bigoplus_{k\in\ZZ}\left(\ZZ[q]/(1-q^{\gcd(k,m)})\ZZ[q]\right)T^{k}.$$
    Indeed, when $m=p$ is prime, the 0th cohomology is a subring of $\ZZ[q][T^{\pm}]/(1-q^{p})$ (hence a subring of $\ZZ[T^{\pm}]\times\ZZ[\zeta_{p}][T^{\pm p}]\subseteq\ZZ[T^{\pm}]\times\ZZ[\zeta_{p}][T^{\pm}]$) and is generated by $T^{p}$ and $[p]_{q}T^{i}$ for $1\leq i\leq p-1$. 
\end{example}
The computations of \Cref{ex: q-Habiro-Hodge cohomology of torus} is suggestive of a connection to Witt vectors since the cohomology lies in the product of rings $\ZZ[T^{\pm}]\times\ZZ[\zeta_{p}][T^{\pm p}]$. Recall that for a $p$-torsion free ring $R$, the $p$-th Witt vectors $W_{p}(R)$ consists of elements $(x_{0},x_{1},\dots)$ has ghost maps $\gh_{1},\gh_{p}:W_{p}(R)\to R$ by $(x_{0},x_{1},\dots)\mapsto x_{0}$ and $(x_{0},x_{1},\dots)\mapsto x_{0}^{p}+px_{1}$, respectively. The image of $(\gh_{1},\gh_{p}):W_{p}(R)\to R\times R$ consists precisely of those pairs $(x,y)\in R\times R$ where $y\equiv x^{p}\pmod{p}$. 
\begin{proposition}[Wagner]\label{prop: q-Witt vectors}
    Let $R=\ZZ[T^{\pm}]$ with the identity framing and $q\dash\HHdg_{(R,\square)}$ its $q$-Habiro-Hodge complex. There is a canonical embedding 
    $$W_{p}(R)\hookrightarrow H^{0}\left(q\dash\HHdg_{(R,\square)}/(1-q^{p})\right)$$
    rendering the diagram 
    {\footnotesize
    $$% https://q.uiver.app/#q=WzAsNyxbMSwxLCJIXnswfVxcbGVmdChxXFxkYXNoXFxISGRnX3soUixcXHNxdWFyZSl9LygxLXFee3B9KVxccmlnaHQpIl0sWzEsMiwiV197cH0oUikiXSxbMywxLCJcXFpaW1Ree1xccG19XVxcdGltZXNcXFpaW1xcemV0YV97cH1dW1Ree1xccG0gcH1dIl0sWzMsMiwiUlxcdGltZXMgUiJdLFswLDMsIih4X3swfSx4X3sxfSkiXSxbNCwzLCIoeF97MH0seF97MH1ee3B9K3B4X3sxfSkiXSxbMCwwLCJcXHZhcnBoaV97cH0oeF97MH0pK1twXV97cX14X3sxfSJdLFsxLDAsIiIsMCx7InN0eWxlIjp7InRhaWwiOnsibmFtZSI6Imhvb2siLCJzaWRlIjoidG9wIn19fV0sWzEsMywiKFxcZ2hfezF9LFxcZ2hfe3B9KSIsMl0sWzMsMl0sWzAsMiwiIiwwLHsic3R5bGUiOnsidGFpbCI6eyJuYW1lIjoiaG9vayIsInNpZGUiOiJ0b3AifX19XSxbNCw1LCIiLDIseyJzdHlsZSI6eyJ0YWlsIjp7Im5hbWUiOiJtYXBzIHRvIn19fV0sWzQsNiwiIiwwLHsic3R5bGUiOnsidGFpbCI6eyJuYW1lIjoibWFwcyB0byJ9fX1dXQ==
    \begin{tikzcd}
        {\varphi_{p}(x_{0})+[p]_{q}x_{1}} \\
        & {H^{0}\left(q\dash\HHdg_{(R,\square)}/(1-q^{p})\right)} && {\ZZ[T^{\pm}]\times\ZZ[\zeta_{p}][T^{\pm p}]} \\
        & {W_{p}(R)} && {R\times R} \\
        {(x_{0},x_{1})} &&&& {(x_{0},x_{0}^{p}+px_{1})}
        \arrow[hook, from=2-2, to=2-4]
        \arrow[hook, from=3-2, to=2-2]
        \arrow["{(\gh_{1},\gh_{p})}"', from=3-2, to=3-4]
        \arrow[from=3-4, to=2-4]
        \arrow[maps to, from=4-1, to=1-1]
        \arrow[maps to, from=4-1, to=4-5]
    \end{tikzcd}$$
    \normalsize}commutative. 
\end{proposition}
\begin{remark}
    On the $q$-Habiro-Hodge cohomologies, we can relate the different specializations by Frobenii and Verschiebungen 
    $$% https://q.uiver.app/#q=WzAsMixbMCwwLCJIXntpfVxcbGVmdChxXFxkYXNoXFxISGRnX3soUixcXHNxdWFyZSl9LygxLXFee21rfSlcXHJpZ2h0KSJdLFsyLDAsIkhee2l9XFxsZWZ0KHFcXGRhc2hcXEhIZGdfeyhSLFxcc3F1YXJlKX0vKDEtcV57bX0pXFxyaWdodCkuIl0sWzAsMSwiRl97a30iLDAseyJvZmZzZXQiOi0xfV0sWzEsMCwiVl97a309XFx0aW1lc1xcZnJhY3sxLXFee21rfX17MS1xXnttfX0iLDAseyJvZmZzZXQiOi0xfV1d
    \begin{tikzcd}
        {H^{i}\left(q\dash\HHdg_{(R,\square)}/(1-q^{mk})\right)} && {H^{i}\left(q\dash\HHdg_{(R,\square)}/(1-q^{m})\right).}
        \arrow["{F_{k}}", shift left, from=1-1, to=1-3]
        \arrow["{V_{k}=\times\frac{1-q^{mk}}{1-q^{m}}}", shift left, from=1-3, to=1-1]
    \end{tikzcd}$$
\end{remark}
More generally, we have the following. 
\begin{proposition}
    Let $R$ be a flat $\ZZ$-algebra. There is a commutative diagram 
    $$% https://q.uiver.app/#q=WzAsNSxbMiwwLCJXX3ttfShSKSJdLFs0LDAsIlxccHJvZF97ZHxtfVIiXSxbNCwxLCJcXHByb2Rfe2R8bX1SW1xcemV0YV97ZH1dIl0sWzIsMSwicVxcZGFzaCBXX3ttfShSKSJdLFswLDEsIldfe219KFIpW3FdLygxLXFee219KSJdLFs0LDMsIiIsMCx7InN0eWxlIjp7ImhlYWQiOnsibmFtZSI6ImVwaSJ9fX1dLFswLDRdLFswLDMsIiIsMix7InN0eWxlIjp7InRhaWwiOnsibmFtZSI6Imhvb2siLCJzaWRlIjoidG9wIn19fV0sWzAsMSwiKFxcZ2hfe2R9KV97ZHxtfSIsMCx7InN0eWxlIjp7InRhaWwiOnsibmFtZSI6Imhvb2siLCJzaWRlIjoidG9wIn19fV0sWzMsMiwiKHFcXGRhc2hcXGdoX3tkfSlfe2R8bX0iLDIseyJzdHlsZSI6eyJ0YWlsIjp7Im5hbWUiOiJob29rIiwic2lkZSI6InRvcCJ9fX1dLFsxLDIsIihSKV97ZH1cXHRvKFJbXFx6ZXRhX3ttL2R9XSlfe20vZH0iLDAseyJzdHlsZSI6eyJ0YWlsIjp7Im5hbWUiOiJob29rIiwic2lkZSI6InRvcCJ9fX1dXQ==
    \begin{tikzcd}
        && {W_{m}(R)} && {\prod_{d|m}R} \\
        {W_{m}(R)[q]/(1-q^{m})} && {q\dash W_{m}(R)} && {\prod_{d|m}R[\zeta_{d}]}
        \arrow["{(\gh_{d})_{d|m}}", hook, from=1-3, to=1-5]
        \arrow[from=1-3, to=2-1]
        \arrow[hook, from=1-3, to=2-3]
        \arrow["{(R)_{d}\to(R[\zeta_{m/d}])_{m/d}}", hook, from=1-5, to=2-5]
        \arrow[two heads, from=2-1, to=2-3]
        \arrow["{(q\dash\gh_{d})_{d|m}}"', hook, from=2-3, to=2-5]
    \end{tikzcd}$$
    where the Frobenii and Verschiebungen are defined on $q\dash W_{m}(R)$. 
\end{proposition}
\begin{remark}
    On the right, the map takes the $d$th factor of the product $\prod_{d|m}R$ to the $\frac{m}{d}$th factor of the product $\prod_{d|m}R[\zeta_{d}]$. 
\end{remark}
\begin{remark}
    There are no restriction maps on the $q$-Witt vectors $q\dash W_{m}(R)$. 
\end{remark}
This shows that on the level of cohomology, the $q$-Habiro-Hodge complex is coordinate independent after specialization. However, due to a theorem of Wagner, this is the best we can do: there is no way to make the $q$-Habiro-Hodge complex itself coordinate independent in the derived category in such a way that remains coordinate independent on specialization. 

\section{Lecture 5 -- 30th May 2025}\label{sec: lecture 5}
Recall from \Cref{def: q-Habiro-Hodge cohomology} that the $q$-Habiro-Hodge cohomology is defined to be the cohomology of the $q$-Habiro-Hodge complex $q\dash\HHdg_{(R,\square)}$ of (\ref{eqn: q-Habiro-Hodge complex}), or equivalently, the group cohomology of $\ZZ^{d}$ on the $\ZZ[\ZZ^{d}]$-module $\Hcal_{(R,\square)}$. More explicitly, for each $m\geq 1$ we have the commutative differential graded algebra
\begin{equation}\label{eqn: q-Habiro-Hodge CDGA}
    \left(H^{\bullet}\left(q\dash\HHdg_{(R,\square)}/(1-q^{m})\right),\times(1-q^{m})\right)
\end{equation}
with the Bockstein operator of multiplication by $(1-q^{m})$ as in \Cref{rmk: Bockstein operator}. 

Let us consider the case of rational coefficients as an example. 
\begin{example}
    Note that $\QQ[q^{\pm}]/(1-q^{m})\cong\prod_{d|m}\QQ(\zeta_{d})$ by $q\mapsto(\zeta_{d})_{d|m}$. After base change to $\QQ$, we the commutative differential graded algebra of (\ref{eqn: q-Habiro-Hodge CDGA}) splits as a product of commutative differential graded algebras. A factor of this product is 
    $$H^{\bullet}\left(q\dash\HHdg_{(R,\square)}\otimes_{\ZZ[q^{\pm}]}\QQ(\zeta_{d})\right)$$
    where by construction we have that 
    \begin{equation}\label{eqn: rationalized q-Habiro-Hodge is rationalized R}
        q\dash\HHdg_{(R,\square)}\otimes_{\ZZ[q^{\pm}]}\QQ(\zeta_{d})\cong R\otimes_{\ZZ[\underline{T}^{\pm}]}\QQ(\zeta_{d})[\underline{T}^{\pm}]
    \end{equation}
    where the $\ZZ[\underline{T}^{\pm}]$-algebra structure on $\QQ(\zeta_{d})[\underline{T}^{\pm}]$ is by $T_{i}\mapsto T_{i}^{d}$ and the operators are given by $\id_{R}\otimes[T_{i}\mapsto \zeta_{d}T_{i}]$ (cf. \Cref{def: root of unity algebra}). Recalling that 0th group cohomology recovers invariants, we observe that the invariants under this action consists of consists of polynomials with $d$th roots, and the action does not extract additional roots of the coordinates giving a canonical isomorphism 
    $$H^{0}\left(q\dash\HHdg_{(R,\square)}\otimes_{\ZZ[q^{\pm}]}\QQ(\zeta_{d})\right)\cong R\otimes_{\ZZ}\QQ(\zeta_{d}).$$
    Having explicitly determined $H^{0}$, the universal property (cf. Proof of \Cref{thm: surjection from q witt vectors}) implies that there is a morphism of commutative differential graded algebras 
    $$\Omega^{\bullet}_{R\otimes_{\ZZ}\QQ(\zeta_{d})/\ZZ}\longrightarrow H^{\bullet}\left(q\dash\HHdg_{(R,\square)}\otimes_{\ZZ[q^{\pm}]}\QQ(\zeta_{d})\right).$$
\end{example}
This map can be shown to be an isomorphism. 
\begin{proposition}\label{prop: rationalized Habiro cohomology}
    Let $(R,\square)$ be a smooth framed $\ZZ$-algebra and fix $m\geq0$. For each $d|m$, there is an isomorphism of commutative differential graded algebras 
    $$\Omega^{\bullet}_{R\otimes_{\ZZ}\QQ(\zeta_{d})/\ZZ}\xrightarrow{\sim} H^{\bullet}\left(q\dash\HHdg_{(R,\square)}\otimes_{\ZZ[q^{\pm}]}\QQ(\zeta_{d})\right).$$
\end{proposition}
\begin{proof}[Proof Outline]
    Using the identification of (\ref{eqn: rationalized q-Habiro-Hodge is rationalized R}) above, we note that the action by the commuting operators is trivial. So both modules are in each degree free of the same rank and are isomorphic in degree 0. This produces morphisms in all higher degrees, which can be shown to be isomorphisms. 
\end{proof}
In other words, rationally, $q$-Habiro-Hodge cohomology at a fixed root of unity $m$ is entirely determined by the algebraic de Rham cohomology at each of its factors. 
\begin{proposition}\label{prop: Habiro cohomology is ZZ torsion free}
    Let $(R,\square)$ be a smooth framed $\ZZ$-algebra and fix $m\geq0$. Each $H^{i}(q\dash\HHdg_{(R,\square)}/(1-q^{m}))$ is $\ZZ$-torsion free and there exists an injection 
    \begin{equation}\label{eqn: ZZ-torsion-free injective}
    % https://q.uiver.app/#q=WzAsMixbMCwwLCJIXntpfVxcbGVmdChxXFxkYXNoXFxISGRnX3soUixcXHNxdWFyZSl9LygxLXFee219KVxccmlnaHQpIl0sWzIsMCwiXFxwcm9kX3tkfG19XFxPbWVnYV57aX1fe1JcXG90aW1lc197XFxaWn1cXFFRKFxcemV0YV97ZH0pL1xcWlp9LiJdLFswLDEsIi1cXG90aW1lc197XFxaWn1cXFFRIiwwLHsic3R5bGUiOnsidGFpbCI6eyJuYW1lIjoiaG9vayIsInNpZGUiOiJ0b3AifX19XV0=
        \begin{tikzcd}
            {H^{i}\left(q\dash\HHdg_{(R,\square)}/(1-q^{m})\right)} && {\prod_{d|m}\Omega^{i}_{R\otimes_{\ZZ}\QQ(\zeta_{d})/\ZZ}.}
            \arrow["{-\otimes_{\ZZ}\QQ}", hook, from=1-1, to=1-3]
        \end{tikzcd}
    \end{equation}
\end{proposition}
The proof is fairly elementary, albeit computational, and hence omitted. Moreover, the target is manifestly canonically independent of coordinates, and we can show that $H^{i}(q\dash\HHdg_{(R,\square)}/(1-q^{m}))$ is independent of coordinates by showing its image is so. 
\begin{theorem}[Wagner]\label{thm: image after rationalization is independent}
    Let $(R,\square)$ be a smooth framed $\ZZ$-algebra and fix $m\geq0$. The image of the map (\ref{eqn: ZZ-torsion-free injective})
    $$% https://q.uiver.app/#q=WzAsMixbMCwwLCJIXntpfVxcbGVmdChxXFxkYXNoXFxISGRnX3soUixcXHNxdWFyZSl9LygxLXFee219KVxccmlnaHQpIl0sWzIsMCwiXFxwcm9kX3tkfG19XFxPbWVnYV57aX1fe1JcXG90aW1lc197XFxaWn1cXFFRKFxcemV0YV97ZH0pL1xcWlp9LiJdLFswLDEsIi1cXG90aW1lc197XFxaWn1cXFFRIiwwLHsic3R5bGUiOnsidGFpbCI6eyJuYW1lIjoiaG9vayIsInNpZGUiOiJ0b3AifX19XV0=
    \begin{tikzcd}
        {H^{i}\left(q\dash\HHdg_{(R,\square)}/(1-q^{m})\right)} && {\prod_{d|m}\Omega^{i}_{R\otimes_{\ZZ}\QQ(\zeta_{d})/\ZZ}.}
        \arrow["{-\otimes_{\ZZ}\QQ}", hook, from=1-1, to=1-3]
    \end{tikzcd}$$
    is given by the degree $i$ piece of the $q$-de Rham-Witt complex $q\dash W_{m}\Omega_{R/\ZZ}^{i}$ and hence independent of coordinates. 
\end{theorem}
\begin{proof}[Proof Outline]
    Recall that the $q$-de Rham-Witt complex $q\dash W_{m}\Omega_{R/\ZZ}^{\bullet}$ of \Cref{prop: q-Witt vectors} is a commutative differential graded algebra whose degree zero piece is the $q$-Witt vectors $q\dash W_{m}(R)$. By a universal property argument, there is a surjection 
    $$% https://q.uiver.app/#q=WzAsMixbMCwwLCJcXE9tZWdhX3txXFxkYXNoIFdfe219KFIpL1xcWlp9XntcXGJ1bGxldH0iXSxbMiwwLCJIXntcXGJ1bGxldH1cXGxlZnQocVxcZGFzaFxcSEhkZ197KFIsXFxzcXVhcmUpfS8oMS1xXnttfSlcXHJpZ2h0KSJdLFswLDEsIiIsMCx7InN0eWxlIjp7ImhlYWQiOnsibmFtZSI6ImVwaSJ9fX1dXQ==
    \begin{tikzcd}
        {\Omega_{q\dash W_{m}(R)/\ZZ}^{\bullet}} && {H^{\bullet}\left(q\dash\HHdg_{(R,\square)}/(1-q^{m})\right)}
        \arrow[two heads, from=1-1, to=1-3]
    \end{tikzcd}$$
    which factors over the $q$-de Rham-Witt complex. By explicitly identifying the relations of the surjection $\Omega^{\bullet}_{q\dash W_{m}(R)/\ZZ}\twoheadrightarrow q\dash W_{m}\Omega^{\bullet}_{R/\ZZ}$, the relations on $\Omega^{\bullet}_{q\dash W_{m}(R)/\ZZ}$ can be seen to coincide with the relations of the image of (\ref{eqn: ZZ-torsion-free injective}) in $\prod_{d|m}\Omega^{i}_{R\otimes_{\ZZ}\QQ(\zeta_{d})/\ZZ}$ producing the desired isomorphism
    $$% https://q.uiver.app/#q=WzAsMyxbMCwwLCJcXE9tZWdhX3txXFxkYXNoIFdfe219KFIpL1xcWlp9XntcXGJ1bGxldH0iXSxbMiwwLCJIXntcXGJ1bGxldH1cXGxlZnQocVxcZGFzaFxcSEhkZ197KFIsXFxzcXVhcmUpfS8oMS1xXnttfSlcXHJpZ2h0KSJdLFsxLDEsInFcXGRhc2ggV197bX1cXE9tZWdhXntcXGJ1bGxldH1fe1IvXFxaWn0iXSxbMCwxLCIiLDAseyJzdHlsZSI6eyJoZWFkIjp7Im5hbWUiOiJlcGkifX19XSxbMCwyLCIiLDIseyJzdHlsZSI6eyJoZWFkIjp7Im5hbWUiOiJlcGkifX19XSxbMiwxLCJcXHNpbSIsMl1d
    \begin{tikzcd}
        {\Omega_{q\dash W_{m}(R)/\ZZ}^{\bullet}} && {H^{\bullet}\left(q\dash\HHdg_{(R,\square)}/(1-q^{m})\right)} \\
        & {q\dash W_{m}\Omega^{\bullet}_{R/\ZZ}}
        \arrow[two heads, from=1-1, to=1-3]
        \arrow[two heads, from=1-1, to=2-2]
        \arrow["\sim"', from=2-2, to=1-3]
    \end{tikzcd}$$
\end{proof}
\begin{remark}
    Note that the $q$-de Rham-Witt complex $q\dash W_{m}\Omega^{\bullet}_{R}$ is distinct from the intial free commutative differential graded algebra over the $q$-Witt vectors $q\dash W_{m}(R)$, the (ordinary) de Rham complex of the $q$-Witt vectors $\Omega^{\bullet}_{q\dash W_{m}(R)}$.
\end{remark}
\begin{remark}
    The $q$-Witt vectors are not $q$-analogues of the Witt vectors, in the sense that specialization at $q=1$ does not recover the ordinary construction. Regardless, these are closely related constructions as exhibited in \Cref{prop: q-Witt vectors}. 
\end{remark}
While the preceding constructions show the richness of specializations of $q$-Habiro-Hodge cohomology at roots of unity, we show that this construction does not globalize. In particular, we show (an variant of) Wagner's theorem \cite[Thm. 5.1]{WagnerQWittQHodge}: a no-go result showing that the framing is a necessary part of the definition of the $q$-Habiro-Hodge complex. 
\begin{theorem}[Wagner; {\cite[Thm. 5.1]{WagnerQWittQHodge}}]\label{thm: Wagner no-go}
    There is no functor from smooth $\ZZ$-algebras to the commutative algebra objects of the derived $\infty$-category $\Dscr(\ZZ[q^{\pm}])$
    $$% https://q.uiver.app/#q=WzAsNSxbMCwwLCJcXG1hdGhzZntBbGd9XntcXG1hdGhzZntzbX19X3tcXFpafSJdLFsyLDAsIlxcbWF0aHNme0NBbGd9XFxsZWZ0KFxcRHNjcihcXFpaW3Fee1xccG19XSlcXHJpZ2h0KSJdLFszLDAsIjsiXSxbNCwwLCJSIl0sWzYsMCwicVxcZGFzaFxcSEhkZ197Un0iXSxbMCwxXSxbMyw0LCIiLDIseyJzdHlsZSI6eyJ0YWlsIjp7Im5hbWUiOiJtYXBzIHRvIn19fV1d
    \begin{tikzcd}
        {\mathsf{Alg}^{\mathsf{sm}}_{\ZZ}} && {\CAlg\left(\Dscr(\ZZ[q^{\pm}])\right)} & {;} & R && {q\dash\HHdg_{R}}
        \arrow[from=1-1, to=1-3]
        \arrow[maps to, from=1-5, to=1-7]
    \end{tikzcd}$$
    such that the following hold:
    \begin{enumerate}[label=(\roman*)]
        \item For all \'{e}tale framings $\square:\ZZ[\underline{T}^{\pm}]\to R$ there is an isomorphism of $\Dscr(\ZZ[q^{\pm}])$ commutative algebras $q\dash\HHdg_{R}\simeq(\Hcal_{(R,\square)})^{h\ZZ^{d}}$. 
        \item The isomorphism of (i) induces an isomorphism $H^{0}\left(q\dash\HHdg_{R}\otimes_{\ZZ[q^{\pm}]}\QQ(\zeta_{d})\right)\cong R\otimes_{\ZZ}\QQ(\zeta_{d})$. 
    \end{enumerate}
\end{theorem}
\begin{remark}
    In the statement of the theorem, the commutative algebra object $q\dash\HHdg_{R}$ is not the $q$-Habiro-Hodge complex $q\dash\HHdg_{(R,\square)}$ of \Cref{def: q-Habiro-Hodge cohomology} which depends on the framing $\square:\ZZ[\underline{T}^{\pm}]\to R$. 
\end{remark}
\begin{remark}
    It is likely that the theorem holds true without condition (ii), but it may be possible to write down for any $R$ some arbitrarily complicated framing $\square:\ZZ[\underline{T}^{\pm}]\to R$ and some arbitrarily complicated isomorphism $q\dash\HHdg_{R}\simeq(\Hcal_{(R,\square)})^{h\ZZ^{d}}$ for algebras in sufficiently many variables. Condition (ii) prescribes the additional data needed to avoid the situation of the preceding discussion by requiring that the putative object $q\dash\HHdg_{R}$ at least has degree 0 cohomology that agrees with what we have constructed thus far. 
\end{remark}
\begin{remark}
    The statement of \Cref{thm: Wagner no-go} differs from \cite[Thm. 5.1]{WagnerQWittQHodge} in several ways: Wagner works only with specializations modulo $(1-q^{m})$ in terms of $q$-de Rham Witt forms, but is also able to conduct the proof using only the module structure. 
\end{remark}
The proof of \Cref{thm: Wagner no-go} will require significant $\infty$-categorical machinery, in particular the language of animation, which we recall in \Cref{appdx: on animation}. For the proof, we will require some results about the cotangent complex. 
\begin{lemma}\label{lem: vanishing of cotangent complex of perfect Fp algebras}\marginpar{The instructor attributes this result as one of the inspiration for the tilting construction in prismatic cohomology.}
    Let $R$ be a perfect $\FF_{p}$-algebra. Then $\LL_{R/\FF_{p}}=0$. 
\end{lemma}
\begin{proof}
    Since $R$ is perfect, we compute that for any $x\in R$, 
    $$\mathrm{d}x=\mathrm{d}x^{p}=px^{p-1}\mathrm{d}x=0.$$
    Moreover, the Frobenius map gives an isomorphism $\Omega^{1}_{R/\FF_{p}}\to\Omega^{1}_{R/\FF_{p}}$ which is zero by the computation above. The animation of the Frobenius map $\LL_{R/\FF_{p}}\to\LL_{R/\FF_{p}}$ remains an isomorphism, and is zero since the animation of the zero functor is zero. In particular, the zero map gives an isomorphism $\LL_{R/\FF_{p}}\xrightarrow{\sim}\LL_{R/\FF_{p}}$, showing that $\LL_{R/\FF_{p}}=0$. 
\end{proof}
Note that this only gives vanishing of the cotangent complex over $\FF_{p}$. However, in the case of $\ZZ$-algebras, we can show the following. 
\begin{lemma}\label{lem: mod p q-Hodge is R mod p}
    Let $R$ be a flat $\ZZ$-algebra such that $R/(p)$ is a perfect $\FF_{p}$-algebra. Then for $q\dash\HHdg_{R}$ as in \Cref{thm: Wagner no-go}, $\left(q\dash\HHdg_{R}/^{L}(p,1-q)\right)\cong R/(p)[0]$. 
\end{lemma}
\begin{proof}
    Observe that $q\dash\HHdg_{R}/(1-q)$ has a canonical exhaustive filtration by $$\tau^{\leq i}\left(q\dash\HHdg_{R}/(1-q)\right)$$ with $i$th associated graded $\Omega^{i}_{R/\ZZ}[-i]$ as in (\ref{eqn: de Rham complex at trivial root of unity}). So animating the functor $R\mapsto q\dash\HHdg_{R}/(p,1-q)$ we observe that $\left(q\dash\HHdg_{R}/^{L}(p,1-q)\right)$ has an exhaustive filtration with associated gradeds $(\LL^{i}_{R/\ZZ}/^{L}(p))[-i]$, but we have $\LL^{i}_{R/\ZZ}/^{L}(p)\cong\LL^{i}_{(R/^{L}(p))/\FF_{p}}$ which vanishes for all strictly positive $i$ by the factorization of the functor $\LL^{i}_{(-/^{L}(p))/\FF_{p}}$ as 
    $$\Ani\left(\bigwedge^{i}(-)\right)\circ\Ani\left(\Omega^{1}_{(-/(p))/\FF_{p}}\right)$$
    and we just observe that in degree 0 we simply recover the quotient $R/(p)$. 
\end{proof}
\begin{corollary}\label{corr: p-completion mod 1-q}
    Let $R$ be a flat $\ZZ$-algebra such that $R/(p)$ is a perfect $\FF_{p}$-algebra. Then for $q\dash\HHdg_{R}$ as in \Cref{thm: Wagner no-go},
    $$\left(q\dash\HHdg_{R}/^{L}(1-q)\right)^{\wedge}_{p}\cong (R^{\wedge}_{p})[0].$$
    Further, $\left(q\dash\HHdg_{R}/^{L}(1-q)\right)^{\wedge}_{(p,q-1)}\cong (R[[q-1]])[0]$. 
\end{corollary}
\begin{proof}
    The $p$-adic completion $\left(q\dash\HHdg_{R}/^{L}(1-q)\right)^{\wedge}_{p}$ is the unique lift of the quotient $\left(q\dash\HHdg_{R}/^{L}(p,1-q)\right)$ to characteristic zero, and the unique lift of $R/(p)[0]$ computed in \Cref{lem: mod p q-Hodge is R mod p} is precisely $(R^{\wedge}_{p})[0]$.  
\end{proof}
\begin{proposition}
    Let $R$ be a flat $\ZZ$-algebra such that $R/(p)$ is a perfect $\FF_{p}$-algebra. If $S=R/(f)$ for some $f\in R$ a non-zerodivisor, then
    $\left(q\dash\HHdg_{S}/^{L}(1-q)\right)^{\wedge}_{(p,q-1)}$
    is concentrated in degree 0 and flat over $\ZZ_{p}[[q-1]]$.
\end{proposition}
\begin{proof}
    We have $(\LL_{S/\ZZ})_{p}^{\wedge}\simeq\LL_{S/R}\simeq S[1]$, the latter isomorphism by $R\to S$ being a regular closed immersion. Thus the $q$-Habiro-Hodge complex has after $(p,q-1)$-adic completion the desired properties by animating the $q$-Habiro-Hodge complex modulo $(q-1)$.
\end{proof}
We now proceed with the proof of \Cref{thm: Wagner no-go}. 
\begin{proof}[Proof Outline of \Cref{thm: Wagner no-go}]
    Pick $R$ admitting some \'{e}tale framing $\square:\ZZ[\underline{T}^{\pm}]\to R$ and consider $(R,\square)$ as a smooth framed $\ZZ$-algebra. By (i), there exists an isomorphism 
    \begin{equation}\label{eqn: assumption i identification}
        H^{0}\left(q\dash\HHdg_{R}/(1-q^{m})\right)\cong \left(\Hcal_{(R,\square)}/(1-q^{m})\right)^{h\ZZ^{d}}.
    \end{equation}
    Thus $H^{0}\left(q\dash\HHdg_{R}/(1-q^{m})\right)$ is $\ZZ$-torsion free, as the right hand side -- which by unwinding the definitions is $H^{0}\left(q\dash\HHdg_{(R,\square)}/(1-q^{m})\right)$ -- is $\ZZ$-torsion free by \Cref{prop: Habiro cohomology is ZZ torsion free} so there is an embedding into $\prod_{d|m}H^{0}\left(q\dash\HHdg_{R}\otimes_{\ZZ[q^{\pm}]}\QQ(\zeta_{d})\right)$ by rationalization inducing the diagram
    $$% https://q.uiver.app/#q=WzAsMyxbMCwwLCJIXnswfVxcbGVmdChxXFxkYXNoXFxISGRnX3tSfS8oMS1xXnttfSlcXHJpZ2h0KSJdLFsyLDAsIlxccHJvZF97ZHxtfUheezB9XFxsZWZ0KHFcXGRhc2hcXEhIZGdfe1J9XFxvdGltZXNfe1xcWlpbcV57XFxwbX1dfVxcUVEoXFx6ZXRhX3tkfSlcXHJpZ2h0KSJdLFsyLDEsIlxccHJvZF97ZHxtfVJcXG90aW1lc197XFxaWn1cXFFRKFxcemV0YV97ZH0pIl0sWzAsMSwiLVxcb3RpbWVzX3tcXFpafVxcUVEiLDAseyJzdHlsZSI6eyJ0YWlsIjp7Im5hbWUiOiJob29rIiwic2lkZSI6InRvcCJ9fX1dLFsxLDIsIlxcd3IgXFxoc3BhY2V7MC4yY219XFx0ZXh0eyhpaSl9Il0sWzAsMiwiIiwyLHsic3R5bGUiOnsidGFpbCI6eyJuYW1lIjoiaG9vayIsInNpZGUiOiJ0b3AifX19XV0=
    \begin{tikzcd}
        {H^{0}\left(q\dash\HHdg_{R}/(1-q^{m})\right)} && {\prod_{d|m}H^{0}\left(q\dash\HHdg_{R}\otimes_{\ZZ[q^{\pm}]}\QQ(\zeta_{d})\right)} \\
        && {\prod_{d|m}R\otimes_{\ZZ}\QQ(\zeta_{d})}
        \arrow["{-\otimes_{\ZZ}\QQ}", hook, from=1-1, to=1-3]
        \arrow[hook, from=1-1, to=2-3]
        \arrow["{\wr \hspace{0.2cm}\text{(ii)}}", from=1-3, to=2-3]
    \end{tikzcd}$$
    with the isomorphism on the right by assumption (ii) of the theorem. The image of the map $H^{0}\left(q\dash\HHdg_{R}/(1-q^{m})\right)\hookrightarrow\prod_{d|m}R\otimes_{\ZZ}\QQ(\zeta_{d})$ is necessarily the $q$-Witt vectors $q\dash W_{m}(R)$ after using the identification of (\ref{eqn: assumption i identification}) and applying \Cref{thm: image after rationalization is independent}. This shows 
    $$H^{0}\left(q\dash\HHdg_{R}/(1-q^{m})\right)\cong q\dash W_{m}(R)$$
    and is further compatible with specializations at $\zeta_{d}$ for $d$ dividing $m$. The universal property then induces a surjective map of commutative differential graded algebras 
    $$% https://q.uiver.app/#q=WzAsMixbMiwwLCJcXGxlZnQoSF57XFxidWxsZXR9XFxsZWZ0KHFcXGRhc2hcXEhIZGdfe1J9LygxLXFee219KVxccmlnaHQpLCBcXHRpbWVzKDEtcV57bX0pXFxyaWdodCkiXSxbMCwwLCJcXE9tZWdhXntcXGJ1bGxldH1fe3FcXGRhc2ggV197bX0oUikvXFxaWn0iXSxbMSwwLCIiLDAseyJzdHlsZSI6eyJoZWFkIjp7Im5hbWUiOiJlcGkifX19XV0=
    \begin{tikzcd}
        {\Omega^{\bullet}_{q\dash W_{m}(R)/\ZZ}} && {\left(H^{\bullet}\left(q\dash\HHdg_{R}/(1-q^{m})\right), \times(1-q^{m})\right)}
        \arrow[two heads, from=1-1, to=1-3]
    \end{tikzcd}$$
    which is surjective, inducing an isomorphism $q\dash W_{m}\Omega^{i}_{R/\ZZ}\xrightarrow{\sim}H^{i}\left(q\dash\HHdg_{R}/(1-q^{m})\right)$ after a choice of framing -- note that after taking the quotient by $(1-q^{m})$, the we have $H^{i}\left(q\dash\HHdg_{R}/(1-q^{m})\right)\cong H^{i}\left(q\dash\HHdg_{(R,\square)}/(1-q^{m})\right)\cong q\dash W_{m}\Omega^{i}_{R/\ZZ}$, the first isomorphism by the framing-independence at specializations result of \Cref{thm: surjection from q witt vectors} and the second as discussed in the proof of \Cref{thm: image after rationalization is independent}.\marginpar{The instructor remarks that multiple proofs in this lecture were statements about the empty set, which may very well be the author's favorite set.}

    We show the failure at the trivial root of unity, and other roots of unity can be treated by a a similar argument using a Frobenius twist of the element. By \Cref{corr: p-completion mod 1-q} and uniqueness of deformations of \'{e}tale algebras, we have an isomorphism 
    $$\left(R_{p}^{\wedge}[[q-1]]\right)[0]\xrightarrow{\sim}\left(q\dash\HHdg_{R}\right)^{\wedge}_{(p,1-q)}$$
    giving us an explicit description of $q\dash\HHdg_{R}$ after $(p,q-1)$-adic completion. To get a contradiction, let $R=\ZZ_{p}\langle t^{1/p^{\infty}}\rangle/(t-p)$ and we have $(\LL_{R/\ZZ})_{p}^{\wedge}\cong R[1]$. So $(\LL^{i}_{R/\ZZ})_{p}^{\wedge}\cong R[i]$ so $\left(q\dash\HHdg_{R}\right)^{\wedge}_{(p,1-q)}$ is $R$ concentrated in degree 0. In particular, there must exist a map 
    $$\left(q\dash\HHdg_{\ZZ_{p}\langle t^{1/p^{\infty}}\rangle}\right)^{\wedge}_{(p,q-1)}\longrightarrow\left(q\dash\HHdg_{R}\right)^{\wedge}_{(p,1-q)}$$
    by functoriality. The source is given by $\left(\ZZ_{p}\langle t^{1/p^{\infty}}\rangle[[q-1]]\right)[0]$ by the above, and $\left(q\dash\HHdg_{R}\right)^{\wedge}_{(p,1-q)}$ is given by adjoining $q$-divided powers of $(t-p)$ to the source. The element $t-p$ in the source must map to the unique element of the form $\frac{t-p}{q-1}$ as the source is $(q-1)$-torsion free. 
    
    Similarly, at other roots of unity, $t-p$ in the source maps to the unique element $\frac{\varphi_{p}(t-p)-[p]_{q}\delta_{p}(t-p)}{1-q^{p}}$ where here we have used the $\delta$-ring structure on the $q$-de Rham-Witt complex. From here, we can observe that $\delta(t-p)$ is a unit vanishing in $R/(p)$ so $R\cong 0$ by applying the derived Nakayama lemma, a contradiction. 
\end{proof}

\section{Lecture 6 -- 27th May 2025}\label{sec: lecture 6}
We continue with the proof of the result of van der Put-Schneider and Huber stating that the space of van der Put points on $\Sp(A)$ is homeomorphic to $\Spa(A,A^{\circ})$. 

We can define a morphism $\Spa(A,A^{\circ})\to \Sp(A)^{*}$ by 
\begin{equation}\label{eqn: vdP points map}
    (\nu/\sim)\mapsto\xi_{\nu}=\left\{R_{\Sp(A)}(f_{0}|f_{1},\dots,f_{n}):\nu(f_{i})\leq\nu(f_{0}),\forall 1\leq i\leq n\right\}.
\end{equation}
This can be shown to be well-defined. 
\begin{proposition}\label{prop: map from Spa exists}
    Let $A$ be an affinoid $K$-algebra, $\Omega=R_{\Sp(A)}(f_{0}|f_{1},\dots,f_{n})$, and $B=\Ocal_{\Sp(A)}(\Omega)$. Then the map (\ref{eqn: vdP points map}) exists and is well-defined. 
\end{proposition}
\begin{proof}
    We first show that $\xi_{\nu}$ is a van der Put point. If $\Omega\subseteq\Theta\subseteq X$ and $\Omega\in\xi_{\nu}$ then $\nu\in R_{\Spa(A,A^{\circ})}(f_{0}|f_{1},\dots,f_{n})$ extends to $\nu_{\Omega}\in\Spa(\Ocal_{\Sp(A)}(\Omega),\Ocal_{\Sp(A)}(\Omega)^{\circ})$. Consider the restriction map $\Ocal_{\Sp(A)}(\Theta)\to\Ocal_{\Sp(A)}(\Omega)$. The map is continuous and the image of $\Ocal_{\Sp(A)}(\Theta)^{\circ}$ is in $\Ocal_{\Sp(A)}(\Omega)^{\circ}$. By \Cref{lem: key lemma vdP points}, there is a unique extension to $\nu_{\Theta}\in\Spa(\Ocal_{\Sp(A)}(\Theta),\Ocal_{\Sp(A)}(\Theta)^{\circ})$ so $\Theta\in\xi_{\nu}$. 

    Having shown the upwards closure property, it remains to show the sieve properties. Let $\Omega=R_{\Sp(A)}(f_{0}|f_{1},\dots,f_{n}),\Theta=R_{\Sp(A)}(g_{0}|g_{1},\dots,g_{m})\in\xi_{\nu}$. We have $\nu(f_{0}g_{0})\geq\nu(f_{i}g_{j})$ for all $1\leq i\leq n,1\leq j\leq m$ so $\Omega\cap\Theta=R_{\Sp(A)}(f_{0}g_{0}|f_{i}g_{j})\in\xi_{\nu}$. This shows verifies the second van der Put point condition. Now for $\Omega\in\xi_{\nu}$, let $\Scal$ be a covering sieve of $\Omega$. We need to show that $\Scal\cap\xi_{\nu}$ is nonempty. We proceed by induction on the (necessarily finite) Laurent order of the sieve. If $\ofrak_{L}(\Scal)=0$ then $\Omega\in\Scal$ and the assertion follows. Observe that there is $g\in\Ocal_{\Sp(A)}(\Omega)$ such that $\ofrak_{L}(\Scal|_{\Omega_{1}})<\ofrak_{L}(\Scal)$ or $\ofrak_{L}(\Scal|_{\Omega_{2}})<\ofrak_{L}(\Scal)$ where $\Omega_{1}=R_{\Omega}(g|1),\Omega_{2}=R_{\Omega}(1|g)$. If $\nu_{\Omega}(g)\leq 1$ then by \Cref{lem: key lemma vdP points} replacing $\Sp(A)$ by $\Omega$ and $\nu$ by $\nu_{\Omega}$ we get an extension to an element of $\Spa(\Ocal_{\Sp(A)}(\Omega_{2}),\Ocal_{\Sp(A)}(\Omega_{2})^{\circ})$. Arguing similarly in the case of $\nu_{\Omega}(g)>1$ we get an extension to an element of $\Spa(\Ocal_{\Sp(A)}(\Omega_{1}),\Ocal_{\Sp(A)}(\Omega_{1})^{\circ})$. The $\Omega_{i}$ are rational open subsets and by the induction assumption $\emptyset\neq(\Scal|_{\Omega_{i}})\cap\xi_{\nu}\subseteq\Scal\cap\xi_{\nu}$. The former is nonemtpy, and thus so is the latter. This too shows the well-definedness of the map. 
\end{proof}
\Cref{prop: map from Spa exists} does the bulk of the work of proving the theorem of Huber and van der Put-Schneider. We first show that this map is a bijection. 
\begin{definition}[$\xi$-Infinitesmal]\label{def: xi-infinitesmal}
    Let $A$ be an affinoid $K$-algebra and $\xi\in\Sp(A)^{*}$ be a van der Put point. An element $a\in A$ is $\xi$-infinitesmal if and only if $R_{\Sp(A)}(\varepsilon|a)\in\xi$ for all $\varepsilon\in K^{\times}$. 
\end{definition} 
\begin{definition}[$\xi$-Ordering]\label{def: xi-ordering}
    Let $A$ be an affinoid $K$-algebra and $\xi\in\Sp(A)^{*}$ be a van der Put point. If $a$ is not $\xi$-infinitesmal, then we say $b\preceq_{\xi}a$ if and only if the following equivalent conditions hold:
    \begin{enumerate}[label=(\roman*)]
        \item $R_{\Sp(A)}(a|b,\varepsilon)\in\xi$ for some $\varepsilon\in K^{\times}$. 
        \item $R_{\Sp(A)}(a|b,\varepsilon)\in\xi$ for all $\varepsilon\in K^{\times}$ with $|\varepsilon|$ small enough. 
    \end{enumerate}
\end{definition}
\begin{remark}
    We have for $a$ $\xi$-infinitesmal, $b\preceq_{\xi}a$ if and only if $b$ is $\xi$-infinitesmal and we say $a\simeq_{\xi}b$ if and only if $b\preceq_{\xi}a$ and $a\preceq_{\xi}b$. 
\end{remark}
We can show that the ordering of \Cref{def: xi-ordering} defines a linear partial order on the non-$\xi$-infintesmal elments of $A$. 
\begin{proposition}\label{prop: xi-ordering is linear}
    Let $\xi$ be a van der Put point on $\Sp(A)$. The $\xi$-ordering is a linear partial order on the monoid $\{a\in A:a\text{ not }\xi\text{-infinitesmal}\}$. 
\end{proposition}
\begin{proof}
    If $a,b$ are not $\xi$-infinitesmal then $R_{\Sp(A)}(a|\varepsilon),R_{\Sp(A)}(b|\varepsilon)\in\xi$ for some $\varepsilon\in K^{\times}$. Their intersection $R_{\Sp(A)}(a|\varepsilon)\cap R_{\Sp(A)}(b|\varepsilon)=R_{\Sp(A)}(ab|\varepsilon^{2})$ is in $\xi$ and the covering $R_{\Sp(A)}(a|b,\varepsilon)\cup R_{\Sp(A)}(b|a,\varepsilon)$ is admissable (i.e. in the Grothendieck topology) as it is finite, hence one of $R_{\Sp(A)}(a|b,\varepsilon), R_{\Sp(A)}(b|a,\varepsilon)$ lie in $\xi$ giving one of $b\preceq_{\xi}a$ or $a\preceq_{\xi}b$. 

    To show this partial order is linear, suppose we have $a,b,c$ non-infinitesmal and $ac\preceq_{\xi}bc$. Then $R_{\Sp(A)}(bc|ac,\varepsilon)\in\xi$ and $R_{\Sp(A)}(c|\delta)\in\xi$ for $\varepsilon,\delta\in K^{\times}$. When $\widetilde{\varepsilon}\in K^{\times}$ such that $|\widetilde{\varepsilon}|\cdot\Vert c|A\Vert_{\max}\leq |\varepsilon|$ we have $R_{\Sp(A)}(c|\delta)\cap R_{\Sp(A)}(bc|ac,\epsilon)\subseteq R_{\Sp(A)}(b|a,\widetilde{\varepsilon})$ so $a\preceq_{\xi}b$, that is, the monoid is cancellative. 
\end{proof}
The data of the linear order allows us to define a map to a linearly ordered group, and in fact prescribes a point of the adic spectrum. 
\begin{proposition}
    Let $\xi$ be a van der Put point on $\Sp(A)$ and $\Gamma$ denote the quotient of the linearly ordered monoid $\{a\in A:a\text{ not }\xi\text{-infinitesmal}\}$ by the equivalence $\simeq_{\xi}$. There is a continuous map 
    \begin{equation}\label{eqn: defines a point of adic space}
        \nu_{\xi}:A\to\Gamma\cup\{0\}\text{ by }\nu_{\xi}(a)=\begin{cases}
            0 & a\text{ is }\xi\text{-infinitesmal} \\
            [a]_{\xi} & \text{otherwise}
        \end{cases}
    \end{equation}
    with $\nu_{\xi}\in\Spa(A,A^{\circ})$. 
\end{proposition}
\begin{proof} (TODO: the proof makes no sense)
    It is clear that $\nu_{\xi}$ is a valuation. If $\gamma\in\Gamma$ with $\gamma=\frac{[a]_{\xi}}{[b]_{\xi}}$ then for $R_{\Sp(A)}(a|\varepsilon)\cap R_{\Sp(A)}(b|\varepsilon)\in\xi$ and $\delta$ such that $\delta\cdot\Vert b|A\Vert_{\max}<\varepsilon$ then $\Vert f|A\Vert_{\max}<\varepsilon\delta$ implies that $\nu_{\xi}(f)<\gamma$ showing continuity. $\nu_{\xi}$ satisfies the correct compatibility condition with respect to $A^{\circ}$ as powerbounded elements are $\xi$-infinitesmal. 
\end{proof}
This is inverse to the construction (\ref{eqn: vdP points map}). 
\begin{proposition}\label{prop: mutually inverse}
    Let $A$ be an affinoid $K$-algebra. The constructions (\ref{eqn: vdP points map}) and (\ref{eqn: defines a point of adic space}) are mutually inverse. 
\end{proposition}
\begin{proof}
    Suppose $\xi$ is given.
Let $\Omega=R_{\Sp(A)}(f_{0}|f_{1},\dots,f_{n})\in\xi$ then $$R_{\Sp(A)}(f_{0}|f_{1},\dots,f_{n},\varepsilon)=R_{\Sp(A)}(f_{0}|f_{1},\dots,f_{n})\in\xi$$ when $\varepsilon$ is small enough, showing $\nu_{\xi}(f_{i})\leq\nu_{\xi}(f_{0})$ and $\Omega\in\xi_{\nu_{\xi}}$. Suppose that $\nu\in\xi_{\nu_\xi}$, that is, $\nu_{\xi}(f_{i})\leq\nu_{\xi}(f_{0})$. Since not all $f_{i}\in\supp(\nu_{\xi})=\{f\in A:f\text{ is }\xi\text{-infinitesmal}\}$ we have $f_{0}\in\supp(\nu_{\xi})$ hence $f_{0}$ is not $\xi$-infinitesmal. Taking $\varepsilon\in K^{\times}$ small, all $R_{\Sp(A)}(f_{0}|f_{i},\varepsilon)\in\xi$ hence 
    $$\bigcap_{i=1}^{n}R_{\Sp(A)}(f_{0}|f_{1},\dots,f_{n})\subseteq R_{\Sp(A)}(f_{0}|f_{1},\dots,f_{n})$$ and $\bigcap_{i=1}^{n}R_{\Sp(A)}(f_{0}|f_{1},\dots,f_{n})\in\xi$. This shows $\xi=\xi_{\nu_{\xi}}$. 

    Dually suppose $\nu$ is given. It follows from the definitions and continuity that $a\in A$ is $\xi$-infinitesmal if and only if $\nu(a)=0$ and $a\geq b$ if and only if $\nu(a)\geq \nu(b)$ so indeed $\nu_{\xi_\nu}=\nu$. 
\end{proof}
This proves the bijection in full. It remains to show the assertion regarding the Krull dimension. 

For the statement on Krull dimension, we consider some general properties of specializations in $\Spa(A,A^{+})$ for $A$ Tate. 
\begin{lemma}\label{lem: cts map to spec A}
    Let $(A,A^{+})$ be a Huber pair with $A$ Tate. There is a continuous morphism $S:\Spa(A,A^{+})\to\spec(A)$ by $\nu\mapsto\supp(\nu)$
\end{lemma}
\begin{proof}
    We have $S^{-1}(\spec(A_{f}))=\bigcup_{m\geq0}R_{\Spa(A,A^{+})}(f|s^{m})$ for a topologically nilpotent unit $s\in A^{\times}\cap A^{\circ\circ}$. 
\end{proof}
The map $S$ detects when points have disjoint neighborhoods in $\Spa(A,A^{+})$. 
\begin{lemma}\label{lem: disjoint neighborhoods on Spa}
    Let $(A,A^{+})$ be a Huber pair with $A$ Tate. Then $S(x)\neq S(y)$ if and only if $x,y\in\Spa(A,A^{+})$ have disjoint open neighborhoods. 
\end{lemma}
\begin{proof}
    Without loss of generality, let $a\in A$ be such that $|a|_{x}>0$ but $|a|_{y}=0$. By continuity of $|\cdot|_{x}$ and $\lim_{m\to\infty}s^{m}=0$, $|a|_{x}\geq|s^{m}|_{x}$ when $m$ is sufficiently large. Thus $x\in R_{\Spa(A,A^{+})}(a|s^{m}),y\in R_{\Spa(A,A^{+})}(s^{m+1}|a)$ and these rational open subsets are disjoint. 
\end{proof}
Recall that for $X$ a space, a point $y\in X$ is a generification of $x\in X$ if $x\in\overline{\{y\}}$. 
\begin{remark}
    Since $\Spa(A,A^{+})$ is spectral, each irreducible component has a unique generic point so each point of $\Spa(A,A^{+})$ has a maximal generification.
\end{remark}
\begin{definition}[Vertical generification]\label{def: vertical generification}
    Let $(\nu/\sim)\in\Spa(A,A^{\circ})$, $\Gamma$ the group of values of $\nu$, and $\Lambda\subseteq\Gamma$ a convex subgroup. We define a valuation 
    $$(\nu/\Lambda)(a)=\begin{cases}
        0 & \nu(a) = 0 \\
        \nu(a)\Lambda & \text{otherwise.}
    \end{cases}$$ 
\end{definition}
These generifications are vertical in the sense that they occur within a single fiber of the map $S:\Spa(A,A^{\circ})\to\spec(A)$ of \Cref{lem: cts map to spec A}

\section{Lecture 7 -- 4th July 2025}\label{sec: lecture 7}
We continue our discussion of the construction of the analytic Habiro ring $\Hcal^{\an}$ and the functor $(-)^{\Hab}:\Sch_{\ZZ}^{\mathsf{sft}}\to\mathsf{AnStack}_{\Hcal^{\an}}$ from schemes separated and of finite type over the integers to analytic stacks over the Habiro ring which commutes with finite limits and gluing and such that the functor $X\mapsto\Dscr_{\QCoh}(X^{\Hab})$ is a six-functor formalism. It turns out that one can extend the functor $(-)^{\Hab}$ from $\Sch^{\mathsf{sft}}_{\ZZ}$ to Berkovich spaces of finite type over $\ZZ$. 

At least in the case of schemes, the functor $(-)^{\Hab}$ by $X\mapsto X^{\Hab}$ is fully determined by $(\A^{1}_{\ZZ})^{\Hab}$ as a ring object in stacks. In the setting of Berkovich spaces, we need to supply additional structure.\marginpar{Recall here that the Berkovich spectrum is defined to be the set of multiplicative seminorms bounded by the norm.} 

For this, recall that the Berkovich affine line $\A^{1,\Berk}_{\ZZ}$ admits a norm map $|T|$ to $[0,\infty)$ taking a point $|\cdot|_{x}$ in $\A^{1,\Berk}_{\ZZ}$ to its value $|T|_{x}$ on the coordinate $T$. Open discs in the Berkovich affine line $\A^{1,\Berk}_{\ZZ}$ are of the form $D_{r}(0)^{\circ}$, the preimage of $[0,r)\subseteq[0,\infty)$ under the map $|T|$. Similarly, the closed disc $D_{r'}(0)=\bigcap_{r'<r}D_{r}(0)^{\circ}$ is the intersection of all strictly larger open discs is the preimage of $[0,r']\subseteq[0,\infty)$ under $|T|$. This definition of closed discs captures its overconvergent nature, where a convergent function on the closed disc converges already in a slightly larger open disc. In particular, covers of the half-interval pull back to covers of the Berkovich affine line by discs. Under the functor $(-)^{\Hab}$, we would expect a similar correspondence on coverings. For this to make sense, we will require a notion of normed ring stacks: there is a norm map $N:(\A^{1,\Berk}_{\ZZ})^{\Hab}\to[0,\infty)_{\Betti}$ satisfying the usual axioms of a non-Archimedean norm where $[0,\infty)_{\Betti}$ is the analytic Betti stack associated to the topological space $[0,\infty)$. In particular, the value of the $(-)^{\Hab}$-functor will be determined on Berkovich spaces by $\left(\A^{1,\Berk}_{\ZZ}\right)^{\Hab}$ as a normed ring stack.  

Ongoing work of Aoki provides an $(\infty,2)$-categorical realizations of Berkovich motives are equivalent to normed ring stacks with certain properties. 
\begin{remark}
    We should be careful to distinguish between the existence of a norm on an analytic ring and a norm on a ring stack over an analytic ring. These are related in the sense that a norm on an analytic ring is precisely the norm on the affine line over it as a ring stack. 
\end{remark}
With this language in hand, we can restate our desideratum as finding a normed ring stack $(\A^{1}_{\ZZ})^{\Hab}$ over an analytic ring $\Hcal^{\an}$ such that $(\GG_{m})^{\Hab}$ is $\GG_{m,\Hcal^{\an}}/q^{\ZZ}$ as in \Cref{ex: Habiro stack on Gm}. The composition 
$$\GG_{m,\Hcal^{an}}\longrightarrow\GG_{m,\Hcal^{\an}}/q^{\ZZ}\cong(\GG_{m})^{\Hab}\xrightarrow{N}[0,\infty)_{\Betti}$$
exhibits $\GG_{m,\Hcal^{\an}}$ as a normed analytic stack. This constructs a norm on nonzero elements of $\Hcal^{\an}$ and by conventions on norms, a norm on the analytic ring $\Hcal^{\an}$ in the sense that the map $\AnSpec(\Hcal^{\an}[T])=\A^{1}_{\Hcal^{\an}}\to[0,\infty]_{\Betti}$ satisfies the expected properties and where the preimage of $[0,\infty)_{\Betti}$ is $\A^{1,\an}_{\Hcal^{\an}}$ the analytic affine line over the analytic Habiro ring. 
\begin{example}
    For a normed analytic ring $A$, there is a map $\A^{1}_{A}\to[0,\infty]_{\Betti}$ such that for any $A$-algebra $B$ and $f\in B$ the diagram 
    $$% https://q.uiver.app/#q=WzAsNSxbMSwwLCJcXEFeezF9X3tBfSJdLFszLDAsIlswLFxcaW5mdHldX3tcXEJldHRpfSJdLFsxLDEsIlxcQW5TcGVjKEIpIl0sWzAsMCwiVCJdLFswLDEsImYiXSxbMyw0LCIiLDAseyJzdHlsZSI6eyJ0YWlsIjp7Im5hbWUiOiJtYXBzIHRvIn19fV0sWzIsMF0sWzAsMSwifFR8Il0sWzIsMSwifGZ8IiwyXV0=
    \begin{tikzcd}
        T & {\A^{1}_{A}} && {[0,\infty]_{\Betti}} \\
        f & {\AnSpec(B)}
        \arrow[maps to, from=1-1, to=2-1]
        \arrow["{|T|}", from=1-2, to=1-4]
        \arrow[from=2-2, to=1-2]
        \arrow["{|f|}"', from=2-2, to=1-4]
    \end{tikzcd}$$
    induces a norm $|f|$ on $\AnSpec(B)$. In particular $|f|$ takes a point $|\cdot|_{x}\in\AnSpec(B)$ to $|f|_{x}$.  
\end{example}
\begin{example}
    Let $\ZZ((u))$ the Laurent series in $u$. Its solid analytic ring structure $A=\ZZ((u))_{\square}$ is a normed analytic ring for $0<|u|<1$.\marginpar{Note that the $\square$ here denotes solidity, not a framing as has been convention thus far.} Saying that $|u|=c$ for $0<c<1$ is saying that the map $|u|:\AnSpec(A)\to[0,\infty]_{\Betti}$ is the constant map with value $c$. The preimage of $[0,r]_{\Betti}\subseteq[0,\infty]_{\Betti}$ is the overconvergent disc $D_{0}(r)^{\dagger}=\{T\in\ZZ((u))_{\square}:|T|\leq r\}=\{T:|T|\leq |u|^{r}\}$. Equivalently, the sequence $\left(\frac{T^{k}}{u^{k-1}}\right)_{k}$ is a nullsequence. 
\end{example}
We can produce an affine analytic stack of overconvergent functions on this disc. 
\begin{definition}[Overconvergent Functions on the Disc]\label{def: overconvergent functions on the disc}
    The ring of overconvergent functions on the disc of radius $r$ is 
    \begin{align*}
        \ZZ((u))\langle T\rangle_{r}^{\dagger} &= \left\{\sum_{n\geq0}a_{n}T^{n}:a_{n}\in\ZZ((u)),\exists r'>r\text{ s.t. }|a_{n}|r'^{n}\to0\right\} \\
        &= \left\{\sum_{n\geq0}a_{n}T^{n}:a_{n}\in\ZZ((u)),\exists r'>r\text{ s.t. }2^{-\nu_{u}(a_{n})}\to0\right\}.
    \end{align*}
\end{definition}
This language gives us a more precise definition of the overconvergent disc. 
\begin{definition}[Overconvergent Disc]\label{def: overconvergent disc}
    The overconvergent disc of radius $r$ is the analytic spectrum of the overconvergent functions on the disc $\AnSpec(\ZZ((u))\langle T\rangle_{r}^{\dagger})_{\square}$. 
\end{definition}
This shows that $(\ZZ((u))\langle T\rangle_{r}^{\dagger})_{\square}$ is an idempotent $\ZZ((u))_{\square}[T]$-algebra: the canonical map 
$$(\ZZ((u))\langle T\rangle_{r}^{\dagger})_{\square}\xrightarrow{\id\otimes^{L}[\ZZ((u))_{\square}[T]\to(\ZZ((u))\langle T\rangle_{r}^{\dagger})_{\square}]}(\ZZ((u))\langle T\rangle_{r}^{\dagger})_{\square}\otimes_{\ZZ((u))_{\square}[T]}(\ZZ((u))\langle T\rangle_{r}^{\dagger})_{\square}$$ 
is an isomorphism. These become injections on taking $\AnSpec(-)$ giving $$\AnSpec(\ZZ((u))\langle T\rangle_{r}^{\dagger})_{\square}\to\AnSpec\ZZ((u))[T].$$
In the preceding discussion it was crucial that we used the condensed or topological nature of the algebra. 

We can now define the analytic affine line.
\begin{definition}[Analytic Affine Line over $\ZZ((u))_{\square}$]\label{def: analytic affine line over analytic Habiro}
    The analytic affine line over $\ZZ((u))_{\square}$ is 
    $$\A^{1,\an}_{\ZZ((u))_{\square}}=\bigcup_{r<\infty}D_{0}(r)^{\dagger}\subseteq\A^{1}_{\ZZ((u))_{\square}}.$$
\end{definition}
Functions on $\A^{1,\an}_{\ZZ((u))_{\square}}$ should be thought of as entire functions in $T$, that is, 
$$\left\{\sum_{n\geq0}a_{n}T^{n}:a_{n}\in\ZZ((u)),\forall r<\infty, |a_{n}|r^{n}\to0\right\}.$$
\begin{remark}
    There are far fewer functions on $\A^{1}_{\ZZ((u))_{\square}}$ than on $\A^{1,\an}_{\ZZ((u))_{\square}}$. 
\end{remark}
We can now define the analytic Habiro ring. 
\begin{definition}[Analytic Habiro Ring]\label{def: analytic Habiro ring}
    The analytic Habiro ring $\Hcal^{\an}$ is the free $\ZZ((u))_{\square}$-algebra with element $q$ where $|q|=1$ such that:
    \begin{enumerate}[label=(\roman*)]
        \item The sequence $((q;q)_{n})_{n\geq1}$ has rapid decay in the sense that for all $k\geq 1$ $\left(\frac{(q;q)_{n}}{u^{kn}}\right)_{n\geq1}$ is a nullsequence. 
        \item For all $n\geq 1$, $1-q^{n}$ is invertible in $\Hcal^{\an}$ and for all $\varepsilon>0$ the sequence $\left(\frac{u^{\varepsilon n}}{1-q^{n}}\right)_{n\geq1}$ is a nullsequence. 
    \end{enumerate}
\end{definition}
\begin{remark}
    This definition is only preliminary. Indeed, one would only want (i) in principle, but working with this object is significantly more difficult without (ii). 
\end{remark}
 
The analytic Habiro ring is a promotion of the Habiro ring, and admits a comparison map from it. In fact, this is already implied by the first condition. 
\begin{lemma}\label{lem: map from Habiro to analytic Habiro}
    There exists a morphism $\Hcal\to\Hcal^{\an}$. 
\end{lemma}
\begin{proof}
    Any element of the ordinary Habiro ring $\Hcal$ can be written as $\sum_{n\geq0}f_{n}(q)(q;q)_{n}$ where $f_{n}(q)\in\ZZ[q^{\pm}]$. Its image in $\Hcal^{\an}$ is of the form $\sum_{n\geq0}u^{n}f_{n}(q)\frac{(q;q)_{n}}{u^{n}}$. The terms $\left(\frac{(q;q)_{n}}{u^{n}}\right)_{n\geq1}$ form a nullsequence and thus $\sum_{n\geq0}u^{n}f_{n}(q)\frac{(q;q)_{n}}{u^{n}}$ is summable as Laurent polynomials are of bounded norm and $u^{n}$ has exponential decay. 
\end{proof}
The second condition in fact implies that the Habiro cohomology of the disc is trivial.
\begin{proposition}
    The Habiro cohomology of $D_{0}(1)^{\dagger}=\AnSpec(\ZZ((u))\langle T\rangle_{1}^{\dagger})_{\square}$ is trivial. 
\end{proposition} 
\begin{proof}
    The Habiro cohomology is computed via the complex $\Hcal^{\an}\langle T\rangle^{\dagger}_{1}\to\Hcal^{\an}\langle T\rangle^{\dagger}_{1}$ by $T^{n}\mapsto (1-q^{n})T^{n-1}$. The cohomology complex should be quasi-isomorphic to this one, and by invertibility of $(1-q^{n})$ and the growth condition in (ii) imply the complex is zero in higher degrees, showing the claim. 
\end{proof}
The triviality of cohomology on the disc is analogous to case of Berkovich motives, and makes the theory simpler in many ways. In particular (ii) removes undesirable phenomena of cyclotomic torsion. 

Having considered the case of the Habiro ring, we can consider analytic $\GG_{m}$ over the analytic Habiro ring. Recall that in the ordinary case, we have that $\GG_{m}^{\Hab}=\GG_{m,\Hcal}/q^{\ZZ}$ where the quotient prescribes the descent data of modules with a $q$-connection. 
\begin{definition}[Analytic $\GG_{m}$ on the Habiro Ring]\label{def: analytic Gm on Habiro ring}
    $(\GG_{m})^{\Hab}$ is the quotient $\GG^{\an}_{m,\Hcal^{\an}}/G$ where 
    $$G=\left\{x:((x;q)_{n})_{n\geq1}\text{ has rapid decay}\right\}\subseteq\GG_{m,\Hcal^{\an}}.$$
\end{definition}
\begin{remark}
    $$G=\AnSpec\left(\colim_{k}\Hcal^{\an}\left[\left(\frac{(x;q)_{n}}{u^{kn}}\right)_{n\geq0}\right]\right)$$
    where $\Hcal^{\an}\left[\left(\frac{(x;q)_{n}}{u^{kn}}\right)_{n\geq0}\right]$ is the free module on the nullsequence. Assumption (i) of \Cref{def: analytic Habiro ring} implies $q\in G$, and we should think of $G$ as a thickened $q^{\ZZ}$. 
\end{remark}
\begin{lemma}
    $f:*/q^{\ZZ}\to */G$ induces a fully faithful functor $f^{*}:\Dscr_{\QCoh}(*/G)\to\Dscr_{\QCoh}(*/q^{\ZZ})$. In particular, $\Dscr_{\QCoh}(\GG_{m}^{\Hab})=\Dscr(\GG_{m,\Hcal^{\an}}^{\an}/G)$ is a full subcategory of $\Dscr(\GG_{m,\Hcal^{\an}}^{\an}/q^{\ZZ})$. 
\end{lemma}
\begin{proof}[Proof Outline]
   Writing $(xy;q)_{n}$ as a polynomial in $(x;q)_{r}$ and $(y;q)_{n-r}$ with coefficients given by a $q$-power-multiple of a $q$-binomial, condition (ii) shows that $f_{*}$ of the unit of the source is the unit, giving full faithfulnes. 
\end{proof}
In this sense the category of coefficients for Habiro cohomology on $\GG_{m}^{\Hab}$ form a full subcategory of quasicoherent sheaves on $\GG_{m}/q^{\ZZ}$ satisfying a convergence condition. 

We now show that the analytic Habiro ring is nonempty. 
\begin{example}
    Consider 
    $$B_{1}=\QQ((u))_{\square}\langle q-1\rangle_{0}^{\dagger}\left[\frac{1}{q-1}\right]$$
    the ring of overconvergent ``meromorphic'' functions at the point $q=1$. 
    \begin{enumerate}[label=(\roman*)]
        \item $(q;q)_{n}$ is divisible by $(1-q)^{n}$ and for all $k$, $\left(\frac{(1-q)^{n}}{u^{kn}}\right)_{n}\to0$ as overconvergence implies the absolute value of $(1-q)$ is less than any power of $u$ by going to a smaller disc. 
        \item Using $1-q^{n}=(1-q)[n]_{q}$ where $[n]_{q}$ is a unit of norm 1 so 
        $$\frac{u^{\varepsilon n}}{1-q^{n}}=\frac{1}{1-q}\left(\frac{u^{\varepsilon n}}{[n]_{q}}\right)$$
        so the increasing powers of $u$ being a nullsequence imply this is a nullsequence too. 
    \end{enumerate}
    exhibiting the $\ZZ((u))_{\square}$-algebra $B_{1}$ as a specialization of the analytic Habiro ring. In particular, the analytic Habiro ring $\Hcal^{\an}$ is nonzero. 
\end{example}
\begin{remark}
    The main reason the conditions (i) and (ii) do not contradict each other in this example is that (ii) only asks that the sequences are not too large integrally, but their product behaves like a factorial so decays faster. 
\end{remark}
More generally, for all $m\geq 1$, we can define 
$$B_{m}=\QQ(\zeta_{m})((u))_{\square}\langle q-\zeta_{m}\rangle_{0}^{\dagger}\left[\frac{1}{q-\zeta_{m}}\right]$$
which also satisfies the conditions of the analytic Habiro ring. One can also show that $\Hcal^{\an}/p\neq0$ and connects to the theory of Berkovich motives in characteristic $p$, and in fact defines a new ring stack in that setting.  

Moreover, this construction is related to $q$-divided powers as we now discuss. Recall that to define the de Rham stack integrally, we use the quotient $\GG_{a}/\GG_{a}^{\sharp}$ in place of $\widehat{\GG_{a}}$ where $\GG_{a}^{\sharp}=\spec\left(\ZZ\left[\frac{T^{n}}{n!}\right]\right)$. So in the settting of $q$-de Rham cohomology, we would expect to use $\GG_{a,\Hcal}/\GG_{a,\Hcal}^{q\dash\sharp}$ where $\GG_{a}^{q\dash\sharp}=\spec\left(\Hcal\left[\frac{T^{n}}{(q;q)_{n}}\right]\right)$ but this is not a subgroup for addition, so this is not the correct approach. We can try and do a similar approach using $\GG_{m,\Hcal}/\GG_{m,\Hcal}^{q\dash\sharp}$ where $\GG_{m,\Hcal}^{q\dash\sharp}=\spec\left(\Hcal\left[\frac{(T,q)_{n}}{(q;q)_{n}}\right]\right)$ which for a long time was the instructor's candidate for $\GG_{m}^{\Hab}$, but this does not work as $q$-divided powers are in general poorly behaved. In particular, difficulties arise in defining addition. 
\appendix 
\section{Explicit Elements of the Habiro Ring \\ (d'apr\`{e}s Garoufalidis-Wheeler)}\label{appdx: explicit elements}
This appendix contains the proof sketch of \Cref{ex: element of Habiro ring}. The interested reader is encouraged to consult \cite{GWClassesHabiroCoh} for further details. 

We want to show that the element 
$$\frac{1}{1-T_{1}-\dots-T_{d}}=\sum_{k_{1},\dots,k_{d}\geq0}\binom{k_{1}+\dots+k_{d}}{k_{1} \dots k_{d}}T_{1}^{k_{1}}\dots T_{d}^{k_{d}}\in\ZZ[[\underline{T}]]$$
has a $q$-deformation. More precisely, we seek to show that for $R=\ZZ[T_{1},\dots,T_{d},\frac{1}{1-T_{1}-\dots-T_{d}}]$ that the element 
\begin{equation}\label{eqn: Habiro element}
    \sum_{k_{1},\dots,k_{d}\geq0}\left[\substack{k_{1}+\dots+k_{d} \\ k_{1}\text{ }\dots\text{ }k_{d}}\right]_{q}T_{1}^{k_{1}}\dots T_{d}^{k_{d}}\in\ZZ[q^{\pm}][[\underline{T}]]
\end{equation}
obtained by $q$-deformation of the multinomial lies in $\Hcal_{(R,\square)}$ where $\square:\ZZ[q^{\pm}][\underline{T}]\to R$ is the obvious map. For this, it sufficse to show that $R^{(m)}[[q-\zeta_{m}]]\subseteq\ZZ[\zeta_{m}][[\underline{T},q-\zeta_{m}]]$. 

We have that 
$$R^{(m)}=\ZZ\left[T_{1},\dots,T_{d},\frac{1}{1-T_{1}^{m}-\dots -T_{d}^{m}}\right]$$
and that the Frobenius gluing is already completely determined by the injectivity $R^{(m)}[[q-\zeta_{m}]]\hookrightarrow\ZZ[\zeta_{m}][[\underline{T},q-\zeta_{m}]]$ as it can be checked after $\underline{T}$-adic completion. Note, furthermore, that 
{\footnotesize
$$\ZZ\left[T_{1},\dots,T_{d},\frac{1}{1-T_{1}^{m}-\dots-T_{d}^{m}}\right]=\QQ\left[T_{1},\dots,T_{d},\frac{1}{1-T_{1}^{m}-\dots-T_{d}^{m}}\right]\bigcap\ZZ[[T_{1},\dots,T_{d}]]$$
\normalsize}as subrings of $\QQ[[T_{1},\dots,T_{d}]]$, so it suffices to verify the statement rationally. Using $q=\zeta_{m}\exp(h)$, we get an isomorphism $\QQ(\zeta_{m})[[q-\zeta_{m}]]\cong\QQ(\zeta_{m})[[h]]$ and seek to develop (\ref{eqn: Habiro element}) as a power series in $h$: 
\begin{align*}
    1-q^{i} &= 1-\exp(ih) = -ih - \frac{i^{2}}{2} h^{2}-\dots \\
    (q;q)_{n}&=(-1)^{n}!h^{n}\left(1 + \frac{n(n+1)}{4}+\dots\right)
\end{align*}
Using that 
$$\left[\substack{k_{1}+\dots+k_{d} \\ k_{1}\text{ }\dots\text{ }k_{d}}\right]_{q}=\binom{k_{1}+\dots+k_{d}}{k_{1}\dots k_{d}}\cdot O(h)$$
where $O(h)$ is a power series in $h$ with coefficients in $\QQ[k_{1},\dots,k_{d}]$, we have that each term in the power series expansion in $h$ at $m=1$ is of the form 
\begin{equation}\label{eqn: h series expansion}
    \sum_{k_{1},\dots,k_{d}\geq0}\binom{k_{1}+\dots+k_{d}}{k_{1}\dots k_{d}}P(k_{1},\dots,k_{d})T_{1}^{k_{1}}\dots T_{d}^{k_{d}}.
\end{equation}
We then use the following lemma. 
\begin{lemma}
    Let $P(k_{1},\dots, k_{d})\in\QQ[k_{1},\dots,k_{d}]$ as in (\ref{eqn: h series expansion}) lies in $$R=\QQ\left[T_{1},\dots,T_{d},\frac{1}{1-T_{1}-\dots-T_{d}}\right].$$ 
\end{lemma}
\begin{proof}
    Without loss of generality, we can take $P$ to be a monomial $k_{1}^{a_{1}}\dots k_{d}^{a_{d}}$. We get, up to a constant, that $(\nabla_{1}^{\log})^{a_{1}}\dots(\nabla_{d}^{\log})^{a_{d}}$ of $\frac{1}{1-T_{1}-\dots-T_{d}}$ lies in $R$. 
\end{proof}
More generally the power series expansion at $m$ is given by 
\begin{equation}\label{eqn: h series expansion at m}
    \sum_{k_{1},\dots,k_{d}\geq0}\binom{mk_{1}+\dots+mk_{d}}{mk_{1}\dots mk_{d}}P(k_{1},\dots,k_{d})T_{1}^{mk_{1}}\dots T_{d}^{mk_{d}}
\end{equation}
which by similar arguments can be shown to lie in $R$ as well. 
\include{Appendix 2}
\printbibliography
\end{document}