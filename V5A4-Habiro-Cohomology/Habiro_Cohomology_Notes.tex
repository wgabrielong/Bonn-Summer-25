\documentclass{amsart}
\usepackage[margin=1.5in]{geometry} 
\usepackage{amsmath}
\usepackage{tcolorbox}
\usepackage{amssymb}
\usepackage{amsthm}
\usepackage{lastpage}
\usepackage{fancyhdr}
\usepackage{accents}
\usepackage{hyperref}
\usepackage{xcolor}
\usepackage{color}
\usepackage[bbgreekl]{mathbbol}
\DeclareSymbolFontAlphabet{\mathbb}{AMSb}
\DeclareSymbolFontAlphabet{\mathbbl}{bbold}
\input{shortcuts.tex}
\setlength{\headheight}{40pt}


\newenvironment{solution}
  {\renewcommand\qedsymbol{$\blacksquare$}
  \begin{proof}[Solution]}
  {\end{proof}}
\renewcommand\qedsymbol{$\blacksquare$}

\usepackage{amsmath, amssymb, tikz, amsthm, csquotes, multicol, footnote, tablefootnote, biblatex, wrapfig, float, quiver, mathrsfs, cleveref, enumitem, stmaryrd, marginnote, todonotes, euscript, bbm}
\addbibresource{refs.bib}
\theoremstyle{definition}
\newtheorem{theorem}{Theorem}[section]
\newtheorem{lemma}[theorem]{Lemma}
\newtheorem{corollary}[theorem]{Corollary}
\newtheorem{exercise}[theorem]{Exercise}
\newtheorem{question}[theorem]{Question}
\newtheorem{example}[theorem]{Example}
\newtheorem{proposition}[theorem]{Proposition}
\newtheorem{conjecture}[theorem]{Conjecture}
\newtheorem{remark}[theorem]{Remark}
\newtheorem{definition}[theorem]{Definition}
\numberwithin{equation}{section}
\setuptodonotes{color=blue!20, size=tiny}
\begin{document}
\large
\title[Habiro Cohomology -- Bonn, Summer 2025]{V5A4 -- Habiro Cohomology \\ Summer Semester 2025}
\author{Wern Juin Gabriel Ong}
\address{Universit\"{a}t Bonn, Bonn, D-53113}
\email{wgabrielong@uni-bonn.de}
\urladdr{https://wgabrielong.github.io/}
\maketitle
\section*{Preliminaries}
These notes roughly correspond to the course \textbf{V5A4 -- Habiro Cohomology} taught by Prof. Peter Scholze at the Universit\"{a}t Bonn in the Summer 2025 semester. These notes are \LaTeX-ed after the fact with significant alteration and are subject to misinterpretation and mistranscription. Use with caution. Any errors are undoubtedly my own and any virtues that could be ascribed to these notes ought be attributed to the instructor and not the typist. Recordings of the lecture are availible at the following link:
\begin{center}
  \href{https://archive.mpim-bonn.mpg.de/id/eprint/5155/}{\texttt{archive.mpim-bonn.mpg.de/id/eprint/5155/}}
\end{center}

\tableofcontents
\section{Lecture 1 -- 7th April 2025}\label{sec: lectuer 1}
\section{Lecture 2 -- 24th April 2025}\label{sec: lecture 2}
We define coherent sheaves. 
\begin{definition}[Coherent Sheaves]\label{def: coherent sheaves}\marginpar{Definition 2.2}
    Let $A$ be an affinoid $K$-algebra and $\Fcal$ an $\Ocal_{\Sp(A)}$-module. $\Fcal$ is coherent if it is of the form $\widetilde{M}$ for some finitely generated $A$-module $M$. 
\end{definition}
The coherence condition can be shown to be local in the sense that any sheaf of modules for which there exists a covering sieve consisting of a trivialization by a cover by rational opens on which the modules are finitely generated is coherent. We show this as a consequence of a sequence of results. 
\begin{proposition}\label{prop: asymptotic sections of structure sheaf}\marginpar{Proposition 2.3}
  Let $A$ be an affinoid $K$-algebra and $\Omega_{1},\Omega_{2}$ a rational cover of $\Sp(A)$ with intersection $\Omega_{12}$. If $f_{12}\in\Ocal_{\Sp(A)}(\Omega_{12})$ then $f_{12}=f_{1}|_{\Omega_{12}}+f_{2}|_{\Omega_{12}}$ for $f_{i}\in\Ocal_{\Sp(A)}(\Omega_{i})$ with $\Vert f_{i}\Vert=O(\Vert f_{12}\Vert)$. 
\end{proposition}
\begin{proof}
    Omitted. 
\end{proof}
\begin{lemma}\label{lem: weak bound for double intersection elements}\marginpar{Lemma 2.1}
    Let $\Mcal$ be a sheaf of $\Ocal_{\Sp(A)}$-modules on $\Sp(A)$ and $\Omega_{1},\Omega_{2}$ a rational cover of $\Sp(A)$ on which $\Mcal|_{\Omega_{i}}=M_{i}$ with $M_{i}$ finitely generated. If $m_{12}\in\Mcal(\Omega_{12})$ then there are $m_{i}\in\Mcal(\Omega_{i})$ with $\Vert m_{i}|\Mcal(\Omega_{i})\Vert=O(\Vert m_{12}\Vert)$ (the constant independent of $m_{12}$) and such that 
    $$\left\Vert m_{12}-m_{1}|_{\Omega_{12}}-m_{2}|_{\Omega_{12}}\right\Vert\leq\frac{1}{2}\left\Vert m_{12}\right\Vert.$$
\end{lemma}\todo{Finish proof.}
%\begin{proof}
    %Let $(m_{j}^{(1)})_{j=1}^{N_{1}},(m_{j}^{(2)})_{j=1}^{N_{2}},(m_{j}^{(12)})_{j=1}^{N_{12}}$ be generators for $M_{1},M_{2},M_{12}$, respectively. Note that all $\mu^{(12)}$ with $\max_{1\leq j\leq N_{12}}\Vert \mu^{(12)}-m_{j}^{(12)}\Vert\leq\frac{1}{2}$ generate $\Mcal(\Omega_{12})$. By density of $\Mcal(\Omega_{i})$ in $\Mcal(\Omega_{12})$ these can be chosen from $\Mcal(\Omega_{i})|_{\Omega_{12}}$. Thus, without loss of generality, we can take $(m^{(1)}_{j}|_{\Omega_{12}})_{j=1}^{N_{1}}$. 

    %We write 
    %$$m_{12}=\sum_{j=1}^{N_{1}}f_{j}\cdot(m_{j}^{(1)}|_{\Omega_{12}})$$
    %with $f_{j}\in\Ocal_{\Sp(A)}(\Omega_{12})$
%\end{proof}
\begin{lemma}\label{lem: module element decomposition}\marginpar{Lemma 2.2}
    Let $\Mcal$ be a sheaf of $\Ocal_{\Sp(A)}$-modules on $\Sp(A)$ and $\Omega_{1},\Omega_{2}$ a rational cover of $\Sp(A)$ on which $\Mcal|_{\Omega_{i}}=M_{i}$ with $M_{i}$ finitely generated. If $m_{12}\in \Mcal(\Omega_{12})$ then 
    $$m_{12}=m_{1}|_{\Omega_{12}}+m_{2}|_{\Omega_{12}}$$
    where $m_{i}\in\Mcal(\Omega_{i})$ and $\Vert m_{i}|\Mcal(\Omega_{i})\Vert=O(\Vert m_{12}|\Mcal(\Omega_{12})\Vert)$ with the implied constant independent of $m_{12}$. 
\end{lemma}
\begin{proof}
    Let $C$ be the implied constant of \Cref{lem: weak bound for double intersection elements}. We can define recursively 
    \begin{align*}
        m_{12} = m_{12}^{(0)} &= m_{1}^{(0)}|_{\Omega_{12}}+m_{2}^{(0)}|_{\Omega_{12}}+m_{12}^{(1)} \\
        m_{12}^{(1)} &= m_{1}^{(1)}|_{\Omega_{12}} + m_{2}^{(0)}|_{\Omega_{12}} + m_{12}^{(2)} \\
        \vdots &\hspace{2cm}\vdots
    \end{align*}
    where $\Vert m_{12}^{(i+2)}|\Mcal(\Omega_{12})\Vert\leq\frac{1}{2}\Vert m_{12}^{(i)}|\Mcal(\Omega_{12})\Vert$ and $\Vert m_{j}^{(i)}|\Mcal(\Omega_{j})\Vert\leq C\Vert m_{12}^{(i)}|\Mcal(\Omega_{12})\Vert$, hence $\Vert m_{12}^{(i)}|\Mcal(\Omega_{12})\Vert\leq \frac{1}{2^{i}}\Vert m_{12}|\Mcal(\Omega_{12})\Vert$ and $\Vert m_{j}^{(i)}|\Mcal(\Omega_{j})\Vert\leq\frac{C}{2^{i}}\Vert m_{12}|\Mcal(\Omega_{12})\Vert$ and the assertion follows with $m_{j}=\sum_{i=0}^{\infty}m_{j}^{(i)}\in\Mcal(\Omega_{j})$ with the implied constant $C$. 
\end{proof}
From this we deduce: 
\begin{corollary}\label{corr: sum decomposition smaller than epsilon}\marginpar{Corollary 2.1}
    Let $\Mcal$ be a sheaf of $\Ocal_{\Sp(A)}$-modules on $\Sp(A)$ and $\Omega_{1},\Omega_{2}$ a rational cover of $\Sp(A)$ on which $\Mcal|_{\Omega_{i}}=M_{i}$ with $M_{i}$ finitely generated. If $m_{12}\in\Mcal(\Omega_{12})$ and $\varepsilon>0$ then there are $m_{i}\in\Mcal(\Omega_{i})$ such that $m_{12}=m_{1}|_{\Omega_{12}}+m_{2}|_{\Omega_{12}}$ and $\Vert m_{2}|\Mcal(\Omega_{2})\Vert<\varepsilon$. 
\end{corollary}
\begin{proof}
    Choose $m_{2}'\in\Mcal(\Omega_{2})$ such that $\Vert m_{12}-m_{2}|_{\Omega_{12}}|\Mcal(\Omega_{12})\Vert<\delta$ then $m_{12}-m_{2}'=m_{1}+m_{2}''$ with $\Vert m_{1}|\Mcal(\Omega_{1})\Vert+\Vert m_{2}''|\Mcal(\Omega_{2})\Vert\leq C\cdot\delta$ with $C$ as in \Cref{lem: module element decomposition}. Then choose $\delta$ such that $C\cdot\delta<\varepsilon$. 
\end{proof}
\begin{corollary}\label{corr: generating tuple}\marginpar{Corollary 2.2}
    Let $\Mcal$ be a sheaf of $\Ocal_{\Sp(A)}$-modules on $\Sp(A)$ and $\Omega_{1},\Omega_{2}$ a rational cover of $\Sp(A)$ on which $\Mcal|_{\Omega_{i}}=M_{i}$ with $M_{i}$ finitely generated. There are $(\mu_{j})_{j=1}^{N_{1}}\in\Mcal(\Omega)$ such that the $\mu_{j}|_{\Omega_{1}}$ generate $\Mcal(\Omega_{1})$. 
\end{corollary}
\begin{proof}
    By \Cref{corr: sum decomposition smaller than epsilon}, $m_{j}^{(1)}|_{\Omega_{12}}=\mu_{j}^{1}|_{\Omega_{12}}+\mu_{j}^{(2)}|_{\Omega_{12}}$ where $\mu_{j}^{(k)}\in\Mcal(\Omega_{k})$ and $\Vert \mu_{j}^{(1)}|\Mcal(\Omega)\leq\varepsilon$ for any $\varepsilon$. Let 
    \begin{align*}
        \mu_{j}|_{\Omega_{1}} &= m_{j}^{(1)} - \mu_{j}^{(1)} \\
        \mu_{j}|_{\Omega_{2}} &= \mu_{j}^{(2)}
    \end{align*}
    then $\Vert \mu_{j}|_{\Omega_{1}}-m_{j}^{(1)}|\Mcal(\Omega_{j})\Vert\leq\varepsilon$ and when $\varepsilon=\frac{1}{2}$ the assertion follows. 
\end{proof}
It follows that the set 
$$\{g^{-k}\cdot m_{1}|_{\Omega_{12}}|m_{1}\in\Mcal(\Omega_{1})\}$$
is dense in $\Mcal(\Omega_{12})$, recalling here that $\Omega_{1}=R_{A}(1|g)$. 
\begin{corollary}\label{corr: fractional sum decomposition}\marginpar{Corollary 2.3}
    Let $\Mcal$ be a sheaf of $\Ocal_{\Sp(A)}$-modules on $\Sp(A)$ and $\Omega_{1}=R_{A}(1|g),\Omega_{2}=R_{A}(g|1)$ for $g\in A$ a rational cover of $\Sp(A)$ on which $\Mcal|_{\Omega_{i}}=M_{i}$ with $M_{i}$ finitely generated. If $m_{12}\in\Mcal(\Omega_{12})$ and $\varepsilon>0$ then 
    $$m_{12}=g^{-k}\cdot m_{1}|_{\Omega_{12}}+m_{2}|_{\Omega_{12}}$$
    where $m_{i}\in\Mcal(\Omega_{i})$ and $\Vert m_{2}|\Mcal(\Omega_{2})\Vert\leq\varepsilon$. 
\end{corollary}
\begin{proof}
    One need only repeat the arguments of \Cref{corr: sum decomposition smaller than epsilon}
\end{proof}
\begin{corollary}\label{corr: generate restriction to Omega2}\marginpar{Corollary 2.4}
    Let $\Mcal$ be a sheaf of $\Ocal_{\Sp(A)}$-modules on $\Sp(A)$ and $\Omega_{1}=R_{A}(1|g),\Omega_{2}=R_{A}(g|1)$ for $g\in A$ a rational cover of $\Sp(A)$ on which $\Mcal|_{\Omega_{i}}=M_{i}$ with $M_{i}$ finitely generated. There are $(\mu_{j}^{(2)})_{j=1}^{N_{2}}\in\Mcal(\Omega)$ such that $\mu_{j}|_{\Omega_{2}}$ generate $\Mcal(\Omega_{2})$. 
\end{corollary}
\begin{proof}
    Write 
    \begin{align*}
        m_{j}^{(2)}|_{\Omega_{12}} &= g^{-k}\mu_{j}^{(1)}|_{\Omega_{12}} + \mu_{j}^{(2)}|_{\Omega_{12}} \\
        \mu_{j}|_{\Omega_{2}}&=\mu_{j}^{(1)} \\
        \mu_{j}|_{\Omega_{2}}&= g^{k}(m_{j}^{(2)}-\mu_{j}^{(2)})
    \end{align*}
    and the assertion follows when $\varepsilon\leq\frac{1}{2}$ in which case $\Vert m_{j}^{(2)} - g^{k}\cdot \mu_{j}|\Mcal(\Omega_{2})\Vert\leq\frac{1}{2}$. 
\end{proof}
\section{Lecture 3 -- 9th May 2025}\label{sec: lecture 3}
In \Cref{sec: lecture 2}, we constructed the $q$-de Rham and $q$-Hodge complexes for $\A^{d}_{\ZZ},(\GG_{m})^{d}$ using the $q$-derivatives $\nabla_{i}^{q}$ and modifited $q$-derivatives $\widetilde{\nabla}_{i}^{q}$, respectively. 

We now consider the construction of the $q$-de Rham and $q$-Hodge complexes more generally in the case where the $T_{i}$ are invertible and using the logarithmic $q$-derivative $\nabla_{i}^{q,\log}=T_{i}\nabla_{i}^{q}$. We first define these in the case $R=\ZZ[q^{\pm}][\underline{T}^{\pm}]=\ZZ[q^{\pm}][T_{1}^{\pm},\dots,T_{d}^{\pm}]$. 
\begin{definition}[Logarithmic $q$-Derivative]\label{def: logarithmic q-derivative}
    Let $R=\ZZ[q^{\pm}][T_{1}^{\pm},\dots,T_{d}^{\pm}]$. The logarithmic $q$-derivative $\nabla_{i}^{q,\log}:R\to R$ for $1\leq i\leq d$ is defined by 
    $$\nabla_{i}^{q}(f)=\frac{\gamma_{i}(f)-f}{q-1}.$$
\end{definition}
\begin{definition}[Modified Logarithmic $q$-Derivative]\label{def: modified logarithmic q-derivative}
    Let $R=\ZZ[q^{\pm}][T_{1}^{\pm},\dots,T_{d}^{\pm}]$. The modified logarithmic $q$-derivative $\widetilde{\nabla}_{i}^{q,\log}:R\to R$ for $1\leq i\leq d$ is defined by 
    $$\widetilde{\nabla}_{i}^{q}(f)=\gamma_{i}(f)-f.$$
\end{definition}
\begin{remark}\label{rmk: commutation for log q-connections}
    The commutation relation for the ordinary logarithmic $q$-derivative are given by $\gamma_{i}T_{i}=qT_{i}\gamma_{i}$ since multiplying by $T_{i}$ and applying the map $T_{i}\mapsto qT_{i}$ is the same as applying the map $T_{i}\mapsto qT_{i}$ and multiplying by $qT_{i}$. 
\end{remark}
\begin{example}
    Using \Cref{rmk: commutation for log q-connections}, we deduce that category of logarithmic $q$-connections on $\GG_{m}$ are modules over the ring $\ZZ[q^{\pm}]\{T^{\pm},\gamma\}/(\gamma T-q T\gamma)$ (cf. \Cref{ex: A1 with Weyl algebra}). 
\end{example}

We undertake the task of constructing the $q$-de Rham and $q$-Hodge complexes for general smooth $\ZZ$-schemes $X$ locally admitting an \'{e}tale framing. For simplicity, we will restrict our attention to the case where $X=\spec(R)$ with $R$ a smooth $\ZZ$-algebra and $\square:X\to(\GG_{m})^{d}$ is \'{e}tale (equivalently, $\ZZ[T_{1}^{\pm},\dots,T_{d}^{\pm}]\to R$ \'{e}tale). 

If we were to mirror the constructions of \Cref{def: q-connections on modules,def: modified q-connections on modules}, we would want to produce $R[q^{\pm}]$-modules with commuting semilinear endomorphisms $\gamma_{i,M}:M\to M$ (used to produce $\nabla^{q}_{i,M},\widetilde{\nabla}_{i,M}^{q}$). This semilinearity ought be defined in terms of $\gamma_{i,R}:R[q^{\pm}]\to R[q^{\pm}]$ which extend $\gamma_{i}$ on $\ZZ[q^{\pm}][\underline{T}^{\pm}]$, but there is no reason such maps should exist. Put in other -- more geometric -- terms, the automorphisms $\gamma_{i}$ on $(\GG_{m})^{d}$ need not lift along the map $\square:X\to(\GG_{m})^{d}$.

Completion allows us to resolve this issue: after $(q-1)$-adic completion, there are unique such $\gamma_{i,R}:R[[q-1]]\to R[[q-1]]$ restricting to the identity modulo $(q-1)$. This is a consequence of the infintesmal lifting property for (formally) \'{e}tale maps \cite[\href{https://stacks.math.columbia.edu/tag/00UP}{Tag 00UP}]{stacks-project}: 
$$% https://q.uiver.app/#q=WzAsNCxbMCwxLCJSW1txLTFdXSJdLFsyLDAsIlJbW3EtMV1dIl0sWzAsMCwiXFxaWltbcS0xXV1bXFx1bmRlcmxpbmV7VH1ee1xccG19XSJdLFsyLDEsIlIiXSxbMSwzLCJcXHBtb2R7KHEtMSl9Il0sWzAsMywiXFxwbW9keyhxLTEpfSIsMl0sWzIsMCwiXFxzcXVhcmUiLDJdLFsyLDEsIlxcc3F1YXJlXFxjaXJjXFxnYW1tYV97aX0iXSxbMCwxLCJcXGV4aXN0cyFcXGdhbW1hX3tpLFJ9IiwxLHsic3R5bGUiOnsiYm9keSI6eyJuYW1lIjoiZG90dGVkIn19fV1d
\begin{tikzcd}
	{\ZZ[[q-1]][\underline{T}^{\pm}]} && {R[[q-1]]} \\
	{R[[q-1]]} && R.
	\arrow["{\square\circ\gamma_{i}}", from=1-1, to=1-3]
	\arrow["\square"', from=1-1, to=2-1]
	\arrow["{\pmod{(q-1)}}", from=1-3, to=2-3]
	\arrow["{\exists!\gamma_{i,R}}"{description}, dotted, from=2-1, to=1-3]
	\arrow["{\pmod{(q-1)}}"', from=2-1, to=2-3]
\end{tikzcd}$$
More formally, $\square:\ZZ[\underline{T}^{\pm}]\to R$ is \'{e}tale, and \'{e}taleness is preserved under base change, so $\square:\ZZ[[q-1]][\underline{T}^{\pm}]\to R[[q-1]]$ is \'{e}tale and $R[[q-1]]\to R$ is an infinitesmal thickening, so the desired lift exists rendering the entire diagram commutative. Geometrically, $(q-1)$-adic completion the automorphisms $\gamma_{i}$ on $(\GG_{m})^{d}$ are infinitesmally close to the identity, hence lift uniquely along the framing map (that is, the framing map on schemes $\square:\spec(R[[q-1]])\to\ZZ[[q-1]][\underline{T}^{\pm}]$). This allows us to define (modified/logarithmic) $q$-derivatives and the notion of modules with (modified/logarithmic) $q$-connection. This notion is illustrated in the following equivalence of categories. 
\begin{lemma}\label{lem: equivalence of categories}
    There is an equivalence of categories 
    $$\left\{\substack{\text{\'{e}tale }\ZZ[q^{\pm}][\underline{T}^{\pm}]/(q-1)^{n} \\ \text{algebras}}\right\}\simeq\left\{\substack{\text{\'{e}tale }\ZZ[\underline{T}^{\pm}] \\ \text{algebras}}\right\}.$$
\end{lemma}
\begin{proof}
    See \cite[\href{https://stacks.math.columbia.edu/tag/039R}{Tag 039R}]{stacks-project}. 
\end{proof}
\begin{theorem}[Bhatt-Scholze, {\cite[\S 16]{PrismsPrismatic}}; Wagner, {\cite[Thm. 1.5]{WagnerQWittQHodge}}]
    Let $(R,\square)$ be a smooth framed $\ZZ$-algebra. The complex $q\Omega_{(R,\square)/\ZZ[[q-1]]}$ given by 
    $$R[[q-1]]\xrightarrow{(\nabla_{i}^{q})_{i=1}^{d}}\bigoplus_{i=1}^{d}R[[q-1]]\longrightarrow\dots$$
    as an object of $\Dscr(\ZZ[[q-1]])$ is canonically independent of the choice of coordinates. 
\end{theorem}
Such coordinate independence is somewhat easy to deduce in the case where $R$ is a $\QQ$-algebra. 
\begin{example}[{\cite[Lem. 4.1]{qDeformations};\cite[Lem. 12.4]{BMS1}}]\label{ex: translation between connections and q-connections}
    Consider the case of a smooth framed $\QQ$-algebra $(R,\square)$ where $\square:\spec(R)\to\GG_{m}$. We can use Taylor's theorem to write 
    $$f(qT)=f(T)+\log(q)(\nabla^{\log}f)(T)+\frac{1}{2}\log(q)^{2}((\nabla^{\log})^{2}f)(T)+\dots$$
    where $\log(q)=\sum_{n\geq0}(-1)^{n-1}\frac{(q-1)^{n}}{n}\in\QQ[[q-1]]$ so taking the difference of $f(qT)$ and $f(T)$, we find the operators $\nabla^{q,\log},\widetilde{\nabla}^{q,\log}$ are given by 
    \begin{align*}
        \nabla^{q,\log} &= \frac{\log(q)}{(q-1)}(\nabla^{\log}f)(T)+\frac{1}{2}\frac{\log(q)^{2}}{(q-1)}((\nabla^{\log})^{2}f)(T)+\dots\\
        \widetilde{\nabla}^{q,\log} &= \log(q)(\nabla^{\log}f)(T)+\frac{1}{2}\log(q)^{2}((\nabla^{\log})^{2}f)(T)+\dots.
    \end{align*}
    Using that $\widetilde{\nabla}^{\log}=\log(q)\nabla^{\log}$ we get 
    $$\widetilde{\nabla}^{q,\log}=\widetilde{\nabla}^{\log}+\frac{1}{2}(\widetilde{\nabla}^{\log})^{2}+\dots$$
    we get that $\widetilde{\nabla}^{q,\log}=\exp(\widetilde{\nabla}^{\log})+1$. In particular, for smooth framed $\QQ$-algebras, the data of modified logarithmic $q$-connections are equivalent to modified logarithmic connections up to a transformation, and allow us to interpolate between the two structures. 
\end{example}
\Cref{ex: translation between connections and q-connections} yields the following more general result. 
\begin{proposition}\label{prop: R-modules with ordinary connection}
    Let $(R,\square)$ be a smooth framed $\QQ$-algebra. There is an symmetric monoidal equivalence of categories 
    $$\left\{\substack{(q-1)\text{-adically complete }R[[q-1]] \\ \text{-modules with }q\text{-connection}}\right\}\simeq\left\{\substack{(q-1)\text{-adically complete }R[[q-1]] \\ \text{-modules with connection}}\right\}.$$
    Moreover, these categories are independent of choice of coordinates on $R[[q-1]]$. 
\end{proposition}
\begin{proof}[Proof Outline]
    The computation of \Cref{ex: translation between connections and q-connections} in several variables (cf. \cite[Lem. 4.1]{qDeformations}) shows an equivalence of data between modified logarithmic $q$-connections and modified logarithmic connections, and since we are working over $\QQ$ and the torus, these are the same as ordinary ($q$-)connections. Thus for a fixed $(q-1)$-adically complete $R[[q-1]]$-module $M$ with $q$-connection, there is a unique ordinary connection with which it can be endowed, and conversely.
    
    The latter statement follows from the observation that the latter category of $(q-1)$-adically complete $R[[q-1]]$-modules with connection are visibly coordinate independent. 
\end{proof}
As in the case of $(\GG_{m})^{d}$ in \Cref{prop: algebra structures on RHom unit Ad and Gm} (i), we have in this case the following result. 
\begin{corollary}\label{corr: modules with q-connection are coordinate independent}
    Let $(R,\square)$ be a framed $\QQ$-algebra and denote the category of $(q-1)$-adically complete $R[[q-1]]$-modules with $q$-connection by $q\Mod_{R[[q-1]]}$. The $q$-de Rham complex $q\Omega_{(R,\square)/\QQ}$ computes $R\Hom_{q\Mod_{R[[q-1]]}}(\mathbbm{1},\mathbbm{1})$ and is canonically independent of coordinates. 
\end{corollary}
\todo{Semi-final up to here.}
The case of modified $q$-connections is more subtle as the convergence of the logarithm becomes problematic. 
\begin{definition}[Logarithmic $q$-Connections]\label{def: logarithmic q-connection}
    Let $R$ be a $\QQ$-algebra. A $h$-connection over $R[h]$ is an $R[h]$-module $M$ with a map $\widetilde{\nabla}_{M}:M\to M\otimes_{R}\Omega^{1}_{R/\QQ}$ satisfying $(\widetilde{\nabla}_{M})^{2}=0$ and 
    $$\widetilde{\nabla}_{M}(fm)=h\cdot\nabla(f)\cdot m+f\cdot\widetilde{\nabla}_{M}(m).$$
\end{definition}
Such constructions appear in Hodge and twistor theory under the name of Higgs fields. 

We would like to see an analogue of \Cref{prop: R-modules with ordinary connection}. 
\begin{proposition}\label{prop: R-modules with modified connection}
    Let $(R,\square)$ be a smooth framed $\QQ$-algebra. There is an symmetric monoidal equivalence of categories 
    $$\left\{\substack{(q-1)\text{-adically complete }R[[q-1]]\text{-modules} \\ \text{with modified }q\text{-connection s.t. }\widetilde{\nabla}^{q,\log}_{i,M}\text{'s are top. nil.}}\right\}\simeq\left\{\substack{h\text{-adically complete }R[[q-1]]\text{-modules} \\ \text{with }h\text{-connection s.t. }\widetilde{\nabla}_{M}\text{ is top. nil.}}\right\}$$
    $$% https://q.uiver.app/#q=WzAsMixbMiwwLCJcXGxlZnQoTSwoXFx3aWRldGlsZGV7XFxuYWJsYX1fe2ksTX1ee1xcbG9nfSlfe2k9MX1ee2R9XFxyaWdodCkiXSxbMCwwLCJcXGxlZnQoTSwoXFx3aWRldGlsZGV7XFxuYWJsYX1ee3EsXFxsb2d9X3tpLE19KV97aT0xfV57ZH1cXHJpZ2h0KSJdLFswLDEsIiIsMCx7InN0eWxlIjp7InRhaWwiOnsibmFtZSI6Im1hcHMgdG8ifX19XV0=
    \begin{tikzcd}
        {\left(M,(\widetilde{\nabla}^{q,\log}_{i,M})_{i=1}^{d}\right)} && {\left(M,(\widetilde{\nabla}_{i,M}^{\log})_{i=1}^{d}\right)}
        \arrow[maps to, from=1-3, to=1-1]
    \end{tikzcd}$$
    where $\widetilde{\nabla}^{q,\log}_{i,M}=\exp(\widetilde{\nabla}^{\log}_{i,M})-1$. 
\end{proposition}
\begin{remark}
    The topological nilpotence of the endomorphisms ensure covergence of the exponential. 
\end{remark}
Once again, observing that the right hand side is coordinate independent, we get coordinate independence for modules with modified $q$-connections. 
\begin{corollary}\label{corr: modules with modified q-connection are coordinate independent}
    Let $(R,\square)$ be a framed $\QQ$-algebra and denote the category of $(q-1)$-adically complete $R[[q-1]]$-modules with modified $q$-connection where the operators $\widetilde{\nabla}_{i,M}^{q,\log}$ are topologically nilpotent by $q\widetilde{\Mod}_{R[[q-1]]}$. The $q$-Hodge complex $q\Hdg_{(R,\square)/\QQ}$ computes $R\Hom_{q\widetilde{\Mod}_{R[[q-1]]}}(\mathbbm{1},\mathbbm{1})$ and is canonically independent of coordinates. 
\end{corollary}
Deferring the discussion of coordinate independence integrally -- which can be done by similarly isolating subcategories of modules with convergence conditions on their $q$-connections -- we seek to understand the preceding constructions not just in the $(q-1)$-adically complete case to the Habiro case, namely at all roots of unity. 

In the preceding discussion, $(q-1)$-adic completion allowed us to leverage \'{e}taleness of the map to produce a unique lift of the endomorphism on $\ZZ[q^{\pm}][\underline{T}^{\pm}]$ since $\gamma_{i}$ was infinitesmally close to the identity after $(q-1)$-adic completion. But noticing that $\zeta_{p}$ is $p$-adically close to 1, we can attempt a similar approach. 
\begin{example}
    Let $(R,\square)$ be a framed $\ZZ$-algebra with $\square:\spec(R)\to\GG_{m}$. This gives a map 
    $$\ZZ[T^{\pm}]_{p}^{\wedge}[[q-1]]\longrightarrow R_{p}^{\wedge}[[q-1]]$$
    which on specialization to $q=\zeta_{p}$ yields 
    $$\ZZ_{p}[\zeta_{p}]\langle T^{\pm}\rangle\longrightarrow R_{p}^{\wedge}[\zeta_{p}]$$
    where using that $\zeta_{p}$ is close to 1 $p$-adically, $\gamma:\ZZ_{p}[\zeta_{p}]\langle T^{\pm}\rangle\to\ZZ_{p}[\zeta_{p}]\langle T^{\pm}\rangle$ by $T\mapsto qT$ lifts uniquely to an endomorphism $\gamma_{R}:R_{p}^{\wedge}[\zeta_{p}]\to R_{p}^{\wedge}[\zeta_{p}]$. However, $R[\zeta_{p}]\hookrightarrow R^{\wedge}_{p}[\zeta_{p}]$ may not have image stable under $\gamma_{R}$, for example, in the case of $\GG_{m}\setminus\{1\}$.  
\end{example}
So as seen in the example above, we will require an alternative description. For this, we produce an endomorphism of $R_{p}^{\wedge}[\zeta_{p}]$ that does globalize using the Frobenius map $\varphi:\ZZ[T^{\pm}]\to\ZZ[T^{\pm}]$ by $T\mapsto T^{p}$ lifts uniquely to $R^{\wedge}_{p}$ and reduces to the Frobenius map on $R/(p)$. This produces an isomorphism
$$\ZZ[T^{\pm 1/p}]\otimes_{\ZZ[T^{\pm}]}R_{p}^{\wedge}\longrightarrow R_{p}^{\wedge}$$ by $T^{1/p}\mapsto T$ and thus 
$$\ZZ[\zeta_{p},T^{\pm 1/p}]\otimes_{\ZZ[T^{\pm}]}R^{\wedge}_{p}\longrightarrow R_{p}^{\wedge}[\zeta_{p}]$$
by the Frobenius once more. The map $\gamma:R_{p}^{\wedge}[\zeta_{p}]\to R_{p}^{\wedge}[\zeta_{p}]$ is induced by the map $\id_{R^{\wedge}_{p}}\otimes(T^{1/p}\mapsto \zeta_{p}T^{1/p})$ but $\ZZ[\zeta_{p},T^{\pm 1/p}]\otimes_{\ZZ[T^{\pm}]}R^{\wedge}_{p}$ contains $R\otimes_{\ZZ[T^{\pm}]}\ZZ[\zeta_{p},T^{\pm 1/p}]$ as a subring, and since $\gamma$ is the identity on $R$, $\gamma$ is stable as an endomorphism. 
\begin{definition}
    Let $m\geq 1$. $\ZZ[\zeta_{m},\underline{T}^{\pm}]$-algebra $R^{(m)}=R\otimes_{\ZZ[\underline{T}^{\pm}]}\ZZ[\zeta_{m},\underline{T}^{\pm 1/m}]$ with algebra structure given by $T_{i}\mapsto T_{i}^{1/m}$ with action by $\gamma^{(m)}_{i}=\id_{R}\otimes(T_{i}\mapsto \zeta_{m}T_{i})$ lifting $T_{i}\mapsto\zeta_{m}T_{i}$ on $\ZZ[\zeta_{m},\underline{T}^{\pm}]$. 
\end{definition}
\begin{example}
    Let $X=\GG_{m}\setminus\{1\}$ and $R=\ZZ[T^{\pm},\frac{1}{1-T}]$. Then $R^{(m)}=\ZZ[T^{\pm},\frac{1}{1-T^{m}}]$ with the structure of a $\ZZ[\zeta_{m},T^{\pm}]$-algebra by $T\mapsto T^{1/m}$.
\end{example}
By uniqueness of deformation for \'{e}tale algebras, we can deform from $q=\zeta_{m}$ to the completion at $\Phi_{m}(q)$, the $m$th cyclotomic polynomial, yielding a (formally) \'{e}tale $\ZZ[q,\underline{T}^{\pm}]^{\wedge}_{\Phi_{m}(q)}$-algebra $R_{m}$ with lifts $\gamma_{i,m}:R_{m}\to R_{m}$. So for any $m$ we can define the categories and complexes as before. While only defined at each $m$ separately, we can use the fact that $\zeta_{m}$ and $\zeta_{pm}$ agree in characteristic $p$ to glue the construction globally using the Frobenius, yielding a complex over the Habiro ring $\Hcal_{(R,\square)}$. 
\section{Lecture 4 -- 23rd May 2025}\label{sec: lecture 4}
Using the gluing procedure of (\ref{eqn: gluing map Frobenius}) gives a procedure to correcting of the overspecification of prescribing a local algebra $R^{(m)}$ for each positive integer $m$ in characteristic $p$ -- that is, gluing $R^{(m)},R^{(m')}$ where $m_{0}$ is coprime to $p$ and $m=m_{0}p^{a},m'=m_{0}p^{b}$ using the Frobenius. 
\begin{proposition}\label{prop: explicit elements of HR}
    Let $(R,\square)$ be a smooth framed $\ZZ$-algebra. The Habiro ring $\Hcal_{(R,\square)}$ of $(R,\square)$ is given by 
    {\footnotesize
    \begin{equation}\label{eqn: Habiro ring of framed algebra}
        \Hcal_{(R,\square)}=\left\{(f_{m})_{m\geq 1}\in\prod_{m\geq 1}R^{(m)}[[q-\zeta_{m}]]:\substack{\forall m\in\NN,\text{ }\forall p\text{ prime} \\\varphi_{p}(f_{pm})=f_{m}\in (R^{(m)})_{p}^{\wedge}[[q-\zeta_{m}]]\cong (R^{(pm)})_{p}^{\wedge}[[q-\zeta_{pm}]]}\right\}
    \end{equation}
    \normalsize}where $\varphi_{p}$ lifts the Frobenius on $R^{(m)}/(p)$ by raising each variable to the $p$-th power and fixes $q$ and $\zeta_{m}$. 
\end{proposition}
\begin{remark}
    There is an obvious map from the Habiro ring of the torus \Cref{def: Habiro ring of base} $\Hcal_{\ZZ[\underline{T}^{\pm}]}\to\Hcal_{(R,\square)}$ endowing the Habiro ring of $(R,\square)$ with the structure of a $\Hcal_{\ZZ[\underline{T}^{\pm}]}$-algebra.  
\end{remark}
Let us consider some explicit elements of the Habiro ring. 
\begin{example}\label{ex: element of Habiro ring}\marginpar{The lecture contained a fairly substantive sketch of the proof \Cref{ex: element of Habiro ring}, which the author has defered to \Cref{appdx: explicit elements} for continuity of exposition.}
    Let $R=\ZZ[T_{1},\dots,T_{d},\frac{1}{1-T_{1}-\dots-T_{d}}]$ with framing $\square:\ZZ[T_{1},\dots,T_{d}]\to R$. The element 
    $$\sum_{k_{1},\dots,k_{d}\geq0}\left[\substack{k_{1}+\dots+k_{d} \\ k_{1}\text{ }\dots\text{ }k_{d}}\right]_{q}T_{1}^{k_{1}}\dots T_{d}^{k_{d}}\in\ZZ[q][[\underline{T}]]$$
    is an element of the Habiro ring $\Hcal_{(R,\square)}$ where 
    $$\left[\substack{k_{1}+\dots+k_{d} \\ k_{1}\text{ }\dots\text{ }k_{d}}\right]_{q}=\frac{(q;q)_{k_{1}+\dots+k_{d}}}{(q;q)_{k_{1}}\dots(q;q)_{k_{d}}}$$
    is the $q$-deformation of the multinomial $\binom{k_{1}+\dots+k_{d}}{k_{1}\dots k_{d}}$. More generally, explicit elements of the Habiro ring can be constructed by considering $q$-deformations of rational functions (vis. \Cref{ex: legendre family} and surrounding discussion). 
\end{example}
Returning to a discussion of Habiro cohomology of a smooth $\ZZ$-algebra with framing $\square:\spec(R)\to(\GG_{m})^{d}$, we recall that there are lifts of the automorphism $\gamma_{i}$ to $\Hcal_{(R,\square)}$: more explicitly, for a section $(f_{m})_{m\geq0}$, the action $\gamma_{i}$ acts by $(f_{m})_{m\geq1}\mapsto (\gamma_{i}^{(m)}(f_{m}))_{m\geq1}$ where $\gamma_{i}^{(m)}$ is the automorphism given in \Cref{def: root of unity algebra}. This produces a $\ZZ^{d}$-action on $\Hcal_{(R,\square)}$, and we can define Habiro-Hodge cohomology to be the group cohomology of the action of $\ZZ^{d}$ on $\Hcal_{(R,\square)}$. 
\begin{definition}[$q$-Habiro-Hodge Cohomology]\label{def: q-Habiro-Hodge cohomology}
    Let $(R,\square)$ be a smooth framed $\ZZ$-algebra. The $q$-Habiro-Hodge cohomology is the cohomology of the complex $q\dash\HHdg_{(R,\square)}$ given by 
    $$\Hcal_{(R,\square)}\xrightarrow{(\gamma_{i}-1)_{i=1}^{d}}\bigoplus_{i=1}^{d}\Hcal_{(R,\square)}\longrightarrow\dots.$$
\end{definition}
For this to be functorial, we would expect this to be coordinate independent, at least at the level of derived categories. As a first step, we study the cohomology of the complex modulo $(1-q^{m})$ -- that is, at specalizations to roots of unity. 

If $m=1$, then $\Hcal_{(R,\square)}/(1-q)\cong R$ and all differentials are zero, so 
$$H^{i}\left(q\dash\HHdg_{(R,\square)}/(1-q)\right)\cong R^{\oplus\binom{d}{i}}\cong \Omega^{i}_{R/\ZZ}$$
and is therefore independent of coordinates since the middle term is so. 
\begin{remark}
    While \emph{a priori} we only have a isomorphism to a free module of a certain rank, there is additonal structure that allows us to identify this with the module of K\"{a}hler differentials: the Bockstein map associated to the triangle 
    $$q\dash\HHdg_{(R,\square)}/(1-q)\xrightarrow{\times(1-q)}q\dash\HHdg_{(R,\square)}/(1-q)^{2}\longrightarrow q\dash\HHdg_{(R,\square)}/(1-q)$$
    $$\hspace{9cm}\longrightarrow \left(q\dash\HHdg_{(R,\square)}/(1-q)\right)[1]$$
    inducing 
    $$H^{i}\left(q\dash\HHdg_{(R,\square)}/(1-q)\right)\longrightarrow H^{i+1}\left(q\dash\HHdg_{(R,\square)}/(1-q)\right)$$
    which gives a derivation 
    $$H^{0}\left(q\dash\HHdg_{(R,\square)}/(1-q)\right)\longrightarrow H^{1}\left(q\dash\HHdg_{(R,\square)}/(1-q)\right)$$
    and hence an isomorphism $H^{1}\left(q\dash\HHdg_{(R,\square)}/(1-q)\right)\to\Omega^{1}_{R/\ZZ}$. In addition, the ring structure on cohomology induces the structure of a commutative differential graded algebra on $H^{\bullet}\left(q\dash\HHdg_{(R,\square)}/(1-q)\right)$ and this structure is in fact independent of coordinates on the nose and not just up to quasi-isomorphism.
\end{remark}

For general $m$, $H^{\bullet}\left(q\dash\HHdg_{(R,\square)}/(1-q^{m})\right)$ has the strucuture of a commutative differential graded algebra that is coordinate independent. 
\begin{theorem}[Wagner; {\cite[Prop. 5.7]{WagnerMSThesis}}]\label{thm: surjection from q witt vectors}
    Let $R$ be a smooth framed $\ZZ$-algebra. There is a canonical surjection 
    $$W_{m}(R)[q]/(1-q^{m})\longrightarrow H^{0}\left(q\dash\HHdg_{(R,\square)}/(1-q^{m})\right)$$
    inducing 
    $$\Omega_{W_{m}(R)[q]/(1-q^{m})}\longrightarrow H^{\bullet}\left(q\dash\HHdg_{(R,\square)}/(1-q)\right)$$ 
    which is coordinate independent, degreewise surjective, and with kernel independent of coordinates. 
\end{theorem}
\begin{proof}[Proof Outline]
    For every commutative differential graded algebra $B$ receiving a map from a commutative ring $A$ in 0th cohomology, there is an induced map from the initial commutative differential graded algebra generated by $A$ to $B$ -- the latter being the de Rham complex. 
\end{proof}
This produces a description of $H^{i}\left(q\dash\HHdg_{(R,\square)}/(1-q)\right)$ that is visibly independent of coordinates, being the quotient of coordinate-independent objects. 

In fact we can do better. For any $R$, there is a notion of $q$-Witt vectors $q\dash W_{m}(R)$ and $q$-de Rham-Witt complexes $q\dash W_{m}\Omega_{R}$ which is a commutative differential graded algebra with first term $q\dash W_{m}(R)$ isomorphic to $H^{\bullet}\left(q\dash\HHdg_{(R,\square)}/(1-q^{m})\right)$. 
\begin{theorem}[Wagner; {\cite[Thm. 5.7]{WagnerMSThesis}}]\label{thm: }
    Let $R$ be a smooth framed $\ZZ$-algebra. There is an isomorphism 
    $$q\dash W_{m}\Omega_{R}^{\bullet}\longrightarrow H^{\bullet}\left(q\dash\HHdg_{(R,\square)}/(1-q)\right).$$
\end{theorem}
\begin{remark}
    This is related to the classical construction of the de Rham-Witt complex, though the sense in which the preceding constructions are $q$-deformations are quite subtle. 
\end{remark}
\begin{remark}
    One can often reduce to the case of computing on the torus, since many of the constructions ``commute with \'{e}tale maps'' in the sense that they are preserved under \'{e}tale base change. 
\end{remark}
Based on this, one might hope that these complexes are independent of coordinates. 
\begin{example}\label{ex: q-Habiro-Hodge cohomology of torus}
    Let $R=\ZZ[T^{\pm}]$. The $q$-Habiro-Hodge complex is given by 
    $$\ZZ[q][T^{\pm}]/(1-q^{m})\xrightarrow{\gamma-1}\ZZ[q][T^{\pm}]/(1-q^{m})$$
    by $T^{k}\mapsto(q^{k}-1)T^{k}$. We can compute the kernel of this map -- the 0th cohomology -- by noting that the map preserves the degree of $T$, we can compute the kernel in each degree to see that it is given by 
    $$\bigoplus_{k\in\ZZ}\left(\frac{\frac{q^{m}-1}{q^{\gcd(k,m)}-1}\ZZ[q]}{(q^{m}-1)\ZZ[q]}\right)T^{k}\cong\bigoplus_{k\in\ZZ}\left(\ZZ[q]/(1-q^{\gcd(k,m)})\ZZ[q]\right)T^{k}.$$
    We similarly compute first cohomology to see it is also given by 
    $$\bigoplus_{k\in\ZZ}\left(\ZZ[q]/(1-q^{\gcd(k,m)})\ZZ[q]\right)T^{k}.$$
    Indeed, when $m=p$ is prime, the 0th cohomology is a subring of $\ZZ[q][T^{\pm}]/(1-q^{p})$ (hence a subring of $\ZZ[T^{\pm}]\times\ZZ[\zeta_{p}][T^{\pm p}]\subseteq\ZZ[T^{\pm}]\times\ZZ[\zeta_{p}][T^{\pm}]$) and is generated by $T^{p}$ and $[p]_{q}T^{i}$ for $1\leq i\leq p-1$. 
\end{example}
The computations of \Cref{ex: q-Habiro-Hodge cohomology of torus} is suggestive of a connection to Witt vectors since the cohomology lies in the product of rings $\ZZ[T^{\pm}]\times\ZZ[\zeta_{p}][T^{\pm p}]$. Recall that for a $p$-torsion free ring $R$, the $p$-th Witt vectors $W_{p}(R)$ consists of elements $(x_{0},x_{1},\dots)$ has ghost maps $\gh_{1},\gh_{p}:W_{p}(R)\to R$ by $(x_{0},x_{1},\dots)\mapsto x_{0}$ and $(x_{0},x_{1},\dots)\mapsto x_{0}^{p}+px_{1}$, respectively. The image of $(\gh_{1},\gh_{p}):W_{p}(R)\to R\times R$ consists precisely of those pairs $(x,y)\in R\times R$ where $y\equiv x^{p}\pmod{p}$. 
\begin{proposition}[Wagner]\label{prop: q-Witt vectors}
    Let $R=\ZZ[T^{\pm}]$ with the identity framing and $q\dash\HHdg_{(R,\square)}$ its $q$-Habiro-Hodge complex. There is a canonical embedding 
    $$W_{p}(R)\hookrightarrow H^{0}\left(q\dash\HHdg_{(R,\square)}/(1-q^{p})\right)$$
    rendering the diagram 
    {\footnotesize
    $$% https://q.uiver.app/#q=WzAsNyxbMSwxLCJIXnswfVxcbGVmdChxXFxkYXNoXFxISGRnX3soUixcXHNxdWFyZSl9LygxLXFee3B9KVxccmlnaHQpIl0sWzEsMiwiV197cH0oUikiXSxbMywxLCJcXFpaW1Ree1xccG19XVxcdGltZXNcXFpaW1xcemV0YV97cH1dW1Ree1xccG0gcH1dIl0sWzMsMiwiUlxcdGltZXMgUiJdLFswLDMsIih4X3swfSx4X3sxfSkiXSxbNCwzLCIoeF97MH0seF97MH1ee3B9K3B4X3sxfSkiXSxbMCwwLCJcXHZhcnBoaV97cH0oeF97MH0pK1twXV97cX14X3sxfSJdLFsxLDAsIiIsMCx7InN0eWxlIjp7InRhaWwiOnsibmFtZSI6Imhvb2siLCJzaWRlIjoidG9wIn19fV0sWzEsMywiKFxcZ2hfezF9LFxcZ2hfe3B9KSIsMl0sWzMsMl0sWzAsMiwiIiwwLHsic3R5bGUiOnsidGFpbCI6eyJuYW1lIjoiaG9vayIsInNpZGUiOiJ0b3AifX19XSxbNCw1LCIiLDIseyJzdHlsZSI6eyJ0YWlsIjp7Im5hbWUiOiJtYXBzIHRvIn19fV0sWzQsNiwiIiwwLHsic3R5bGUiOnsidGFpbCI6eyJuYW1lIjoibWFwcyB0byJ9fX1dXQ==
    \begin{tikzcd}
        {\varphi_{p}(x_{0})+[p]_{q}x_{1}} \\
        & {H^{0}\left(q\dash\HHdg_{(R,\square)}/(1-q^{p})\right)} && {\ZZ[T^{\pm}]\times\ZZ[\zeta_{p}][T^{\pm p}]} \\
        & {W_{p}(R)} && {R\times R} \\
        {(x_{0},x_{1})} &&&& {(x_{0},x_{0}^{p}+px_{1})}
        \arrow[hook, from=2-2, to=2-4]
        \arrow[hook, from=3-2, to=2-2]
        \arrow["{(\gh_{1},\gh_{p})}"', from=3-2, to=3-4]
        \arrow[from=3-4, to=2-4]
        \arrow[maps to, from=4-1, to=1-1]
        \arrow[maps to, from=4-1, to=4-5]
    \end{tikzcd}$$
    \normalsize}commutative. 
\end{proposition}
\begin{remark}
    On the $q$-Habiro-Hodge cohomologies, we can relate the different specializations by Frobenii and Verschiebungen 
    $$% https://q.uiver.app/#q=WzAsMixbMCwwLCJIXntpfVxcbGVmdChxXFxkYXNoXFxISGRnX3soUixcXHNxdWFyZSl9LygxLXFee21rfSlcXHJpZ2h0KSJdLFsyLDAsIkhee2l9XFxsZWZ0KHFcXGRhc2hcXEhIZGdfeyhSLFxcc3F1YXJlKX0vKDEtcV57bX0pXFxyaWdodCkuIl0sWzAsMSwiRl97a30iLDAseyJvZmZzZXQiOi0xfV0sWzEsMCwiVl97a309XFx0aW1lc1xcZnJhY3sxLXFee21rfX17MS1xXnttfX0iLDAseyJvZmZzZXQiOi0xfV1d
    \begin{tikzcd}
        {H^{i}\left(q\dash\HHdg_{(R,\square)}/(1-q^{mk})\right)} && {H^{i}\left(q\dash\HHdg_{(R,\square)}/(1-q^{m})\right).}
        \arrow["{F_{k}}", shift left, from=1-1, to=1-3]
        \arrow["{V_{k}=\times\frac{1-q^{mk}}{1-q^{m}}}", shift left, from=1-3, to=1-1]
    \end{tikzcd}$$
\end{remark}
More generally, we have the following. 
\begin{proposition}
    Let $R$ be a flat $\ZZ$-algebra. There is a commutative diagram 
    $$% https://q.uiver.app/#q=WzAsNSxbMiwwLCJXX3ttfShSKSJdLFs0LDAsIlxccHJvZF97ZHxtfVIiXSxbNCwxLCJcXHByb2Rfe2R8bX1SW1xcemV0YV97ZH1dIl0sWzIsMSwicVxcZGFzaCBXX3ttfShSKSJdLFswLDEsIldfe219KFIpW3FdLygxLXFee219KSJdLFs0LDMsIiIsMCx7InN0eWxlIjp7ImhlYWQiOnsibmFtZSI6ImVwaSJ9fX1dLFswLDRdLFswLDMsIiIsMix7InN0eWxlIjp7InRhaWwiOnsibmFtZSI6Imhvb2siLCJzaWRlIjoidG9wIn19fV0sWzAsMSwiKFxcZ2hfe2R9KV97ZHxtfSIsMCx7InN0eWxlIjp7InRhaWwiOnsibmFtZSI6Imhvb2siLCJzaWRlIjoidG9wIn19fV0sWzMsMiwiKHFcXGRhc2hcXGdoX3tkfSlfe2R8bX0iLDIseyJzdHlsZSI6eyJ0YWlsIjp7Im5hbWUiOiJob29rIiwic2lkZSI6InRvcCJ9fX1dLFsxLDIsIiIsMSx7InN0eWxlIjp7InRhaWwiOnsibmFtZSI6Imhvb2siLCJzaWRlIjoidG9wIn19fV1d
    \begin{tikzcd}
        && {W_{m}(R)} && {\prod_{d|m}R} \\
        {W_{m}(R)[q]/(1-q^{m})} && {q\dash W_{m}(R)} && {\prod_{d|m}R[\zeta_{d}]}
        \arrow["{(\gh_{d})_{d|m}}", hook, from=1-3, to=1-5]
        \arrow[from=1-3, to=2-1]
        \arrow[hook, from=1-3, to=2-3]
        \arrow[hook, from=1-5, to=2-5]
        \arrow[two heads, from=2-1, to=2-3]
        \arrow["{(q\dash\gh_{d})_{d|m}}"', hook, from=2-3, to=2-5]
    \end{tikzcd}$$
    where the Frobenii and Verschiebungen are defined on $q\dash W_{m}(R)$. 
\end{proposition}
\begin{remark}
    There are no restriction maps on the $q$-Witt vectors $q\dash W_{m}(R)$. 
\end{remark}
This shows that on the level of cohomology, the $q$-Habiro-Hodge complex is coordinate independent after specialization. However, due to a theorem of Wagner, this is the best we can do: there is no way to make the $q$-Habiro-Hodge complex itself coordinate independent in the derived category in such a way that remains coordinate independent on specialization. 

\section{Lecture 5 -- 22nd May 2025}\label{sec: lecture 5}
We make preparations towards showing the result of van der Put-Schneider and Huber \cite[Thm. 4]{vdPclassification} showing that the space of van der Put points on $\Sp(A)$ for $A$ affinoid is homeomorphic to the adic spectrum $\Spa(A,A^{\circ})$ of the correct Krull dimension. 
\begin{proposition}
    Let $A\to B$ be a morphism of affinoid $K$-algebras. 
    \begin{enumerate}[label=(\roman*)]
        \item If $B$ is finite over $A$, then $B^{\circ}$ is integral -- but not necessarily finite -- over $A^{\circ}$. 
        \item $\widetilde{A}=A^{\circ}/A^{\circ\circ}$ is finite type over $K^{\circ}/K^{\circ\circ}$ with Krull dimension equal to that of $A$. Moreover, $\widetilde{B}$ is finite over $\widetilde{A}$. 
    \end{enumerate}
\end{proposition}
\begin{example}
    If $A=\TT_{n}$, then $\widetilde{A}=A^{\circ}/A^{\circ\circ}\cong K[X_{1},\dots,X_{n}]$ the ordinary polynomial ring in $n$ variables. 
\end{example}
\begin{proposition}\label{prop: powerbounded commutes with polynomial algebra}
    Let $A$ be a nat ring. Then $A\langle X_{1},\dots,X_{n}\rangle^{\circ}\cong A^{\circ}\langle X_{1},\dots,X_{n}\rangle$. 
\end{proposition}
\begin{lemma}\label{lem: key lemma vdP points}
    Let $A$ be an affinoid $K$-algebra, $\Omega=R_{\Sp(A)}(f_{0}|f_{1},\dots,f_{n})$, and $B=\Ocal_{\Sp(A)}(\Omega)$. Consider the following conditions:\marginpar{Lemma 3.1}
    \begin{enumerate}[label=(\alph*)]
        \item $\nu\in R_{\Spa(A,A^{\circ})}(f_{0}|f_{1},\dots,f_{n})$. 
        \item $\nu$ can be extended to an element of $\Spa(B,B^{\circ})$. 
        \item $\nu$ admits a unique continuous extension to $B$. 
    \end{enumerate}
    then the following implications hold:
    \begin{enumerate}[label=(\roman*)]
        \item (a) and (b) are equivalent. 
        \item (a) and (b) imply (c). 
    \end{enumerate}
\end{lemma}
The statement of \Cref{lem: key lemma vdP points} reduces to showing this for Weierstrass domains (where $f_{0}=1$) and Laurent domains (where $f_{1}=1$). 
\begin{lemma}\label{lem: key vdP Weierstrass}
    The statement of \Cref{lem: key lemma vdP points} holds for $f_{0}=1$. 
\end{lemma}\todo{Proof ok?}
\begin{proof}
    Note that in this case $f_{0}=1$ so $\Ocal_{\Sp(A)}(R_{\Sp(A)}(1|f_{1},\dots,f_{n}))$ is of the form $A\langle f_{1},\dots,f_{n}\rangle$. 

    (a)$\Rightarrow$(b) Suppose $\nu\in R_{\Spa(A,A^{\circ})}(1|f_{1},\dots,f_{n})$ so $\nu(f_{i})\leq 1$ for all $i$. For $g\in A\langle X_{1},\dots,X_{n}\rangle$, we set $\nu_{1}(g)=\lim_{n\to\infty}v_{n}$ where $v_{n}=\nu(a_{n})$ and 
    \begin{equation}\label{eqn: leq equation 1}
        a_{n}=\sum_{|\alpha|\leq n}g_{\alpha}f^{\alpha}.
    \end{equation}
    We observe that the limit $\nu_{1}(g)$ converges as by definition of the Tate algebra $A\langle X_{1},\dots,X_{n}\rangle$ there is an $N\in\NN$ such that for each $\varepsilon\in\RR_{>0}$ we have $|g_{\alpha}|<\varepsilon$ for $|\alpha|\geq N$. This implies that 
    \begin{equation}\label{eqn: leq equation 2}
        \nu\left(\sum_{k<|\alpha|\leq l}g_{\alpha}f^{\alpha}\right)\leq\varepsilon
    \end{equation}
    when $\min\{k,l\}\geq N$. If $\lim\inf_{n\to\infty}v_{n}=0$ in (\ref{eqn: leq equation 1}) then $\lim_{n\to\infty}v_{n}=0$. For any $\varepsilon>0$, we can find $N$ such that (\ref{eqn: leq equation 2}) holds, and the assertion follows. However, if $\varepsilon=\lim\inf_{n\to\infty}|v_{n}|$ is positive, then we can find $N$ such that (\ref{eqn: leq equation 2}) holds and $k>N$ with $|v_{k}|\geq\varepsilon$. For $l\geq k$ we have $\nu(a_{l}-a_{k})<v_{k}=\nu(a_{k})$ so $v_{l}=v_{k}$ and the sequence converges. This gives a continuous extension of $\nu$ to a continuous valuation which is bounded by \Cref{prop: powerbounded commutes with polynomial algebra}. 

    (b)$\Rightarrow$(a),(c) Denote $B=\Ocal_{\Sp(A)}(R_{\Sp(A)}(1|f_{1},\dots,f_{n}))$ and suppose that we have $\mu:A\to \Gamma_{\mu}$ a continuous extension of $\nu$ to an element of $\Spa(B,B^{\circ})$. By \Cref{prop: universality of quotients}, $A$ is dense in $B$ so for any $b\in B$ there is a convergent $(a_{n})_{n=1}^{\infty}$ converging to $b$. By continuity of $\mu$, we have 
    $$\mu(b)=\lim_{n\to\infty}\mu(a_{n})=\lim_{n\to\infty}\nu(a_{n})$$
    as sequences in $\Gamma_{\mu}\cup\{0\}$. The topology of $\Gamma_{\nu}\subseteq\Gamma_{\mu}$ is discrete we have in the case when $\mu(b)\neq0$ that the sequence $\nu(a_{n})$ stabilize for $n$ large. So the values of $\nu$ and $\mu$ coincide giving $\Gamma_{\mu}=\Gamma_{\nu}$. Alternatively, viewing the convergence in $\Gamma_{\nu}\cup\{0\}$, we have that $\mu$ is uniquely determined by $\nu$. This shows that (b) implies (c). Now observe that since the images of $f_{i}$ lie in $B^{\circ}$, $\nu(f_{i})=\mu(f_{i})\leq 1$ so $\nu$ is indeed in $R_{\Spa(A,A^{\circ})}(1|f_{1},\dots,f_{n})$. 
\end{proof}
\begin{lemma}\label{lem: key vdP Laurent}
    The statement of \Cref{lem: key lemma vdP points} holds for $f_{1}=1$. 
\end{lemma}
\begin{proof}
    This is essentially the same as \Cref{lem: key vdP Weierstrass}, using the universal property of the localization \Cref{prop: universality of localizations} instead. 
\end{proof}
\begin{proof}[Proof of \Cref{lem: key lemma vdP points}]
    By induction on the Laurent order on the sieve, it suffices to prove the statement in the case of Weierstrass and Laurent domains, which are precisely the cases treated in \Cref{lem: key vdP Weierstrass,lem: key vdP Laurent}. 
\end{proof}
\appendix 
\section{Explicit Elements of the Habiro Ring \\ (d'apr\`{e}s Garoufalidis-Wheeler)}\label{appdx: explicit elements}
This appendix contains the proof sketch of \Cref{ex: element of Habiro ring}. The interested reader is encouraged to consult \cite{GWClassesHabiroCoh} for further details. 

We seek to show that for $R=\ZZ[T_{1},\dots,T_{d},\frac{1}{1-T_{1}-\dots-T_{d}}]$ that the element 
\begin{equation}\label{eqn: Habiro element}
    \sum_{k_{1},\dots,k_{d}\geq0}\left[\substack{k_{1}+\dots+k_{d} \\ k_{1}\text{ }\dots\text{ }k_{d}}\right]_{q}T_{1}^{k_{1}}\dots T_{d}^{k_{d}}\in\ZZ[q^{\pm}][[\underline{T}]]
\end{equation}
lies in $\Hcal_{(R,\square)}$ where $\square:\ZZ[q^{\pm}][\underline{T}]\to R$ is the obvious map. For this, it sufficse to show that $R^{(m)}[[q-\zeta_{m}]]\subseteq\ZZ[\zeta_{m}][[\underline{T},q-\zeta_{m}]]$. 

We have that 
$$R^{(m)}=\ZZ\left[T_{1},\dots,T_{d},\frac{1}{1-T_{1}^{m}-\dots -T_{d}^{m}}\right]$$
and that the Frobenius gluing is already completely determined by the injectivity $R^{(m)}[[q-\zeta_{m}]]\hookrightarrow\ZZ[\zeta_{m}][[\underline{T},q-\zeta_{m}]]$ as it can be checked after $\underline{T}$-adic completion. Note, furthermore, that 
{\footnotesize
$$\ZZ\left[T_{1},\dots,T_{d},\frac{1}{1-T_{1}^{m}-\dots-T_{d}^{m}}\right]=\QQ\left[T_{1},\dots,T_{d},\frac{1}{1-T_{1}^{m}-\dots-T_{d}^{m}}\right]\bigcap\ZZ[[T_{1},\dots,T_{d}]]$$
\normalsize}as subrings of $\QQ[[T_{1},\dots,T_{d}]]$, so it suffices to verify the statement rationally. Using $q=\zeta_{m}\exp(h)$, we get an isomorphism $\QQ(\zeta_{m})[[q-\zeta_{m}]]\cong\QQ(\zeta_{m})[[h]]$ and seek to develop (\ref{eqn: Habiro element}) as a power series in $h$. Using that 
$$\left[\substack{k_{1}+\dots+k_{d} \\ k_{1}\text{ }\dots\text{ }k_{d}}\right]_{q}=\binom{k_{1}+\dots+k_{d}}{k_{1}\dots k_{d}}\cdot O(h)$$
where $O(h)$ is a power series in $h$ with coefficients in $\QQ[k_{1},\dots,k_{d}]$, we have that each term in the power series expansion in $h$ at $m=1$ is of the form 
\begin{equation}\label{eqn: h series expansion}
    \sum_{k_{1},\dots,k_{d}\geq0}\binom{k_{1}+\dots+k_{d}}{k_{1}\dots k_{d}}P(k_{1},\dots,k_{d})T_{1}^{k_{1}}\dots T_{d}^{k_{d}}.
\end{equation}
We then use the following lemma. 
\begin{lemma}
    Let $P(k_{1},\dots, k_{d})\in\QQ[k_{1},\dots,k_{d}]$ as in (\ref{eqn: h series expansion}) lies in $$R=\QQ\left[T_{1},\dots,T_{d},\frac{1}{1-T_{1}-\dots-T_{d}}\right].$$ 
\end{lemma}
\begin{proof}
    Without loss of generality, we can take $P$ to be a monomial $k_{1}^{a_{1}}\dots k_{d}^{a_{d}}$. We get, up to a constant, that $(\nabla_{1}^{\log})^{a_{1}}\dots(\nabla_{d}^{\log})^{a_{d}}$ of $\frac{1}{1-T_{1}-\dots-T_{d}}$ lies in $R$. 
\end{proof}
More generally the power series expansion at $m$ is given by 
\begin{equation}\label{eqn: h series expansion at m}
    \sum_{k_{1},\dots,k_{d}\geq0}\binom{mk_{1}+\dots+mk_{d}}{mk_{1}\dots mk_{d}}P(k_{1},\dots,k_{d})T_{1}^{mk_{1}}\dots T_{d}^{mk_{d}}
\end{equation}
which by similar arguments can be shown to lie in $R$ as well. 
\section{On Animation}\label{appdx: on animation}
We recall the construction of animation following \cite[\S 5.5.8]{HTT} and as discussed in \Cref{sec: lecture 5} and the proof of \Cref{thm: Wagner no-go}.



The construction of animation, or non-Abelian derived functors, dates back to the construction of the cotangent complex, following Quillen and Illusie. We first recall the Dold-Kan correspondence. 
\begin{theorem}[Dold-Kan Correspondence; {\cite[\href{https://stacks.math.columbia.edu/tag/019G}{Tag 019G}]{stacks-project}}]\label{thm: Dold-Kan}
    Let $\Asf$ be an Abelian (1-)category. There is an equivalence of categories between simplicial objects of $\Asf$ and the subcategory of $\Asf$-chain complexes concentrated in non-negative degree. 
\end{theorem}
We now define the cotangent complex (cf. \cite[\href{https://stacks.math.columbia.edu/tag/08PL}{Tag 08PL}]{stacks-project}). 
\begin{definition}[Cotangent Complex]\label{def: cotangent complex}
    The cotangent complex is the chain complex associated to the simplicial $R$-module 
    $$% https://q.uiver.app/#q=WzAsNCxbMywwLCJcXFpaW1JdIl0sWzIsMCwiXFxaWlxcbGVmdFtcXFpaW1JdXFxyaWdodF0iXSxbMSwwLCJcXFpaXFxsZWZ0W1xcWlpcXGxlZnRbXFxaWltSXVxccmlnaHRdXFxyaWdodF0iXSxbMCwwLCJcXGRvdHMiXSxbMywyLCIiLDEseyJvZmZzZXQiOjN9XSxbMywyLCIiLDEseyJvZmZzZXQiOi0zfV0sWzMsMiwiIiwxLHsib2Zmc2V0IjoxfV0sWzMsMiwiIiwxLHsib2Zmc2V0IjotMX1dLFsyLDFdLFsyLDEsIiIsMSx7Im9mZnNldCI6LTN9XSxbMiwxLCIiLDEseyJvZmZzZXQiOjN9XSxbMSwwLCIiLDEseyJvZmZzZXQiOi0xfV0sWzEsMCwiIiwxLHsib2Zmc2V0IjoxfV1d
    \begin{tikzcd}
        \dots & {\ZZ\left[\ZZ\left[\ZZ[R]\right]\right]} & {\ZZ\left[\ZZ[R]\right]} & {\ZZ[R]}
        \arrow[shift right=3, from=1-1, to=1-2]
        \arrow[shift left=3, from=1-1, to=1-2]
        \arrow[shift right, from=1-1, to=1-2]
        \arrow[shift left, from=1-1, to=1-2]
        \arrow[from=1-2, to=1-3]
        \arrow[shift left=3, from=1-2, to=1-3]
        \arrow[shift right=3, from=1-2, to=1-3]
        \arrow[shift left, from=1-3, to=1-4]
        \arrow[shift right, from=1-3, to=1-4]
    \end{tikzcd}$$
    with degeneracy maps augumentation maps under the Dold-Kan correspondence. 
\end{definition}
This construction is motivated by the fact that on smooth (in particular polynomial) $\ZZ$-algebras $R$, the cotangent complex is given by $\LL_{R/\ZZ}\cong\Omega^{1}_{R/\ZZ}[0]$ which has an explicit description in terms of generators and relations. 

More generally, the analogy between derived functors and their non-Abelian counterpart arises from the fact that in both situations one passes to ``free resolutions,'' applies the functor there, and shows the resulting construction is independent of the choices made. 

Before defining animation, we make the following recollections. 
\begin{definition}[Compact Projective Objects]\label{def: compact projective objects}
    Let $\Csf$ be a category with 1-sifted colimits. An object $X\in\Csf$ is compact projective if $\Hom_{\Csf}(X,-)$ commutes with 1-sifted colimits, or equivalently, with filtered colimits and reflexive coequalizers. 
\end{definition}
\begin{example}\label{ex: compact projectives}
    We consider some examples of categories and their compact projective objects. 
    {\begin{center}
        \begin{tabular}{| c | c |}
        \hline 
        $\Csf$ & $\Csf^{\mathsf{cp}}$ \\\hline
        $\Sets$ & finite sets \\\hline
        $\mathsf{CRing}$ & finitely generated polynomial algebras over $\ZZ$ and retracts thereof \\\hline
        $\AbGrp$ & free finitely generated Abelian groups \\\hline
        $\Grp$ & free groups on finitely many generators\\\hline
    \end{tabular}
    \end{center}}
\end{example}
Denoting the full subcategory of $\Csf$ spanned by the compact projective objects $\Csf^{\mathsf{cp}}$, we have by the Yoneda embedding fully faithful embeddings 
$$\Csf^{\mathsf{cp}}\hookrightarrow\mathsf{sInd}^{1}(\Csf^{\mathsf{cp}})\hookrightarrow\Csf$$
where $\mathsf{sInd}^{1}(-)$ is the closure under 1-sifted colimits. 
\begin{definition}[Category Generated by Compact Projectives]\label{def: CPG category}
    Let $\Csf$ be a category with all colimits. $\Csf$ is generated by compact projective objects if there is an equivalence of categories $\mathsf{sInd}^{1}(\Csf^{\mathsf{cp}})\xrightarrow{\sim}\Csf$. 
\end{definition}
We are now ready to define animation of a category in earnest. 
\begin{definition}[Animation of a Category]\label{def: animation of a category}
    Let $\Csf$ be a category admitting all colimits. The animation $\Ani(\Csf)$ of $\Csf$ is the $\infty$-category $\mathsf{sInd}(\Csf^{\mathsf{cp}})$, the closure of (the nerve of) $\Csf^{\mathsf{cp}}$ under $\infty$-categorical sifted colimits. 
\end{definition}
In other words, $\Ani(\Csf)$ is a cocomplete $\infty$-category generated under filtered colimits and geometric realizations of simplicial objects by $\Csf^{\mathsf{cp}}$, or as the localization of the category of simplicial objects of $\Csf$ at the weak equivalences. The animation $\Ani(\Csf)$ of $\Csf$ possesses the universal property that it is initial object in the (non-full) subcategory of cocomplete $\infty$-categories under (the nerve of) $\Csf^{\mathsf{cp}}$ with sifted-colimit preserving functors \cite[\S 5.1.4]{PurityFlatCohomology}. 
\begin{example}\label{ex: animations}
    We consider some examples of categories and their animations.\marginpar{The instructor states that the plural of anima is anima, not animae, despite this being \href{https://logeion.uchicago.edu/morpho/animae}{gramatically incorrect}. The author, having previoiusly studied a classical language, remains insistent on adhering to the correct grammatical conventions.} 
        \begin{center}
        \begin{tabular}{| c | c |}
        \hline 
        $\Csf$ & $\Ani(\Csf)$ \\\hline
        $\Sets$ & $\Ani$, the category of animae or $\infty$-groupoids \\\hline
        $\mathsf{CRing}$ & $\Ani(\Ring)$ the category of animated (commutative) rings \\\hline
        $\AbGrp$ & complexes of Abelian groups conc. in deg. $\geq0$, ie. $\Dscr_{\geq0}(\ZZ)$ \\\hline
        $\Asf$ any Abelian cat. & $\Dscr_{\geq0}(\Asf)$, once again by Dold-Kan \\\hline
    \end{tabular}
    \end{center}
\end{example}
The construction of animation is furthermore functorial, and we can form the animation of functors. For $F:\Csf\to\Dsf$ a 1-sifted colimit preserving functor between cocomplete categories generated by compact projectives, restriction yields a functor $F|_{\Csf^{\mathsf{cp}}}:\Csf^{\mathsf{cp}}\to\Dsf$ and thus a functor to $\Ani(\Dsf)$ by composition. This yields a solid diagram 
$$% https://q.uiver.app/#q=WzAsNCxbMCwwLCJcXENzZl57XFxtYXRoc2Z7Y3B9fSJdLFswLDEsIlxcRHNmIl0sWzIsMSwiXFxBbmkoXFxEc2YpIl0sWzIsMCwiXFxBbmkoXFxDc2YpIl0sWzAsMSwiRnxfe1xcQ3NmXntcXG1hdGhzZntjcH19fSIsMl0sWzEsMiwiXFxBbmkoLSkiLDJdLFswLDMsIlxcQW5pKC0pIiwwLHsic3R5bGUiOnsiYm9keSI6eyJuYW1lIjoiZGFzaGVkIn19fV0sWzMsMiwiXFxBbmkoRikiLDAseyJzdHlsZSI6eyJib2R5Ijp7Im5hbWUiOiJkYXNoZWQifX19XSxbMCwyXV0=
\begin{tikzcd}
	{\Csf^{\mathsf{cp}}} && {\Ani(\Csf)} \\
	\Dsf && {\Ani(\Dsf).}
	\arrow["{\Ani(-)}", dashed, from=1-1, to=1-3]
	\arrow["{F|_{\Csf^{\mathsf{cp}}}}"', from=1-1, to=2-1]
	\arrow[from=1-1, to=2-3]
	\arrow["{\Ani(F)}", dashed, from=1-3, to=2-3]
	\arrow["{\Ani(-)}"', from=2-1, to=2-3]
\end{tikzcd}$$
Since the diagonal composition map $\Csf^{\mathsf{cp}}\to\Ani(\Dsf)$ preserves (nerves of) 1-sifted colimits -- the preservation of $\infty$-categorical sifted colimits is vaccuous as the source is (the nerve of) a 1-category -- the dotted factorization exists by the universal property of animation. 
\begin{definition}[Animation of a Functor]\label{def: animation of a functor}
    Let $F:\Csf\to\Dsf$ be a 1-sifted colimit preserving functor between cocomplete categories generated by compact projectives. The animation $\Ani(F)$ of $F$ is the unique sifted colimit preserving functor $\Ani(\Csf)\to\Ani(\Dsf)$ factoring $\Csf^{\mathsf{cp}}\to\Ani(\Dsf)$. 
\end{definition}
\begin{remark}
    Just as in the case of ordinary categories, animated functors do not necessarily compose well. See \cite[Prop. 5.15]{PurityFlatCohomology} for the requisite hypotheses. 
\end{remark}
Let us return to the cotangent complex. 
\begin{example}
    Let $R$ be a ring and $\Omega^{1}_{(-)/\ZZ}:\mathsf{CRing}\to\AbGrp$ the functor taking each commutative ring to its module of K\"{a}hler differentials. The cotagnent complex $\LL_{-/\ZZ}:\Ani(\Ring)\to\Dscr_{\geq0}(\ZZ)$ is the animation of the functor $\Omega^{1}_{(-)/\ZZ}$. 
\end{example}
We conclude with the following result about the animation of rings and polynomial algebras. 
\begin{proposition}\label{prop: animation of rings and poly algebras agree}
    The animation of the inclusion functor of polynomial $\ZZ$-algebras in finitely many variables into the compact projective objects of commutative rings $\iota:\mathsf{Poly}_{\ZZ}\hookrightarrow\mathsf{CRing}^{\mathsf{cp}}$ is an equivalence $\Ani(\iota):\Ani(\mathsf{Poly}_{\ZZ})\to \Ani(\Ring)$. 
\end{proposition}
\begin{proof}
    Every retract of polynomial algebras over $\ZZ$ in finitely many variables is a quotient, and hence a 1-sifted colimit. In particular, the closure of (the nerve of) $\mathsf{Poly}_{\ZZ}$ under sifted $\infty$-categorical colimits coincides with that of the analogous construction for $\mathsf{CRing}^{\mathsf{cp}}$. 
\end{proof}
\printbibliography
\end{document}